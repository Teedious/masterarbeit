Bei der Nutzung von Multithreading in Spielen gibt es zwei prominente Ansätze, zum einen \ac{sot}, in Bezug auf Rendering also die Nutzung eines separaten Threads, der das Rendern übernimmt, und zum anderen den Einsatz einer Jobarchitektur.

Die Idee der Nutzung eines Render-Threads ergibt sich daraus, dass Simulation des Spiels und Anzeige zwei unabhängige Bereiche sind, die also nebenläufig ausgeführt werden können. Da Rendering sehr rechenintensiv ist, ist es also sinnvoll diese \glspl{Aktivitaet} in einen eigenen Thread auszulagern, damit sie den gesamten Frame für die Ausführung nutzen können. Eine Jobarchitektur ermöglicht es, neben den \glspl{Aktivitaet} des Renderings auch die Simulation in nebenläufige \glspl{Aktivitaet} aufzuteilen. 

Da es sich bei der Blocklib, mit knapp $60000$ Zeilen Code und über $900$ Java Dateien, um ein großes bereits bestehendes System handelt, ist es unrealistisch, in kurzer Zeit das gesamte \gls{Programm} so umzustrukturieren, dass es vollständig eine Jobarchitektur nutzt. Diese Umstrukturierung erfordert tiefes Verständnis für jedes zu ändernde System. Zudem müssen viele Bereiche grundlegend geändert werden, um die Abhängigkeiten der Systeme zu verringern beziehungsweise auszuschließen.

Die Blocklib enthält beispielsweise eine Klasse \class{Context}, die als eine Art \gls{singleton}-Implementierung eines \glspl{servicelocator}~\cite[S.~301~ff.]{Nystrom2015} dient. Da \glspl{singleton} ähnlich wie globale Variablen von überall aufgerufen werden können, wird der Code dadurch schwieriger nachvollziehbar~\cite[S.~108]{Nystrom2015}. Dadurch und durch die ebenfalls resultierende Kopplung unterschiedlicher Komponenten würden bei einer naiven Umsetzung der Jobarchitektur unvorhersehbar Wettkampfbedingungen auftreten.
Man erinnere sich, dass Wettkampfbedingungen auftreten, wenn nebenläufige \glspl{Aktivitaet} auf dieselben Ressourcen zugreifen. Durch den \class{Context} ist es nun schwierig, einen Überblick zu haben von wo aus auf welche Ressourcen zugegriffen wird. Die Wahrscheinlichkeit, dass also Zugriffe existieren, die zu Wettkampfbedingungen führen, ist also sehr hoch, solange dabei nicht sorgsam vorgegangen wird.

Die Architektur muss also beachten, dass es zwar ein Jobsystem geben soll, dieses aber nicht ausschließlich genutzt wird. Da der Performancegewinn, in Form von \ac{fps}, bei der Nutzung eines Render-Threads als hoch zu erwarten ist, soll dieser in die Blocklib integriert werden. Um die Anforderungen von Kapitel~\ref{sec:anforderungen} zu erfüllen, wird ein Jobsystem implementiert. Das Design der Architektur bedient sich also sowohl bei dem Konzept \ac{sot} also auch dem Jobsystem.