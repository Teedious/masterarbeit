Das Arbeiten mit Nebenläufigkeit umfasst einen großen Forschungsbereich. Dabei können durch nebenläufige Programmierung Probleme entstehen, die ohne Nebenläufigkeit nicht auftreten. Andererseits bietet sie beispielsweise die Möglichkeit, die Leistung von Programmen durch die Parallelisierung von Berechnungen zu verbessern. Da das Thema sehr umfangreich ist, werden in diesem Kapitel zunächst einige Grundlagen eingeführt, um eine Basis für das Verständnis der in der Arbeit genutzten Begriffe und Konzepte zu bilden. 

Der erste Abschnitt behandelt sogenannte Petri-Netze. Diese sind dazu geeignet, nebenläufige Vorgänge zu modellieren und werden hier genutzt, um Begriffe wie \enquote{Nebenläufigkeit} formal zu definieren. Abschnitt~\ref{sec:nebenlaufigkeit} führt grundlegende Begriffe des Themas Nebenläufigkeit ein, darunter \enquote{Thread}, \enquote{Kontextwechsel}, \enquote{Synchronisierung} und \enquote{Wettkampfbedingung}. Im darauffolgenden Abschnitt~\ref{sec:nebenlaufigkeitjava} wird beschrieben, wie nebenläufige Konzepte in der Programmiersprache Java\footnote{\url{https://dev.java/}} umgesetzt sind. In Abschnitt~\ref{sec:multigame} wird schließlich erläutert, welche typischen Architekturen in der Spielindustrie eingesetzt werden, um Nebenläufigkeit in Computerspielen effizient nutzen zu können.