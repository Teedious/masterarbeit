Nebenläufigkeit umfasst einen großen Forschungsbereich und führt in der Programmierung zu Problemen, die ansonsten nicht auftreten können. Andererseits bietet sie auch die Möglichkeit die Performance von Programmen enorm zu verbessern. Da das Thema sehr umfangreich ist, werden in diesem Kapitel werden zunächst einige Grundlagen eingeführt, um eine Basis für das Verständnis der in der Arbeit genutzten Begriffe und Konzepte zu bilden. 

Der erste Abschnitt behandelt sogenannte Petri-Netze. Diese sind dazu geeignet nebenläufige Vorgänge zu modellieren und werden hier genutzt um Begriffe wie \enquote{Nebenläufigkeit} formal zu definieren. Wenn das Verständnis der formalen Definitionen nicht nötig ist oder Petri-Netze bereits bekannt sind, kann der folgende Abschnitt übersprungen und direkt in Abschnitt~\ref{sec:nebenlaufigkeit} weitergelesen werden. Dieser Abschnitt führt grundlegende Begriffe des Themas Nebenläufigkeit ein, darunter \enquote{Thread}, \enquote{Kontextwechsel}, \enquote{Synchronisierung} und \enquote{Wettkampfbedingung}. Im darauffolgenden Abschnitt~\ref{sec:nebenlaufigkeitjava} wird beschrieben, wie Konzepte des Themas Nebenläufigkeit in der Programmiersprache Java\footnote{\url{https://dev.java/}} umgesetzt sind. In Abschnitt~\ref{sec:multigame} wird schließlich erläutert, welche typischen Architekturen in der Spielindustrie eingesetzt werden, um Nebenläufigkeit in Computerspielen effizient nutzen zu können.