Nachdem nunmehr die formalen Grundlagen eingeführt worden sind, werden in diesem Abschnitt Grundbegriffe des Bereichs Nebenläufigkeit erläutert, die notwendig für das Verständnis der nächsten Kapitel sind. Besonders die Begriffe \enquote{Thread}, \enquote{Nebenläufigkeit} und \enquote{Wettkampfbedingung} sind wichtig, da diese in der Arbeit intensiv behandelt werden.


\subsubsection{Definition von Thread}
Für die Definition des Begriffs Thread müssen zuerst einige andere Grundbegriffe eingeführt werden. 
Als \gls{Programm} wird in dieser Arbeit die konkrete Niederschrift eines Algorithmus bezeichnet. Die Elemente, aus denen das \gls{Programm} besteht, werden als \glspl{Anweisung} bezeichnet. Bei der Ausführung eines \glsuseri{Programm} werden Einzelschritte durchlaufen, diese bezeichnet man als \glspl{Aktivitaet}. \glspl{Aktivitaet} sind also die Handlungen, die aufgrund der \glspl{Anweisung} des Programms ausgeführt werden. Eine Sequenz von \glspl{Aktivitaet}, die ein isoliertes Problem abarbeitet, wird \emph{Prozess} genannt. Um diese Definition des Prozessbegriffs von späteren Definitionen abzugrenzen, wird er im Folgenden \gls{Rechenprozess} genannt. Die Ausführungseinheit, auf der die Schritte eines \glsuseri{Rechenprozess} durchgeführt werden, wird \emph{Prozessor} genannt~\cite[S.~22]{Herrtwich1989}. In der Regel besitzt ein Rechner mehrere Ausführungseinheiten. Häufig gibt es dabei eine Recheneinheit, die üblicherweise die meisten \glspl{Rechenprozess} ausführt und andere Prozessoren koordiniert, den \ac{cpu}.

Wenn ein \gls{Programm} auf einem Betriebssystem ausgeführt wird, erzeugt das Betriebssystem einen isolierten Adressraum, in dem das \gls{Programm} ausgeführt wird. Ein \gls{Rechenprozess}, der auf diese Weise ausgeführt wird, wird in dieser Arbeit als \emph{(System-)Prozess} bezeichnet. \textcite[S.~125~\psqq]{Tanenbaum2016} liefern eine gute Übersicht über Systemprozesse, dort als Prozesse bezeichnet. Systemprozesse können selbst weitere Systemprozesse starten, wenn das Betriebssystem dies zulässt. Diese Systemprozesse besitzen dann ihren eigenen Adressraum. Die Anzahl der Systemprozesse, die auf einem Rechner laufen, ist meist höher als die Anzahl der Prozessoren des Rechners. Somit können nicht alle Prozesse zur gleichen Zeit ausgeführt werden. Damit dennoch alle Prozesse voranschreiten können, wechseln moderne Betriebssysteme die Systemprozesse, die auf den Prozessoren des Rechners ausgeführt werden, in schneller Folge. Dabei muss das Betriebssystem einige zu den Systemprozessen gehörende Daten, den \emph{Prozesskontrollblock}, tauschen, sodass der jeweils gerade ausführende Systemprozess seinen eigenen Prozesskontrollblock zur Verfügung hat. Dieser Tausch wird \emph{Kontextwechsel} genannt~\cite[S.~59]{Tanenbaum2016}.

Das Speichern und Laden der Prozesskontrollblöcke im Rahmen der Kontextwechsel von Systemprozessen benötigt eine nicht zu vernachlässigende Menge an Zeit. Dieser Zeitverbrauch wird auch als \emph{Overhead} bezeichnet. Moderne Betriebssysteme bieten zur Verminderung des Overheads durch Kontextwechsel die Möglichkeit, \glspl{Rechenprozess} zu erzeugen, die denselben Adressraum wie der erzeugende Systemprozess besitzen. Ein auf diese Art erzeugter \gls{Rechenprozess} heißt \emph{Thread}. Da die Menge der thread-eigenen Daten, des \emph{Threadkontrollblocks}, deutlich geringer ist als die des Prozesskontrollblocks, erzeugt der Kontextwechsel zwischen Threads einen geringeren Overhead. Zudem können Threads aufgrund des gemeinsamen Adressraums einfacher auf geteilte Ressourcen zugreifen. Das vereinfacht die Kooperation zwischen diesen \glsuserii{Rechenprozess}~\cite[S.~139~\psqq]{Tanenbaum2016}.

\paragraph{Anweisungen und Petri-Netze}
Um formal Eigenschaften eines \glsuseri{Programm} zu beschreiben, ist es möglich, dieses mittels eines Petri-Netzes (automatisiert) zu modellieren. Dazu müssen Petri-Netz-Konstrukte genutzt werden, die die \glspl{Anweisung} des \glsuseri{Programm} abbilden können. Für die Analyse nebenläufiger \glspl{Programm} sind nur sogenannte \emph{Rendezvous}-\glspl{Anweisung} sowie Kontrollstrukturen relevant~\cite{Goel1990}. Für diese lassen sich Unter-Petri-Netze definieren, die deren Verhalten modellieren. Eine Anweisung entspricht dann einem bestimmten Unter-Petri-Netz. Eine Auflistung für die Modellierung nebenläufiger \glspl{Programm} benötigter Unter-Petri-Netze ist in \cite{Goel1990} zu finden. Da die Modellierung von Verzweigungen durch Unter-Petri-Netze in \cite[Abbildung 3.1]{Goel1990} (siehe Abbildung~\ref{fig:supnetifelse:old}) nach Auffassung des Autors dieser Arbeit semantisch nicht vollständig präzise ist, wird eine Präzisierung in Abbildung~\ref{fig:supnetifelse} dargestellt. 
\begin{figure}
	\begin{subfigure}[t]{.49\textwidth}
		\centering
		\tikzset{external/export next=false}
		\begin{tikzpicture}[node distance=1.34cm,on grid, auto]
			\node[place, label=$p_1$] (p1) {};
			\node[transV, label=\texttt{if}, above right = of p1] (if) {};
			\node[transV, label=\texttt{else},below right = of p1] (else) {};
			\node[place, label=$p_2$, right = of if] (p2) {};
			\node[place, label=$p_3$, right = of else] (p3) {};
			\node[right = of p2] (dots1){$\dots$};
			\node[right = of p3] (dots2){$\dots$};
			\node[place, label=$p_4$, right = of dots1] (p4) {};
			\node[place, label=$p_5$, right = of dots2] (p5) {};
			\node[transV, label={[xshift=2mm]\texttt{end if}}, below right = of p4] (endif) {};
		
			\draw 
			(p1) edge[post] (if)
			(p1) edge[post] (else)
			(if) edge[post] (p2)
			(else) edge[post] (p3)
			(p2) edge[post] (dots1)
			(p3) edge[post] (dots2)
			(dots1) edge[post] (p4)
			(dots2) edge[post] (p5)
			(p4) edge[post] (endif)
			(p5) edge[post] (endif)
			;
		\end{tikzpicture}
		\subcaption{Nachbildung des Unter-Petri-Netzes aus \cite[Abbildung~3.1]{Goel1990}. Da sowohl $p_4$ als auch $p_5$ Vorbedingungen für \texttt{end if} sind, kann diese Transition nicht feuern, nachdem der \texttt{if}- oder \texttt{else}-Pfad durchlaufen wurden.}\label{fig:supnetifelse:old}
	\end{subfigure}
	\hfill
	\begin{subfigure}[t]{.49\textwidth}
		\centering
		\begin{tikzpicture}[node distance=1.34cm,on grid, auto]
			\node[place, label=$p_1$] (p1) {};
			\node[transV, label=\texttt{if}, above right = of p1] (if) {};
			\node[transV, label=\texttt{else},below right = of p1] (else) {};
			\node[right = of if] (dots1){$\dots$};
			\node[right = of else] (dots2){$\dots$};
			\node[place, label=$p_2$, right = of dots1] (p2) {};
			\node[place, label=$p_3$, right = of dots2] (p3) {};
			\node[transV, label=$t_1$, right = of p2] (t1) {};
			\node[transV, label=$t_2$, right = of p3] (t2) {};
			\node[place, label=$p_4$, below right = of t1] (p4) {};
			\node[transV, label=\texttt{end if}, right = of p4] (endif) {};
		
		
		
			\draw 
			(p1) edge[post] (if)
			(p1) edge[post] (else)
			(if) edge[post] (dots1)
			(else) edge[post] (dots2)
			(dots1) edge[post] (p2) 
			(dots2) edge[post] (p3) 
			(p2) edge[post] (t1)
			(p3) edge[post] (t2)
			(t1) edge[post] (p4)
			(t2) edge[post] (p4)
			(p4) edge[post] (endif)
			;
		\end{tikzpicture}
		\subcaption{Semantisch korrigiertes Unter-Petri-Netz. Die Transition \texttt{end if} besitzt nur die Vorbedingung $p_4$. Folglich kann sie feuern, nachdem einer der beiden Pfade durchlaufen wurde.}\label{fig:supnetifelse:new}
	\end{subfigure}

	\caption[Semantische Präzisierung des Unter-Petri-Netzes zur Modellierung von Verzweigungen.]{Semantische Präzisierung des Unter-Petri-Netzes zur Modellierung von Verzweigungen nach \cite[Abbildung~3.1]{Goel1990}. In (a) wird das ursprüngliche Unter-Petri-Netz gezeigt, in (b) das präzisierte.}\label{fig:supnetifelse}
\end{figure}

Die in Abbildung~\ref{fig:supnetifelse:new} gezeigten Plätze $p_1,p_2,p_3,p_4$ und Transitionen $t_1,t_2$ sind Hilfselemente, die die semantische Korrektheit der Transitionen \code{if}, \code{else} und \code{end if} sicherstellen. Eine Markierung kann exklusiv nur von \code{if} oder von \code{else} zum Feuern verwendet werden. Durch die Transitionen $t_1$ und $t_2$ entsteht eine Markierung in $p_5$, wodurch \code{end if} feuern kann, egal ob anfangs \code{if} oder \code{else} gefeuert hat. $p_2$ und $p_3$ existierten, damit das Petri-Netz ein bipartiter Graph bleibt, sich also immer Plätze und Transitionen \enquote{abwechseln}. Die Modellierung eines \code{switch}-Statements erfolgt analog, indem die Anzahl der Pfade zwischen $p_1$ und $p_4$ angepasst wird.

\subsubsection{Nebenläufigkeit}\label{sec:nebenl}
Im vorherigen Abschnitt wurde beschrieben, dass verschiedene \glspl{Rechenprozess} unabhängig voneinander \enquote{gleichzeitig} ablaufen können. Im Folgenden wird \enquote{gleichzeitig} präziser definiert. Ein \gls{Programm} wird häufig als \emph{Sequenz} von \glspl{Anweisung} verstanden. In der Definition des Begriffs \enquote{\gls{Programm}} in dieser Arbeit wird bewusst auf das Wort \enquote{Sequenz} verzichtet, denn bei vielen \glsuserii{Programm} kann es bei bestimmten \glspl{Anweisung} irrelevant sein, in welcher Reihenfolge oder ob sie sogar gleichzeitig ausgeführt werden. Man betrachte das folgende \gls{Programm} in Listing~\ref{lst:squareSeqEx}, das die Quadratsumme zweier Ganzzahlen (im Folgenden auch Quadratsumme genannt) ausgibt.

\begin{lstlisting}[caption={[Beispiel eines \glsentryuseri{Programm}, das sequentiell die Summe von Quadraten zweier Ganzzahlen berechnet.]Beispiel eines \glsuseri{Programm} das die Summe von Quadraten zweier Ganzzahlen berechnet. Die Berechnung der Quadratzahlen wird nacheinander in einer fest definierten Sequenz durchgeführt.}, label={lst:squareSeqEx},float={!htbp}]
printSummedSquare(int a, int b){
  x <- a*a (*\label{lst:squareSeqEx:aa}*)
  y <- b*b (*\label{lst:squareSeqEx:bb}*)
  print(x+y) (*\label{lst:squareSeqEx:print}*)
}
\end{lstlisting}
Es ist leicht zu erkennen, dass es irrelevant ist, ob zuerst \code{x} oder \code{y} berechnet wird oder die Berechnungen simultan stattfinden. Einzig der Ausgabebefehl in Zeile~\ref{lst:squareSeqEx:print} darf erst ausgeführt werden, nachdem \code{x} und \code{y} ermittelt wurden. Zuzulassen, dass die Ausführreihenfolge in dem \gls{Programm} nicht definiert ist, führt dazu, dass das \gls{Programm} präziser der Problembeschreibung \enquote{gib die Quadratsumme zweier Ganzzahlen aus} entspricht, und ermöglicht auch eine potenziell schnellere Berechnung, da die Quadrate gleichzeitig berechnet werden können.

Paare oder Gruppen von \glspl{Anweisung}, die gleichzeitig oder in beliebiger Reihenfolge ausgeführt werden, heißen \emph{nebenläufig}. Da die Ausführungsreihenfolge der Anweisungen in Zeile~\ref{lst:squareSeqEx:aa} und Zeile~\ref{lst:squareSeqEx:bb} irrelevant ist, dürfen diese Anweisungen auch nebenläufig definiert sein.

Eine Implementierung, die obige Problemstellung der Quadratsumme besser abbildet, könnte unter Nutzung von Nebenläufigkeit wie Listing~\ref{lst:squareConcEx} aussehen. Dabei wird auf die Schreibweise von \textcite[S.~16]{Herrtwich1989} zurückgegriffen. In einem \code{conc}-Block (von englisch \emph{concurrent} -- nebenläufig), definiert durch \code{conc} und \code{conc end}, sind alle \glspl{Anweisung}, die durch \code{||} getrennt sind, als nebenläufig zu verstehen. 

\begin{lstlisting}[caption={[Beispiel eines \glsentryuseri{Programm} mit nebenläufigem Code in einem \code{conc}-Block.]Beispiel eines \glsuseri{Programm} mit nebenläufigem Code in einem \code{conc}-Block. Das \gls{Programm} gibt die Summe von zwei Quadratzahlen aus, wobei die Berechnung der Quadratzahlen nebenläufig stattfindet.}, label={lst:squareConcEx},float={!htbp}]
	printSummedSquare(int a, int b){
		conc 
			x <- a*a ||
			y <- b*b
		conc end
		print(x+y)
	}
\end{lstlisting}

Wie man an dem Beispiel in Listing~\ref{lst:squareConcEx} erkennen kann, beschäftigt sich Nebenläufigkeit besonders mit der Struktur und der Möglichkeit der Zusammensetzung voneinander unabhängiger \glspl{Anweisung}. Daher ist es Aufgabe der nebenläufigen Programmierung, Probleme oder Aufgaben in unabhängige Teile zu zerlegen und zu strukturieren~\cite{Pike2012,Hettel2016}. Blöcke von \glspl{Anweisung}, die (bezüglich der Aufgabe) nebenläufig sein könnten, aber als Anweisungssequenz definiert sind (siehe Zeile 2 und 3 in Listing~\ref{lst:squareSeqEx}), werden in dieser Arbeit als \emph{pseudo-sequentialisiert} bezeichnet.

Bei der Definition nebenläufiger Anweisungen ist Vorsicht geboten, da sichergestellt werden muss, dass zwei Anweisungen auch tatsächlich nebenläufig sein dürfen. Eine \gls{Anweisung} kann beispielsweise durch einen Compiler in mehrere \glspl{Anweisung} geteilt werden oder selbst schon aus mehreren \glspl{Anweisung} bestehen (man denke zum Beispiel an Funktionsaufrufe). Solche \glspl{Anweisung} werden \gls{nichtAtomar} genannt. Eine Menge von nebenläufigen \glspl{Anweisung}, von denen mindestens eine \gls{nichtAtomar} ist, kann \emph{verzahnt} ausgeführt werden. Eine Ausführung von \glspl{Anweisung} ist verzahnt, wenn zwischen der Ausführung der Teile einer \glsuseri{nichtAtomar} \gls{Anweisung} andere \glspl{Anweisung} ausgeführt werden. Abbildung \ref{fig:concAnweisungen} zeigt zur Veranschaulichung die verschiedenen Möglichkeiten, wie zwei nebenläufige \glspl{nichtAtomar} \glspl{Anweisung} im Zeitverlauf ausgeführt werden können, sowohl auf einem als auch auf zwei Prozessoren. Das kann zur Folge haben, dass Anweisungen, die zuerst den Anschein haben, nebenläufig sein zu können, nicht nebenläufig sein dürfen, weil eine verzahnte Ausführung der zugehörigen Aktivitäten ein unerwünschtes Ergebnis liefern würde. 

\begin{figure}[hbt]
	\centering
\newlength\aOne
\newlength\aTwo
\newlength\aThree
\newlength\bOne
\newlength\bTwo
\pgfmathsetlength{\aOne}{2cm}
\pgfmathsetlength{\aTwo}{2.25cm}
\pgfmathsetlength{\aThree}{2.5cm}
\pgfmathsetlength{\bOne}{2.75cm}
\pgfmathsetlength{\bTwo}{3cm}
\begin{subfigure}{\textwidth}
	\centering
\begin{tikzpicture}

	\draw[->] (0,0) -- (13,0) node[right] {Zeit};
	
	\node[draw,fill=red, anchor=south west, minimum width = \aOne] at (0, .5) (A1) {A Teil1};

	\node[draw,fill=red, anchor=west, minimum width = \aTwo] at (A1.east) (A2) {A Teil2};
	\node[draw,fill=red, anchor=west, minimum width = \aThree] at (A2.east) (A3) {A Teil3};
	
	\node[draw,fill=cyan, anchor=south west, minimum width = \bOne] at (0, 1.5) (B1) {B Teil1};
	\node[draw,fill=cyan, anchor=west, minimum width = \bTwo] at (B1.east) (B2) {B Teil2};

	\node[anchor=east] at (B1.west) {Prozessor 1};
	\node[anchor=east] at (A1.west) {Prozessor 2};
\end{tikzpicture}
\subcaption{Gleichzeitige Ausführung der \glspl{Anweisung} A und B auf zwei Prozessoren}
\end{subfigure}
\\[1.5em]
\begin{subfigure}{\textwidth}
	\centering
\begin{tikzpicture}

	\draw[->] (0,0) -- (13,0) node[right] {Zeit};
	
	\node[draw,fill=cyan, anchor=south west, minimum width = \bOne] at (0, .5) (B1) {B Teil1};
	\node[draw,fill=cyan, anchor=west, minimum width = \bTwo] at (B1.east) (B2) {B Teil2};
	\node[draw,fill=red, anchor=west, minimum width = \aOne] at (B2.east) (A1) {A Teil1};
	\node[draw,fill=red, anchor=west, minimum width = \aTwo] at (A1.east) (A2) {A Teil2};
	\node[draw,fill=red, anchor=west, minimum width = \aThree] at (A2.east) (A3) {A Teil3};

	\node[anchor=east] at (B1.west) {Prozessor 1};
\end{tikzpicture}
\subcaption{Sequenzielle Ausführung der \glspl{Anweisung} A und B auf einem Prozessor}
\end{subfigure}
\\[1.5em]
\begin{subfigure}{\textwidth}
	\centering
\begin{tikzpicture}

	\draw[->] (0,0) -- (13,0) node[right] {Zeit};
	
	\node[draw,fill=cyan, anchor=south west, minimum width = \bOne] at (0, .5) (B1) {B Teil1};
	\node[draw,fill=red, anchor=west, minimum width = \aOne] at (B1.east) (A1) {A Teil1};
	\node[draw,fill=red, anchor=west, minimum width = \aTwo] at (A1.east) (A2) {A Teil2};
	\node[draw,fill=cyan, anchor=west, minimum width = \bTwo] at (A2.east) (B2) {B Teil2};
	\node[draw,fill=red, anchor=west, minimum width = \aThree] at (B2.east) (A3) {A Teil3};

	\node[anchor=east] at (B1.west) {Prozessor 1};
\end{tikzpicture}
\subcaption{Verzahnte Ausführung der \glspl{Anweisung} A und B auf einem Prozessor}
\end{subfigure}

\caption[Mögliche Ausführungen nebenläufiger \glsentryplural{Anweisung}.]{Mögliche Ausführungen nebenläufiger \glspl{Anweisung} nach~\cite{Herrtwich1989}.}\label{fig:concAnweisungen}
\end{figure}

Werden nebenläufige \glspl{Anweisung} parallel ausgeführt, spricht man auch von \emph{Multiprocessing}. Handelt es sich bei den Prozessen um Threads, wird der Begriff \emph{Multithreading} genutzt.

\paragraph{Nebenläufigkeit und Petri-Netze}
Um eine eindeutige Bedeutung des hier genutzten Begriffs der Nebenläufigkeit festzulegen, wird dieser nun unter Verwendung des Petri-Netz-Formalismus definiert. Es ist wichtig anzumerken, dass der hier beschriebene Begriff von dem üblichen Verständnis der Nebenläufigkeit in Petri-Netzen abweicht. 

Normalerweise werden Petri-Netze als inhärent nebenläufig verstanden, da durch die Transitionsregel zu jeder Zeit jede beliebige Transition feuern kann, deren Vorbedingungen erfüllt sind. Der hier verwendete Begriff bezieht sich allerdings darauf, dass \glspl{Anweisung} unabhängig voneinander ausführbar sind, ohne sich gegenseitig zu beeinflussen. Das Feuern einer Transition kann aber den Vorbereich einer anderen Transition verändern, wodurch sie beeinflusst wird.

In dieser Arbeit heißen zwei Transitionen $t_1 \neq t_2$ eines Petri-Netzes $N$ genau dann nebenläufig, wenn ${}^\circ t_1 \cap {}^\circ t_2 = \emptyset$ gilt und eine Markierung $m_\text{conc}$ im Erreichbarkeitsgraphen $\E{N}$ existiert, in der sowohl $t_1$ als auch $t_2$ aktiviert sind. Die Vorbereiche der Transitionen dürfen sich also nicht überschneiden.

\subsubsection{Folgen von Nebenläufigkeit}\label{sec:nebenl-folgen}
Wie in Abschnitt \vref{sec:nebenl} beschrieben, muss ein \gls{Programm} keine Sequenz von \glspl{Anweisung} darstellen, sondern kann auch nebenläufige \glspl{Anweisung} enthalten. Diese Möglichkeit hat eine Reihe von Folgen, die es bei der nebenläufigen Programmierung zu beachten gilt.
\paragraph{Nichtdeterminismus}
Enthält ein \gls{Programm} nebenläufige \glspl{Anweisung}, ist die Reihenfolge der daraus resultierenden \glspl{Aktivitaet} nicht definiert und kann sich bei jedem Programmdurchlauf ändern. Die Ausführung ist als direkte Folge der Nebenläufigkeit~\cite[S.~17~\psq]{Herrtwich1989} \emph{nichtdeterministisch}. Erwartet man von einem \gls{Programm} \emph{Determiniertheit}\footnote{Es kann durchaus sein, dass Determiniertheit in einem \gls{Programm} nicht gewünscht ist. Man betrachte beispielsweise ein \gls{Programm}, das einen echten Zufallsgenerator beschreibt. Hier wäre Determiniertheit ein direkter Widerspruch zur Aufgabe des \glsuseri{Programm}.}, also die Eigenschaft, dass gleiche Eingaben immer zu den gleichen Ausgaben führen, ist Nichtdeterminismus in der Regel zu vermeiden. Determiniertheit kann aber auch bei einem Verzicht auf Determinismus sichergestellt werden. Liefert ein \gls{Programm} in jeder beliebigen Ausführreihenfolge dasselbe Ergebnis, ist es trotz Nichtdeterminismus weiterhin determiniert, weil die Ausführreihenfolge für das Ergebnis keine Rolle spielt~\cite[S.~18~\psq]{Herrtwich1989}. 
\paragraph{Nichtreproduzierbarkeit}
Die Nachvollziehbarkeit des Programmablaufs wird erschwert, da das Wissen und die Kontrolle über die Ausführreihenfolge abgegeben wird~\cite[S.~20]{Herrtwich1989}. Tritt beispielsweise ein Fehler auf, kann die Ausführreihenfolge, die zu dem Fehler geführt hat, im Nachhinein nicht ermittelt werden. Dasselbe Problem ergibt sich, wenn ein nichtdeterminiertes \gls{Programm} eine Lösung ausgibt und nachvollzogen werden soll, welche Ausführreihenfolge zu diesem Ergebnis geführt hat. Insbesondere das Testen und Debugging von Software wird dadurch erschwert, da es unmöglich ist bei jedem Test dieselben Bedingungen herzustellen, sodass beispielsweise Tests Fehler nicht verlässlich aufzeigen, weil diese nur bei bestimmten Ausführreihenfolgen auftreten~\cite[S.~20]{Herrtwich1989}. Aus diesem Grund muss ein Test entweder alle möglichen Ausführreihenfolgen simulieren oder es muss anderweitig sichergestellt werden, dass die Ausführreihenfolge der nebenläufigen \glspl{Anweisung} keine Rolle spielt.
\paragraph{Wettkampfbedingungen}
Ressourcen (zum Beispiel Drucker, Variablen und Dateien) und insbesondere Daten spielen in \glsuserii{Programm} eine zentrale Rolle. Greift eine Gruppe von nebenläufigen \glspl{Anweisung} auf dieselben (geteilten) Daten zu, ist die Reihenfolge der Zugriffe, wie alle anderen \glspl{Aktivitaet} der nebenläufigen \glspl{Anweisung}, beliebig. Wenn mindestens eine der \glspl{Anweisung} schreibend auf die geteilten Daten zugreift (diese also verändert), hängt der Zustand der Ausgabe des \glsuseri{Programm} im Allgemeinen von der Reihenfolge der Ausführung ab. Diese Situation wird \emph{Wettkampfbedingung} (engl. race condition) genannt~\cite{Hettel2016}. 

Erwähnenswert ist hier, dass Wettkampfbedingungen nur auftreten können, wenn die geteilten Daten nebenläufig \emph{geändert} werden. Ausschließlich lesende Zugriffe sind unkritisch, da die Daten unabhängig von der Ausführreihenfolge immer identisch sind. Wenn Daten von \glspl{Anweisung} jederzeit in einem Zustand aufgefunden werden, der korrekt ist, werden sie als \emph{threadsicher} bezeichnet.

Formal können Wettkampfbedingungen auch mittels erweiterter Petri-Netze definiert werden. Dazu betrachtet man, ob Variablen des erweiterten Petri-Netzes sich in den Lese- und Schreibmengen von nebenläufigen Transitionen überschneiden. Diese Variablen werden \emph{kritische} Variablen genannt. Die Menge der kritischen Variablen, die zur Existenz von Wettkampfbedingungen führen, ist formal gegeben durch
\begin{align*}
	V_\text{krit} = \;\bigcup_{\mathclap{\substack{s,t\, \in T\\s \neq t\\s, t \text{ nebenläufig}}}} \;\left( \strut{S(s) \cap (S(t) \cup L(t))}\right).
\end{align*}

Diese Definition ergibt sich wie folgt. Es werden alle Paare von Transitionen $s$ und $t$ betrachtet, wobei $s$ und $t$ nebenläufig und (deswegen) nicht identisch sind. Dann werden der Menge $V_\text{krit}$ die Variablen hinzugefügt, die in der Schreibmenge der einen Transition $S(s)$ und ebenfalls in der Schreib- oder Lesemenge der anderen Transition $(S(t) \cup L(t))$ enthalten sind.

Ein Paar von Transitionen $(t_1, t_2)$ eines erweiterten Petri-Netzes \emph{erzeugt} eine Wettkampfbedingung, wenn die Transitionen dazu führen, dass eine Variable in die Menge der kritischen Variablen aufgenommen wird. Eine Markierung $m$ des Erreichbarkeitsgraphen $\E{N}$ eines Petri-Netzes $N$ enthält eine Wettkampfbedingung, wenn ein Paar von Transitionen $(t_1, t_2)$ existiert, das eine Wettkampfbedingung erzeugt, und $t_1$ und $t_2$ in $m$ aktiviert sind.

\begin{figure}
	\centering
	\begin{tikzpicture}[node distance=2cm,on grid, auto]
		\node[place, label=$p_1$] (p1) {};
		\node[transV, label=$t_1$, right = of p1] (t1) {};
		\node[place, label=$p_2$, tokens=1, above right = of t1] (p2) {};
		\node[place, label=$p_3$, tokens=1, below right = of t1] (p3) {};
		\node[transV,label=$t_2$, right = of p2,label=below:{\LSset{a,{\color{red}b}}{a}} ] (t2){};
		\node[transV,label=$t_3$, right = of p3,label=below:{\LSset{b}{{\color{red}b}}}] (t3){};
		\node[place, label=$p_4$, right = of t2] (p4) {};
		\node[place, label=$p_5$, right = of t3] (p5) {};
		\node[transV, label=$t_4$, below right = of p4] (t4) {};
		\node[place, label=$p_6$, right = of t4] (p6) {};
	
	
	
		\draw 
		(p1) edge[post] (t1)
		(t1) edge[post] (p2)
		(t1) edge[post] (p3)
		(p2) edge[post] (t2)
		(p3) edge[post] (t3)
		(t2) edge[post] (p4)
		(t3) edge[post] (p5)
		(p4) edge[post] (t4)
		(p5) edge[post] (t4)
		(t4) edge[post] (p6)
		;
	\end{tikzpicture}
	\caption[Ein erweitertes Petri-Netz, dessen Markierung eine Wettkampfbedingung enthält.]{Ein erweitertes Petri-Netz, dessen Markierung eine Wettkampfbedingung enthält. Die Schreibmenge von $t_3$ enthält die Variable $b$, diese ist allerdings auch in der Lesemenge von $t_2$ enthalten. Die problematische Variable {\color{red}$b$} ist rot markiert. Das Auftauchen von $b$ in der Lesemenge von $t_3$ stellt kein Problem dar.}\label{fig:wettkampfpetri}
\end{figure}

Zur Veranschaulichung des Konzepts ist in Abbildung~\ref{fig:wettkampfpetri} ein erweitertes Petri-Netz gezeigt, dessen Markierung eine Wettkampfbedingung enthält. Die Plätze $p_2$ und $p_3$ enthalten je eine Markierung. Somit müssen nun die Transitionen betrachtet werden, die aktiviert sind. Diese sind $t_2$ und $t_3$. Wie in der Abbildung rot markiert ist, überschneiden sich die Lese- und Schreibmengen der Transitionen, weil beide die Variable $b$ enthalten. Dadurch gilt $V_\text{krit} = \{b\} \neq \emptyset$ und die gezeigte Markierung enthält eine Wettkampfbedingung.

Ein erweitertes Petri-Netz enthält eine Wettkampfbedingung, wenn mindestens eine Markierung seines Erreichbarkeitsgraphen eine Wettkampfbedingung enthält.

Aufgabe der nebenläufigen Programmierung ist es also, \glspl{Anweisung} so zu strukturieren, dass alle von nebenläufigen \glspl{Anweisung} genutzten Daten threadsicher sind oder Daten, die nicht threadsicher sind, explizit synchronisiert werden. Über die Modellierung von Petri-Netzen ausgedrückt, gilt es also die Struktur eines \glsuseri{Programm} so zu definieren, dass in dem Petri-Netz, welches das \gls{Programm} modelliert, keine Wettkampfbedingungen existieren.

\subsubsection{Synchronisierung}
Um das Auftreten von Wettkampfbedingungen beim schreibenden Zugriff von nebenläufigen \glspl{Anweisung} auf geteilte Ressourcen zu vermeiden, muss sichergestellt werden, dass der Endzustand nach der Ausführung der \glspl{Anweisung} unabhängig von der Ausführreihenfolge der daraus resultierenden \glspl{Aktivitaet} ist. Dieser Vorgang wird \emph{Synchronisierung} genannt~\cite[S.~4]{Maurer2019}.

\paragraph{Synchronisierungsanforderungen} Nach \textcite[S.~132~\psqq]{Herrtwich1989} gibt es allgemein zwei Arten von Anforderungen an Synchronisierungsmechanismen, je nachdem, welche Art von Zugriff auf geteilte Ressourcen stattfindet. Sie unterscheiden dabei zwischen \emph{kausal abhängigen} Relationen und \emph{kausal unabhängigen} Relationen zwischen nebenläufigen \glspl{Anweisung}. Ist \gls{Anweisung} $B$ kausal abhängig von \gls{Anweisung} $A$, so dürfen die \glspl{Aktivitaet} von $B$ erst ausgeführt werden, nachdem die \glspl{Aktivitaet} von $A$ abgeschlossen sind. Beispielsweise werden in der Blocklib Chunks im Hintergrund geladen. Benötigt eine \gls{Anweisung} eines Spielelements einen noch nicht geladenen Chunk, so muss die Ausführung der zugehörigen \gls{Aktivitaet} solange aufgeschoben werden, bis der Chunk geladen ist. Bei kausal unabhängigen Relationen muss sichergestellt werden, dass der Zugriff auf die Ressourcen nicht zur gleichen Zeit stattfindet.  Beispielsweise wird die Liste der zu zeichnenden Objekte in der Blocklib von verschiedenen Systemen angepasst. Hier muss also darauf geachtet werden, dass diese Anpassungen zeitlich getrennt stattfinden. Sonst könnten beispielsweise hinzugefügte Elemente verloren gehen, wenn zwei Threads parallel Elemente hinzufügen, weil ein Thread möglicherweise die Änderungen des anderen überschreibt.

\paragraph{Synchronisierungsmethoden} Es gibt mehrere Möglichkeiten, die Synchronisierung zwischen nebenläufigen \glspl{Anweisung} zu erreichen. \textcite{Michael1996} unterteilen Algorithmen diesbezüglich in zwei Kategorien: \emph{blockierende} und \emph{nicht-blockierende}. Blockierende Synchronisierungsalgorithmen definieren sogenannte \emph{kritische Bereiche}. Ein kritischer Bereich ist eine Menge von \glspl{Anweisung}, von deren \glspl{Aktivitaet} nur eine gleichzeitig ausgeführt werden darf. Sind mehrere auszuführende \glspl{Aktivitaet} Teil eines kritischen Bereichs, muss durch \emph{Sperren} (engl. Locks) sichergestellt werden, dass nur eine der \glspl{Aktivitaet} gleichzeitig ausgeführt wird. Möchte ein \gls{Rechenprozess} einen kritischen Bereich betreten, muss er sich die Sperre zunächst aneignen. Ist die Sperre bereits im Besitz eines anderen \glsuseri{Rechenprozess}, muss der \gls{Rechenprozess} warten, bis seine Aneignung der Sperre erfolgreich ist. Dadurch führen solche Synchronisierungsmethoden laut \textcite{Michael1996} möglicherweise dazu, dass \glspl{Rechenprozess} durch andere \glspl{Rechenprozess} beliebig lang beim Zugriff auf geteilte Daten aufgehalten werden können.

Nicht-blockierende Algorithmen dagegen garantieren, dass mindestens einer der beteiligten \glspl{Rechenprozess} mit einer endlichen Anzahl an Schritten endet. Zudem konnte \textcite{Herlihy1991} bereits 1991 zeigen, dass beispielsweise unter Verwendung von atomaren \emph{Compare-and-Swap}-Instruktionen Datenstrukturen implementiert werden können, auf die nicht-blockierend von einer unbegrenzten Zahl an nebenläufigen \glspl{Aktivitaet} zugegriffen werden kann. Durch die Atomarität der genutzten \glspl{Anweisung} wird sichergestellt, dass die geteilten Ressourcen nicht in einen unvorhergesehenen Zustand übergehen. Die eben genannte Instruktion Compare-and-Swap ist eine \gls{Anweisung}, in der atomar der Wert einer Variablen ausgelesen und mit einem übergebenen Wert verglichen wird (Compare). Danach wird, weiterhin atomar, der Inhalt der Variablen mit einem zweiten übergebenen Wert genau dann überschrieben (Swap), wenn die verglichenen Werte identisch waren. Die Instruktion gibt dann den vorherigen Wert zurück. Eine andere Variante der Instruktion (\emph{Compare-and-Set}) gibt zurück, ob der Wert überschrieben wurde oder nicht. Die Atomarität dieser gesamten Instruktion muss durch die Hardware sichergestellt werden, das ist softwareseitig nicht möglich. Alle gängigen \acp{cpu} stellen Instruktionen dieser Art zur Verfügung.
