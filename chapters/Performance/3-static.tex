Neben den beiden selbst erzeugten Szenarien, werden nun Szenarien betrachtet, in denen die Terrain-Generierung der Blocklib genutzt wird. Für die Generierung nutzt die Blocklib einen Pseudo-Zufallszahlengenerator. Das ist ein Algorithmus, der ausgehend von einem übergebenen Wert, dem sogenannten \emph{Seed} Werte erzeugt, die zufällig \emph{wirken}. Der Algorithmus ist allerdings deterministisch. Mit dem selben Seed wird immer die selbe Folge von Zahlen generiert.

\begin{figure}
	\centering
	\includegraphics[width=.49\textwidth]{seed-0.png}
	\hfill
	\includegraphics[width=.49\textwidth]{seed-2.png}
	\caption{Links: Seed 0. Rechts: Seed 2}\label{fig:static}
\end{figure}
Damit SystemA und SystemB bei Messungen mit Terrain-Generierung vergleichbar sind, wird in beiden Systemen der selbe Seed verwendet, wodurch die selbe Welt generiert wird. Abbildung~\ref{fig:static} zeigt zwei Screenshots von generierten Welten. Das linke Bild zeigt die mit dem Seed $0$ generierte Welt, das rechte Bild nutzt als Seed $2$. Hier sieht man beispielhaft, dass bereits kleine Unterschiede im Seed völlig andere Welten erzeugen. Für die folgenden Messungen wird der Seed $0$ verwendet. 

Im Szenario Welt-Statisch bleibt die Kamera wie in den vorherigen beiden Szenarien an ein und der selben Stelle.



\paragraph{\ac{fps}}
\begin{figure}[!htbp]
	\fpsplot{seed-0-static}
	\caption{Seed 0 Statisch}\label{fig:seed-0-static-fps}
\end{figure}
Der Verlauf der \ac{fps} in diesem Szenario ist in Abbildung~\ref{fig:seed-0-static-fps} abgebildet. Vergleicht man die Framerate mit denen der vorherigen Szenarien, lässt sich erkennen, dass die Startphase sich in beiden Systemen verlängert. SystemA benötigt \SI{20}{\second} statt \SI{15}{\second}, um eine konstante Framerate von etwa \SI{201}{\fps} zu erreichen. SystemB erreicht einen stabilen Zustand nach etwa \SI{22}{\second} mit durchschnittlich \SI{468}{\fps}.
Beide Systeme benötigen mit Terrain-Generierung also etwa \SI{5}{\second} länger für den Start. SystemB erreicht nach dem Start eine um \SI{133}{\percent} gesteigerte Framerate gegenüber SystemA. Dieser Wert liegt zwischen den gemessenen Werten der ersten beiden Szenarien.

\paragraph{CPU}
Unter Betrachtung der CPU-Auslastung im Welt-Statisch Szenario, die in Abbildung~\ref{fig:seed-0-static-cpu} zu sehen ist, lässt sich der zusätzliche Aufwand für die Terrain-Generierung der Welt ebenfalls erkennen.
\begin{figure}[!htbp]
	\cpuplot{seed-0-static}
	\caption{Seed 0 Statisch}\label{fig:seed-0-static-cpu}
\end{figure}
Während des Starts der Blocklib lasten beide System die CPU bis zu \SI{97}{\percent} aus. Nach dem Start fällt die Auslastung in beiden Systemen, auf durchschnittlich \SI{13}{\percent} in SystemA und \SI{23}{\percent} in SystemB.
SystemB führt im Vergleich zu SystemA in diesem Szenario also zu einer um \SI{77}{\percent} erhöhten Auslastung der CPU. SystemB lastet die CPU in diesem Szenario mehr aus als in den Szenarien ohne generierter Welt. SystemA dagegen erreicht nach dem Start eine zu den bisherigen Szenarien vergleichbare Auslastung.

\paragraph{GPU}
\begin{figure}[!htbp]
	\gpuplot{seed-0-static}
	\caption{Seed 0 Statisch}\label{fig:seed-0-static-gpu}
\end{figure}

\paragraph{RAM}
\begin{figure}[!htbp]
	\memplot{seed-0-static-single-mem.csv}
	\memplot{seed-0-static-multi-mem.csv}
	\caption{Seed 0 Statisch}\label{fig:seed-0-static-mem}
\end{figure} 
