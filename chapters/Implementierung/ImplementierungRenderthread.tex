Da ein Jobsystem implementiert wird, können alle Simulationsanweisungen auf Job-Threads ausgeführt werden. Damit ist der Thread, der in Java mit dem Start des Programms erzeugt wird, der \code{main}-Thread, unbenutzt und kann als Render-Thread das Rendering ausführen. Alle anderen \glspl{Rechenprozess} werden über das Jobsystem auf anderen Threads durchgeführt. Bei der Implementierung der für einen Render-Thread nötigen Funktionalität müssen in der Blocklib viele verschiedene Klassen angepasst werden. Der Übersichtlichkeit halber werden die Anpassungen in Bereiche eingeteilt. 

In Abschnitt~\ref{sec:loader} wird beschrieben, wie das Laden von Daten auf die \ac{gpu} geändert werden muss, weil dieser Vorgang aus Sicht der Simulation nun wegen der Nebenläufigkeit des Render-Threads asynchron ist. Danach erläutert Abschnitt~\ref{sec:statelessRendering} den Wechsel von einer Render-Architektur, die sich zu zeichnende Elemente merkt, zu einer zustandslosen Architektur. Abschnitt~\ref{sec:saveRenderState} beschreibt, wie darauf aufbauend der Zustand der zu zeichnenden Elemente unter Verwendung von Double-Buffers zwischengespeichert wird, damit er während des Renderings unverändert bleibt und so Pipelining ermöglicht wird. In Abschnitt~\ref{sec:adjustGameLoop} wird beschrieben, wie der Aufbau der Game-Loop an die neue Architektur angepasst wird. Schließlich wird in Abschnitt~\ref{sec:searchRace} anhand eines Beispiels gezeigt, warum eine zentrale Zwischenspeicherung der Render-Daten zur Vermeidung von Wettkampfbedingungen sinnvoll ist.

\subsubsection{Laden von Daten auf die \glsentryuseri{gpu}}\label{sec:loader}
In der Blocklib werden häufig Daten an die \ac{gpu} übergeben. Beispielsweise muss bei der Generierung eines Chunks ein Gitternetz konstruiert werden, das die Form des Terrains im Chunk beschreibt, und dann an die \ac{gpu} gesendet werden. Dafür wird die OpenGL \acs{api} genutzt. Alle Aufrufe an OpenGL müssen auf einem Thread ausgeführt werden, die Blocklib wird aber nebenläufig ausgeführt. Daher muss dafür gesorgt werden, dass \glspl{Anweisung} zwischengespeichert werden, die mit der \ac{gpu} interagieren, aber zum Beispiel durch die Chunk-Generierung angestoßen werden. Die zwischengespeicherten \glspl{Anweisung} können dann von dem Render-Thread gemeinsam ausgelesen und ausgeführt werden. 

Um \glspl{Anweisung} zum Laden von Daten auf die \ac{gpu} zwischenzuspeichern, muss die dafür zuständige Klasse \classLoader{} angepasst werden. Die Klasse wird um eine Warteschlange erweitert, die Objekte des Typs \classRunnable{} enthält. Anstatt die Ladeanweisungen direkt auszuführen, werden sie als Tasks der Warteschlange übergeben und dort als \classRunnable{}-Objekte gespeichert. Die Warteschlange nutzt intern die von Java bereitgestellte Klasse \classConcurrentLinkedQueue{}. Die Warteschlange bietet damit nicht-blockierende Synchronisierung, wodurch mehrere Threads gleichzeitig Elemente der Warteschlange hinzufügen können, ohne dass dies zu Wettkampfbedingungen führen kann. In der Game-Loop der Blocklib wird diese Warteschlange zu Beginn jedes Frames von dem Render-Thread abgearbeitet.

Ein Diagramm der wichtigsten beteiligten Klassen ist in Abbildung~\ref{fig:loaderDiagram} gezeigt. Im Klassendiagramm sind die drei Klassen \classVAOTransmitter{}, \classTextureTransmitter{} und \classDataTransmitter{} abgebildet. Die Klasse \classLoader{} erzeugt Tasks, die dann Methoden dieser Klassen aufrufen, je nachdem, welche Art von Daten an die \ac{gpu} übertragen oder von dort gelöscht werden soll. Die Klasse \classVAOTransmitter{} ermöglicht den Umgang mit sogenannten \acp{vao}. Das ist eine Datenstruktur, die alle relevanten Informationen zur Darstellung eines Modells enthält, beispielsweise eines Chunks oder des Spielercharakters. Mit der Klasse \classTextureTransmitter{} lassen sich Texturen übertragen, die beispielsweise das Aussehen von Blöcken definieren. Der \classDataTransmitter{} wird genutzt, um die Daten an die \ac{gpu} zu senden, die in der Blocklib für die Berechnung von Niederschlag, wie beispielsweise Regen oder Schnee, benötigt werden.

\begin{figure}[htbp]
	\centering
	\includesvg[width=.85\textwidth]{Loader.svg}
	\caption[Klassendiagramm der wichtigsten Klassen, die am Laden und Entfernen von Daten auf die und von der \glsentryshort{gpu} beteiligt sind.]{Klassendiagramm der wichtigsten Klassen, die am Laden und Entfernen von Daten auf die und von der \ac{gpu} beteiligt sind. Zum Laden wird der \classLoader{} aufgerufen, dieser erstellt einen Task, der eine der drei Transmitter-Klassen aufruft. Die so erstellten Tasks werden an die \classDoubleBufferedAsyncQueue{} übergeben, die ein \classCompletableFuture{}-Objekt zurückgibt.}\label{fig:loaderDiagram}
\end{figure}

Die Klassen \classRawModel{} und \classModelTexture{}, welche die Informationen der \acp{vao} und Texturen speichern, implementieren das neu hinzugefügte Interface \classAsyncGraphicsObject{}, das die Methoden 	\code{setLoaded(...)}, 
\code{setUnloaded()} und 
\code{isLoaded()} bereitstellt. Damit kann überprüft werden, ob die Daten bereits auf die \ac{gpu} geladen sind. Die Renderer-Klassen, auf die in Abschnitt~\ref{sec:saveRenderState} genauer eingegangen wird, prüfen damit, ob sie ausschließlich auf der \ac{gpu} befindliche Objekte rendern sollen, und geben andernfalls einen Fehler aus.

Die Abarbeitung aller Ladevorgänge der Klasse \classLoader{} muss synchronisiert sein, da die Simulation sonst beispielsweise versuchen könnte, ein Objekt rendern zu lassen, das später nebenläufig von der \ac{gpu} entfernt werden würde. Nachdem der Ladevorgang von Objekten auf die \ac{gpu} abgeschlossen ist, kann auf allen Threads unabhängig nebenläufig überprüft werden, dass die Objekte gerendert werden können. Dieser Vorgang wird nun mit einem Beispiel verdeutlicht.

\begin{figure}
	\centering
	\includegraphics[height=4.5cm]{player.png}
	\caption[Bildschirmfoto der Spielerfigur in der Blocklib.]{Bildschirmfoto der Spielerfigur, die in diesem Kapitel als Beispiel für die Implementierung des Render-Threads verwendet wird.}\label{fig:player}
\end{figure}

\begin{example}
	Die Klasse \classPlayer{} ist für die Simulation des Spielercharakters zuständig. Zur Veranschaulichung ist ein Bildschirmfoto der Spielerfigur in Abbildung~\ref{fig:player} dargestellt. Wird ein neuer Spielercharakter erstellt, so ruft die Klasse \classPlayer{} die Methode \code{init()} auf, wodurch über verschiedene Methodenaufrufe zum Schluss auch \code{Loader.loadToVAO(...)} aufgerufen wird. Der \classLoader{} erzeugt einen Task mit Daten des \classPlayer{}-Modells und fügt diesen Task in die Warteschlange ein. Im darauffolgenden Frame wird die Warteschlange abgearbeitet und das Modell auf die \ac{gpu} geladen. Damit steht das zu ladende Spielercharakter-Modell ab diesem Frame zur Verfügung und kann genutzt werden, um den Charakter zu zeichnen. 

	Wie der Vorgang des Zeichnens aufgebaut ist, wird in den folgenden Abschnitten erklärt. Dort wird dieses Beispiel fortgeführt.
\end{example}

\subsubsection{Zustandsloses Rendering}\label{sec:statelessRendering}
In der Blocklib implementieren alle zu zeichnenden Elemente, auch Renderables genannt, das Interface \classRenderable{}. Dieses definiert die Funktionalität, die zum Zeichnen eines Elements notwendig ist, sowie die Methoden \code{show()} und \code{hide()}. Das Rendersystem ist so aufgebaut, dass ein Element nach einem Aufruf von \code{show()} so lange gezeichnet wird, bis \code{hide()} aufgerufen wird. Dies wird erreicht, indem eine Datenstruktur im \classMasterRenderer{} alle Renderables speichert. Diese zustandsbehaftete Zeichenmethode birgt die folgenden Vor- und Nachteile im Vergleich zu einem Rendering, das diese Informationen nicht speichert (hier \emph{zustandsloses Rendering} genannt):

\begin{itemize}
	\item[$+$] Da \code{show()} nur einmal aufgerufen werden muss, können auch Systeme ohne Update-Methode das Zeichnen von Elementen veranlassen.
	\item[$+$] Da die Elemente gespeichert sind, müssen sie nicht jedes Mal neu hinzugefügt werden. Das verringert den Rechenbedarf.
	\item[$-$] Werden Renderables häufig ausgetauscht, müssen die alten Elemente jedes Mal entfernt werden.
	\item[$-$] Da die Einführung eines parallelen Render-Threads ansonsten zu Wettkampfbedingungen führen würde, muss diese Datenstruktur aus Sicht des Render-Threads während des Zeichnens unverändert sein. Wie beschrieben wird dazu ein Double-Buffer eingesetzt. Bei einem zustandsbehafteten Double-Buffer müssen beim Wechseln die Elemente des einen Puffers in den anderen Puffer kopiert werden. Das erfordert Zeit, die nicht parallelisiert werden kann, da das Wechseln des Double-Buffers synchronisiert sein muss.
	\item[$-$] Die Existenz von paarweise aufzurufenden Funktionen birgt die Gefahr, dass der zweite Aufruf vergessen wird. Wie bei \code{malloc()} und \code{free()} in der Programmiersprache C entsteht durch einen fehlenden Aufruf von \code{hide()} ein Speicherleck. Zudem wird dann die Anzahl der zu zeichnenden Elemente immer größer und der Rendervorgang wird verlangsamt. Des Weiteren werden möglicherweise Renderables gezeichnet, die nicht gezeichnet werden sollen.
\end{itemize}

Zur Verringerung des Zeichenaufwands implementiert die Klasse \classChunkManager{} das sogenannte \emph{Frustum Culling}. Es wird berechnet, welche Chunks sich im Sichtfeld der Kamera befinden. Nur diese Chunks sollen gezeichnet werden. Dazu entfernt der \classChunkManager{} in jedem Frame alle Chunks mittels \code{hide()} aus der Datenstruktur der zu zeichnenden Elemente und fügt nur die als sichtbar ermittelten Chunks wieder ein. Die Chunks machen mit etwa 
% 334*2/(334*2+4+73+142)
75 \% bis
% 529*2/(529*2+4+73+142)
82 \% einen Großteil aller Renderables aus. Es wird also bereits ein Großteil der zu zeichnenden Elemente in jedem Frame neu hinzugefügt. Des Weiteren existiert aktuell in der Blocklib keine Klasse, die Renderables zu der Datenstruktur hinzufügt, aber keine Update-Methode besitzt. Deswegen werden die Gefahr von vergessenen \code{hide()} Aufrufen und der Kopieraufwand des Double-Buffers als gewichtiger eingeschätzt als die zuvor beschriebenen Vorteile.

Die Datenstruktur ist nun wie folgt implementiert: Die Methode \code{hide()} entfällt ersatzlos. Stattdessen wird der Puffer der Renderables des letzten Bilds nach jedem  Double-Buffer-Tausch geleert. So zeichnet der Renderer Renderables automatisch nicht mehr, sobald sie nicht mehr hinzugefügt werden. Die Methode \code{show()} wird zu \code{draw()} umbenannt, um zu signalisieren, dass es sich um einen einmaligen Vorgang handelt. Alle bisherigen Aufrufe von \code{show()} werden durch \code{draw()} ersetzt. Die Update-Methoden aller Klassen, die \classRenderable{}-Objekte anzeigen lassen, nutzen die neue \code{draw()}-Methode, sodass die Objekte in jedem Frame zu der Liste der zu zeichnenden Objekte hinzugefügt werden und das Rendering-Ergebnis zum vorherigen Stand identisch ist.

\begin{example}[Fortsetzung Beispiel:]
	Für den zu zeichnenden Spielercharakter bedeutet dies, dass nun nicht mehr in der \code{init()}-Methode einmal \code{show()} aufgerufen werden kann, damit der Charakter gezeichnet wird, bis er beispielsweise stirbt und mit \code{hide()} entfernt wird. Stattdessen übernimmt die Aufgabe des Zeichnens nun der bereits vorhandene \classEntityManager{}, der Objekte wie den Spielercharakter verwaltet. In der \code{update(...)}-Methode von \classEntityManager{} wird nun während der Iteration über die verwalteten Elemente zum Schluss für jedes Element die \code{draw()}-Methode aufgerufen. Damit wird der Spielercharakter automatisch gerendert, sobald er der Liste der Elemente von \classEntityManager{} hinzugefügt wird. Wird der Spielercharakter aus der Liste entfernt, verschwindet er automatisch.
\end{example}

\subsubsection{Zwischenspeicherung des gerenderten Spielzustands}\label{sec:saveRenderState}
\begin{figure}
	\centering
	\includesvg[width=.8\textwidth]{Rendering.svg}
	\caption[Klassendiagramm der wichtigsten Klassen, die Daten für das Rendering zwischenspeichern.]{Klassendiagramm der wichtigsten Klassen, die Daten für das Rendering zwischenspeichern. Die Klasse \classMasterRenderer{} delegiert das Rendering an die anderen Renderer.}\label{fig:renderDiagram}
\end{figure}
Aufbauend auf den Änderungen des vorherigen Abschnitts ist nun die Zwischenspeicherung der \classRenderable{}-Objekte vereinfacht, da diese nur für ein Bild gespeichert werden müssen. Die Objekte der Spielwelt werden anders gerendert als die \ac{gui}, da die ihr zugrundeliegende Baumstruktur die Anordnung der Renderables bestimmt. Die \ac{gui} besitzt eine Baumstruktur, weil sich so ausgehend von einem Wurzelelement eine Nutzeroberfläche aufbauen lässt, in der Elemente aus weiteren Elementen aufgebaut sind (beispielsweise ein Menü mit mehreren Einträgen). Jedes Element wird nunmehr relativ zu dem ihm übergeordneten Element platziert. Untergeordnete Elemente liegen im Vergleich zu den ihnen übergeordneten Elementen im Vordergrund. Die Blocklib besitzt eine spezielle Klasse \classUIRenderer{}, die das Rendering der \ac{gui} durchführt. Deswegen unterscheiden sich die Vorgänge zur Zwischenspeicherung des gerenderten Spielzustands der Spielwelt und der \ac{gui}. Abbildung~\ref{fig:renderDiagram} zeigt die wichtigsten Klassen für die Ausführung des Rendering und die Zwischenspeicherung des gerenderten Spielweltzustands. Die Klasse \classMasterRenderer{} übernimmt zwar einen Teil der Zwischenspeicherung, gibt allerdings auch Aufgaben an die anderen in Abbildung~\ref{fig:renderDiagram} gezeigten Klassen ab, wie in den folgenden Abschnitten beschrieben.

Alle in der Abbildung gezeigten Klassen implementieren das Interface \classDoubleBuffer{}, wie in Abbildung~\ref{fig:renderInterfaceDiagram} dargestellt. Zum Großteil wird die Zwischenspeicherung über die von \classDoubleBuffer{} bereitgestellte Methode \code{swapBuffer()} implementiert.

Im folgenden Abschnitt wird die Implementierung der Zwischenspeicherung für die Spielwelt erörtert, danach die der \ac{gui}. Im Anschluss daran wird die Zwischenspeicherung weiterer Elemente des Spielzustands beschrieben, die keine \classRenderable{}-Objekte sind.

\begin{figure}
	\centering
	\includesvg[width=.9\textwidth]{Rendering-Interface.svg}
	\caption{Klassen, die \classDoubleBuffer{} und \classRenderer{} implementieren.}\label{fig:renderInterfaceDiagram}
\end{figure}

\paragraph{Zwischenspeicherung in der Spielwelt} Die Zwischenspeicherung der Renderables erfolgt auf zwei Arten. Zum einen muss gespeichert werden, welche Elemente im aktuellen Frame gezeichnet werden, zum andern muss sichergestellt werden, dass sich der Inhalt der zu zeichnenden \classRenderable{}-Objekte nicht verändert. Wird diese Zwischenspeicherung nicht vorgenommen, können Wettkampfbedingungen auftreten, was zu unerwünschten visuellen Ergebnissen führt. Ein Beispiel für eine solche Wettkampfbedingung in der Blocklib ist das Auftreten von flackernden Schatten, das in Abschnitt~\ref{sec:searchRace} näher beschrieben und in Abbildung~\ref{fig:flackern} dargestellt wird.

Beim Hinzufügen eines \classRenderable{} in die Menge der zu zeichnenden Objekte mittels der \code{draw()}-Methode stellt der \classMasterRenderer{} zuerst sicher, dass sich der zu zeichnende Inhalt nicht mehr verändert. Dieser Vorgang ist schematisch in Abbildung~\ref{fig:copyRenderable} dargestellt. Der Inhalt des \classRenderable{}-Objekts wird in ein Objekt der Klasse \classRenderedRenderable{} kopiert, welches als Behälter für diese Informationen dient. Da in Java das Erzeugen von Objekten einen gewissen Aufwand verursacht, werden diese \classRenderedRenderable{}-Behälter nach Benutzung wiederverwendet. Dazu besitzt die Klasse \classMasterRenderer{} drei Listen: eine Liste von Elementen, die gerade nicht genutzt werden (folgend auch Cache genannt), eine Liste von Elementen, die in diesem Frame befüllt wurden (\var{usedRenderables}) und eine Liste von Elementen, die im letzten Frame befüllt worden sind und in diesem Frame gerendert wurden (\var{lastUsedRenderables}). Ein Behälter wandert also von Cache zu \var{usedRenderables} zu \var{lastUsedRenderables} und dann wieder in den Cache zurück. In jedem Frame wandert der Behälter eine Liste weiter.

\begin{figure}
	\centering
	\begin{tikzpicture}
		\node[draw, minimum size = 1cm, label ={\small\texttt{RenderedRenderable}}] at (1,2) (rr) {};
		\node[draw, minimum size = 1cm, label ={\small\texttt{Renderable}}] at (-3,2) (r) {};
		\node[draw, minimum size = 1cm, label ={Cache}] at (-3,0) (c) {};
		\node[draw, minimum size = 1cm, label ={\small\texttt{usedRenderables}}]  at (3,0) (u) {};
		\draw[->, very thick] (c) -- coordinate[pos=.67] (start)  (u);
		\draw[dashed] (rr) -- (0,0);
		\foreach \i in {-.1,.1,.3}{
		\draw[->] ($(r.center)+(0,\i)$) -- ($(rr.center)+(0,\i)$);
		}
		\draw[->] ($(r.center)+(0,-.3)$) -- node[below] {Daten} ($(rr.center)+(0,-.3)$);
		\node[draw,rounded rectangle,anchor=west]  at (2.5,-1.5) (omr) {\texttt{(Opaque/Transparent)MasterRenderer}};
		\draw[->, very thick] (start) |- (omr.west);
		\foreach \i in {-.3,-.1,.1,.3}{
			\draw ($(c.center)+(-.5,\i)$) -- ($(c.center)+(.5,\i)$);
			\draw ($(u.center)+(-.5,\i)$) -- ($(u.center)+(.5,\i)$);
			}
		\end{tikzpicture}
		\caption[Vorgang der Zwischenspeicherung der Daten während eines \code{draw()}-Aufrufs in \classMasterRenderer{}.]{Vorgang der Zwischenspeicherung der Daten während eines \code{draw()}-Aufrufs in \classMasterRenderer{}. Ein \classRenderedRenderable{}-Behälter (oben mittig) wird aus der Liste der aktuell nicht benötigten Behälter (Cache) entnommen, in \var{usedRenderables} gespeichert und an den korrekten Renderer übergeben. Die Daten des zu zeichnenden Objekts (Renderable, links oben) werden in den \classRenderedRenderable{}-Behälter kopiert. Durch die gestrichelte Linie wird dargestellt, dass das \classRenderedRenderable{}-Objekt, in das die Daten kopiert werden, dasselbe ist, das von dem Cache an \var{usedRenderables} und an den Renderer übergeben wird.}\label{fig:copyRenderable}
\end{figure}


\begin{example}[Fortsetzung Beispiel:]
	Betrachtet man nun das vorherige Beispiel der Spielerfigur (also der Klasse \classPlayer{}), läuft dieser Vorgang wie folgt ab: Die Klasse \classPlayer{} implementiert indirekt das Interface \classRenderable{}, da sie von der Klasse \classEntity{} abgeleitet ist. Nun wird während der Simulation durch den \classEntityManager{} die \code{draw()}-Methode von \classPlayer{} aufgerufen. Das \classPlayer{}-Objekt wird an den \classMasterRenderer{} übergeben. Dieser entnimmt einen \classRenderedRenderable{}-Behälter aus der Liste der gerade nicht verwendeten Behälter (\emph{Cache}) und befüllt diesen mit den Daten aus dem \classPlayer{}-Objekt. Jetzt können Änderungen im \classPlayer{} vorgenommen werden, ohne dass diese die Daten beeinflussen, die zum Zeichnen verwendet werden. Es wird also genau der Zustand dargestellt, in dem sich der \classPlayer{} befindet, während die \code{draw()}-Methode aufgerufen wird. Der mit den Daten aus \classPlayer{} befüllte \classRenderedRenderable{}-Behälter wird nun an den \classOpaqueMasterRenderer{} weitergegeben und in der Liste \var{usedRenderables} abgelegt.
\end{example}

Die Klassen \classOpaqueMasterRenderer{} und \classTransparentMasterRenderer{} sind für das Rendering von undurchsichtigen respektive durchsichtigen Objekten in der Spielwelt zuständig. Beide implementieren das Interface \classDoubleBuffer{}, um zwischenzuspeichern, welche Elemente in einem Frame gezeichnet werden sollen. Wird die \code{draw()}-Methode einer der beiden Klassen aufgerufen, wird das übergebene \classRenderable{}-Objekt in der Menge \var{nextEntries} gespeichert, welche die Elemente enthält, die im folgenden Frame gezeichnet werden sollen. 

\begin{example}[Fortsetzung Beispiel:]
	Nachdem der \classMasterRenderer{} nun also die Daten des \classPlayer{}-Objekts in den \classRenderedRenderable{}-Behälter kopiert hat, ruft er die \code{draw()}-Methode von \classOpaqueMasterRenderer{} auf. Dieser nimmt den Behälter entgegen und speichert ihn in \var{nextEntries}.
\end{example}

Nachdem die Simulation für einen Frame abgeschlossen ist, enthalten die \var{nextEntries} Mengen von \classOpaqueMasterRenderer{} und \classTransparentMasterRenderer{} alle Objekte der Welt, die im nächsten Frame gezeichnet werden sollen. 

\begin{figure}
	\centering
	\begin{tikzpicture}[scale=.9]
		\usetikzlibrary{calc,positioning}
		\newlength\testl
		\pgfmathsetlength\testl{.9cm}
		\node[draw, minimum size = \testl, label={Cache} ] at (0,0) (c) {};
	
		\node[draw,rounded rectangle, anchor=west, font={\small\ttfamily}] at (1.5,1) (ur) {usedRenderables};
		\node[draw,rounded rectangle, anchor=west, font={\small\ttfamily}] at (1.5,-1) (lur) {lastUsedRenderables};
	
		\node[draw, minimum size = \testl, right = 4.5 of ur.west,] (u) {};
		\node[draw, minimum size = \testl, right = 4.5 of lur.west,] (lu) {};
	
		\foreach \i in {-.3,-.1,.1,.3}{
			\draw ($(c.west)+(0,\i)$) -- ($(c.east)+(0,\i)$);
			\draw ($(u.west)+(0,\i)$) -- ($(u.east)+(0,\i)$);
			\draw ($(lu.west)+(0,\i)$) -- ($(lu.east)+(0,\i)$);
		}
	
		\draw (ur.east) -- (u.west);
		\draw (lur.east) -- (lu.west);
		\draw[->] (lu) to [bend right=10] node[above, sloped] {verschieben}  (c);
		\draw (-1, -2) rectangle coordinate (label) (8,2);
			\node[anchor= south] at (label |- 0,2) {Schritt 1};
	
		\begin{scope}[shift={(9.5,0)}]
			\node[draw, minimum width = \testl, minimum height= 2\testl, label={Cache} ] at (0,0) (c) {};
		
			\node[draw,rounded rectangle, anchor=west, font={\small\ttfamily}] at (1.5,1) (ur) {usedRenderables};
			\node[draw,rounded rectangle, anchor=west, font={\small\ttfamily}] at (1.5,-1) (lur) {lastUsedRenderables};
		
			\node[draw, minimum size = \testl, right = 4.5 of ur.west,] (u) {};
			\node[draw, minimum size = \testl, right = 4.5 of lur.west,] (lu) {};
	
			\foreach \i in {-.8,-.6,...,.8}{
				\draw ($(c.west)+(0,\i)$) -- ($(c.east)+(0,\i)$);
			}
		
			\foreach \i in {-.3,-.1,.1,.3}{
				\draw ($(u.west)+(0,\i)$) -- ($(u.east)+(0,\i)$);
			}
		
			\draw (ur.east) -- (lu.west);
			\draw (lur.east) -- (u.west);
			\draw (-1, -2) rectangle coordinate (label) (8,2);
			\node[anchor= south] at (label |- 0,2) {Schritt 2};
		\end{scope}
	
		\end{tikzpicture}
		\caption[Wechsel der Puffer in \classMasterRenderer{} während \code{swapBuffer()}.]{Wechsel der Puffer in \classMasterRenderer{} während \code{swapBuffer()}. In Schritt 1 wird der Inhalt von \var{lastUsedRenderables} in den Cache verschoben. In Schritt 2 werden die Listen von \var{usedRenderables} und \var{lastUsedRenderables} vertauscht. Dabei ist zu beachten, dass nur die Referenzen auf die Listen getauscht und anders als in Schritt 1 nicht alle Listenelemente einzeln verschoben werden.}\label{fig:mrSwapBuffer}
\end{figure}

Im nächsten Frame müssen diese Elemente gezeichnet werden. Zu Beginn des Frames werden alle Puffer der \classRenderer{}-Klassen mittels der \code{swapBuffer}-Methode getauscht. Die Klasse \classMasterRenderer{} fügt, wie in Abbildung~\ref{fig:mrSwapBuffer} dargestellt, alle im letzten Frame für das Rendering verwendeten Behälter (gespeichert in \var{lastUsedRenderables}) der Liste der nicht mehr verwendeten Behälter hinzu und leert dann \var{lastUsedRenderables}. Danach vertauscht er die Listen der benutzten und zuletzt benutzten Behälter, sodass \var{usedRenderables} nun eine leere Liste ist, der während der Simulation wieder benutzte Behälter hinzugefügt werden können. \classOpaqueMasterRenderer{} und \classTransparentMasterRenderer{} tauschen jeweils ihre Puffer \var{entries} und \var{nextEntries} und leeren danach \var{nextEntries}, sodass während der Simulation die im nächsten Frame zu zeichnenden Objekte eingefügt werden können.

\begin{example}[Fortsetzung Beispiel:]
	Nach dieser Phase findet sich nun also der Behälter, der die unveränderten Daten des \classPlayer{}-Objekts enthält, durch den Puffertausch in der Liste \var{lastUsedRenderables} in \classMasterRenderer{} und in \classOpaqueMasterRenderer{} in der \var{entries}-Menge. Während des Renderings zeichnet der \classOpaqueMasterRenderer{} Elemente der \var{entries}-Menge.
	
	Im folgenden Frame werden während \code{swapBuffer()} wieder alle Puffer getauscht, wodurch der \classRenderedRenderable{}-Behälter der \classPlayer{}-Daten aus der Menge von \classOpaqueMasterRenderer{} entfernt wird und in die Menge der nicht genutzten Behälter von \classMasterRenderer{} eingefügt wird, um später wieder als Behälter für die Daten eines anderen \classRenderable{} genutzt werden zu können. Damit ist ein Zyklus der Zwischenspeicherung durchlaufen.
\end{example}

\paragraph{Zwischenspeicherung in der \glsentryshort{gui}}
Die \ac{gui}-Elemente werden von der Klasse \classUIRenderer{} verwaltet und werden nicht jeden Frame zu einer Menge von zu zeichnenden Elementen hinzugefügt. 
Stattdessen wird die Baumstruktur der \ac{gui} durchlaufen und jedes Element gezeichnet, das das Attribut \var{visible} als \code{true} gesetzt hat. Daher kann der im letzten Abschnitt beschriebene Vorgang nicht für das Rendering der \ac{gui} verwendet werden. Dennoch muss auch für die \ac{gui} der Spielzustand zwischengespeichert werden, da sonst ebenfalls Wettkampfbedingungen auftreten. Zum Beispiel kann es vorkommen, dass sich die Breite von gezeichnetem Text scheinbar zufällig von Frame zu Frame verändert, da die gespeicherten Dimensionen und die Textur des Textes durch die Wettkampfbedingung nicht zueinander passen. Für die Zwischenspeicherung des \ac{gui}-Zustands wird die Tatsache ausgenutzt, dass im Gegensatz zu den Objekten der Spielwelt alle \ac{gui}-Elemente von einer Klasse, \classUIItem{}, abgeleitet sind. Diese Klasse definiert das \classRenderable{}-Objekt, das zur Darstellung eines \ac{gui}-Elements genutzt wird. 

Zur Implementierung der Zwischenspeicherung wird das zuvor genutzte \classUIRenderable{}-Objekt durch die in Abbildung~\ref{fig:renderDiagram} bereits dargestellte Klasse \classDoubleBufferedUIRenderable{} ersetzt, die für jedes \ac{gui}-Element die Zwischenspeicherung übernimmt. Die Klasse enthält zwei Objekte des Typs \classUIRenderable{}, die während des \code{swapBuffer()}-Aufrufs synchronisiert getauscht werden. Zudem wird zeitgleich der Inhalt des geänderten \classUIRenderable{} in das ältere und konstante Objekt kopiert. So zeichnet \classUIRenderer{} Elemente, die während des Renderings unverändert bleiben.

\paragraph{Zwischenspeicherung des restlichen gerenderten Spielzustands} Neben den Informationen, die in den  \classRenderable{}-Objekten gespeichert sind, umfasst der für das Rendering relevante Spielzustand noch weitere Daten. 
Die Blocklib enthält sogenannte \emph{globale} \glspl{uniform}. Ein \gls{uniform}~\cite[S.~45~\psqq]{Vries2020} ist eine globale Variable, die in einem \gls{shaderprogram} definiert wird. Ein \gls{shaderprogram}~\cite[S.~32~\psq]{Vries2020} ist ein \gls{Programm}, das auf der \ac{gpu} ausgeführt wird. Die in der Blocklib als globale \glspl{uniform} bezeichneten Variablen sind \glspl{uniform}, die alle \glspl{shaderprogram} nutzen können, also auch über die \glspl{shaderprogram} hinweg global sind. Da in den globalen \glspl{uniform} Informationen wie die Position und Blickrichtung der Kamera gespeichert sind, müssen auch diese Daten während des Renderings unverändert bleiben. Die Simulation ändert aber beispielsweise die Position der Kamera, wenn eine Bewegungstaste gedrückt wird. Daher müssen die globalen \glspl{uniform} ebenfalls zwischengespeichert werden.

Die globalen \glspl{uniform} sind in der Klasse \classGlobalUniforms{} als statische Attribute gespeichert. Aus diesem Grund lässt sich dafür nur mit großem Aufwand ein normaler Double-Buffer verwenden. Alle Objekte, die auf die Attribute zugreifen, müssten so angepasst werden, dass sie nicht auf statische Werte zugreifen, sondern auf eine Instanz der Klasse und alle statischen Attribute müssten zu Objekt-Attributen geändert werden.

Daher wird ein Double-Buffer für die \emph{Änderung} der globalen \glspl{uniform} genutzt. Die Klasse \classMasterRenderer{} besitzt dafür die beiden \classRunnable{}-Objekte \var{setUniforms} und \var{nextSetUniforms}. Die öffentliche Methode \code{setGlobalUniforms()} von \classMasterRenderer{} führt die Änderungen analog zu der Klasse \classLoader{} nun nicht sofort durch, sondern speichert die \glspl{Anweisung} in \var{nextSetUniforms}. In der \code{swapBuffer}-Methode werden \var{setUniforms} und \var{nextSetUniforms} synchronisiert vertauscht. Vor dem Rendering werden die gespeicherten \glspl{Anweisung} ausgeführt. Somit ist der Zustand der globalen \glspl{uniform} aus Sicht des Render-Threads konstant.

\subsubsection{Anpassung der Game-Loop}\label{sec:adjustGameLoop}
Die bisher in der Klasse \classGame{} verortete Game-Loop wird in eine Klasse \classGameLoop{} ausgelagert, um die für den Vorgang nötigen Methoden und Attribute von den restlichen Elementen der \classGame{}-Klasse zu trennen. Da diese beiden Klassen allerdings weiterhin eng miteinander gekoppelt sind, ist \classGameLoop{} als innere Klasse in \classGame{} implementiert. Ein Klassendiagramm von  \classGameLoop{} mit allen Attributen und Methoden ist in Abbildung~\ref{fig:DiagGameLoop} zu sehen.

\begin{figure}[htbp]
	\centering
	\includesvg[width=\textwidth]{GameLoop.svg}
	\caption[Klassendiagramm von \classGameLoop{} und \classRenderThread{}.]{Klassendiagramm der für die Game-Loop zuständigen Klasse \classGameLoop{} und der neuen Rendering-Klasse \classRenderThread{}.}\label{fig:DiagGameLoop}
\end{figure}

Die Klasse definiert eine öffentliche Methode \code{start()}, die die Game-Loop startet. Da die Blocklib als Multiplayer- und Singleplayer-Bibliothek ausgelegt ist, kann die Software auch als Server ohne Bildausgabe genutzt werden. Das verändert den Ablauf der Game-Loop, denn dann muss kein Bild berechnet werden und es existiert kein Render-Thread. Damit diese Unterscheidung während des Starts der Game-Loop durchgeführt werden kann, nutzt die neue Klasse \classGameLoop{} zur Speicherung der durchzuführenden Aktionen \classRunnable{}- und \classSupplier{}-Objekte. Dadurch können in den unterschiedlichen Ausführungsmodi verschiedene Aktionen ausgeführt werden. Das Attribut \var{stepAction} enthält die Aktionen, die während eines Frames beziehungsweise während eines Simulationsschritts ausgeführt werden sollen. In der \code{init()}-Methode wird das Attribut auf \code{GameLoop.stepWindow} gesetzt, wenn eine Bildausgabe erfolgen soll, ansonsten wird die Variable auf \code{GameLoop.stepSimulation} gesetzt. Auch die anderen Attribute werden während der Initialisierung auf die jeweils gewünschten Werte gesetzt. Während der Ausführung der Game-Loop muss dann nicht geprüft werden, welcher Ausführungsmodus beim Start der Blocklib festgelegt worden ist.

In der Methode \code{stepWindow()} werden die Simulation und das Rendering nebenläufig ausgeführt. Dazu wird das Jobsystem genutzt, um die Methode \code{stepSimulation()} nebenläufig zu starten. 

Die \glspl{Anweisung}, die der Render-Thread ausführt, sind in die Klasse \classRenderThread{} ausgelagert. Das Klassendiagramm in Abbildung~\ref{fig:DiagGameLoop} zeigt rechts die Klasse \classRenderThread{} mit ihren Attributen und Methoden. Der Ablauf der Methode \code{stepWindow()} ist in Abbildung~\ref{fig:stepWindowActivity} dargestellt.
\begin{figure}
	\centering
	\includesvg[scale=.8]{GameLoopActivity.svg}
	\caption[Aktivitätsdiagramm der Methode \code{stepWindow()} in \classGameLoop{}.]{Aktivitätsdiagramm der Methode \code{stepWindow()} in \classGameLoop{}. Die Simulation und das Rendering werden nebenläufig ausgeführt, alle anderen \glspl{Aktivitaet} werden automatisch synchronisiert ausgeführt, indem sie außerhalb des nebenläufigen Bereichs sind.}\label{fig:stepWindowActivity}
\end{figure}
Die \classGameLoop{}-Klasse ruft in der \gls{Aktivitaet} \enquote{Tausche Puffer} die Methode \code{swapBuffer()}, in \enquote{Lade Daten auf \ac{gpu}} die \code{loadGraphics()}-Methode und in \enquote{Rendering} \code{render()} auf. Wie in der Abbildung durch die schwarzen Nebenläufigkeits-Balken\footnote{Diese Balken haben keine Verbindung zu Transitionen von Petri-Netzen, obwohl sie ähnlich aussehen. Petri-Netze und Aktivitätsdiagramme sind unterschiedliche Modelle. Bei einem Aktivitätsdiagramm besitzen die Balken die Bedeutung, dass alle Aktivitäten (die abgerundeten Rechtecke), die zwischen den Balken existieren und keine Kanten zueinander besitzen, nebenläufig sind.} zu erkennen ist, werden die Simulation und das Rendering in \code{stepWindow()} nebenläufig ausgeführt. Die restlichen \glspl{Anweisung} werden sequentiell ausgeführt, wie in den vorherigen Abschnitten beschrieben.