Abschließend werden nun noch einige Richtungen skizziert, in die die Blocklib weiterentwickelt werden kann. Kurz erwähnt sei, dass das Thema \enquote{Testen und Verhinderung von Regressionen} eine Aufgabe von großer Wichtigkeit ist,  deren detaillierte Beschreibung allerdings über den Rahmen dieser Arbeit hinausgeht. Durch die fehlende Testabdeckung und inzwischen veraltete Tests, ist es schwierig abzuschätzen, ob Änderungen im Code zu Problemen an anderer Stelle führen. Folglich ist es wichtig, dieses Thema in Bezug auf die Anforderungen in der Blocklib zu erörtern. Relevante Fragen sind hier, welche Art von Testung betrieben werden soll, wie verhindert werden kann, dass Tests veralten, und wie Tests genutzt werden können, um die Qualität der Blocklib zu sichern.

In den folgenden Abschnitten wird nun zuerst die Architektur der Blocklib betrachtet. Eine Kompositions-Wurzel ist ein typisches Software-Architektur-Element. In der Blocklib wird sie allerdings nicht verwendet. Daher werden im folgenden Abschnitt die Vor- und Nachteile erörtert, die sich aus der Nutzung dieses Modells ergeben. Im Anschluss wird ein Thema erörtert, welches das Rendering von Schrift betrifft, da die Art und Weise des Renderings in der Blocklib spezielle Nachteile hat. Zuletzt werden mögliche Verbesserungen für die in dieser Arbeit entwickelten nebenläufigen Architektur beschrieben.