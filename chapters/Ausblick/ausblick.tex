Abschließend werden nun noch einige Richtungen skizziert, in die die Blocklib weiterentwickelt werden kann. Zuerst wird die Architektur der Blocklib betrachtet. Ein typisches Architektur-Element ist für gewöhnlich eine Composition-Root. In der Blocklib wird sie allerdings nicht verwendet. Daher werden im folgenden Abschnitt die Vor- und Nachteile erörtert, die sich aus der Nutzung dieses Modells ergeben. Danach wird ein Thema, das das Rendering von Schrift betrifft erörtert, da die Art und Weise, wie das Rendering in der Blocklib gelöst ist, spezielle Nachteile hat. Als letztes werden mögliche Verbesserungen für die in dieser Arbeit entwickelten nebenläufigen Architektur beschrieben.

Kurz sei noch erwähnt, dass das Thema Testen und Verhinderung von Regressionen eine Aufgabe ist, die hier zwar nicht mehr detailliert beschrieben wird, aber dennoch sehr wichtig ist. Durch die fehlende Testabdeckung und inzwischen veraltete Tests, ist es beinahe unmöglich abzuschätzen, ob Änderungen im Code zu Problemen an anderer Stelle führen. Daher ist dieses Thema wohl eins der wichtigsten, die für die Blocklib erörtert werden sollten. Die größten Fragen sind hier, welche Art von Testing betrieben werden soll, wie sichergestellt werden kann, dass Tests nicht veralten und wie Tests genutzt werden können, um die Qualität der Blocklib zu sichern.