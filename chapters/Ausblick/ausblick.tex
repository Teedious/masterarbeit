Abschließend werden nun noch einige Richtungen skizziert, in die die Blocklib weiterentwickelt werden kann. Kurz sei noch erwähnt, dass das Thema \enquote{Testen und Verhinderung von Regressionen} eine Aufgabe ist, die hier nicht mehr detailliert beschrieben wird, aber dennoch sehr wichtig ist. Durch die fehlende Testabdeckung und inzwischen veraltete Tests, ist es schwierig abzuschätzen, ob Änderungen im Code zu Problemen an anderer Stelle führen. Daher ist es wichtig dieses Thema in Bezug auf die Anforderungen in der Blocklib zu erörtern. Dabei sind die relevanten Fragen welche Art von Testing betrieben werden soll, wie sichergestellt werden kann, dass Tests nicht veralten und wie Tests genutzt werden können, um die Qualität der Blocklib zu sichern.

In den folgenden Abschnitten wird nun zuerst die Architektur der Blocklib betrachtet. Eine Kompositions-Wurzel ist ein typisches Software-Architektur-Element. In der Blocklib wird sie allerdings nicht verwendet. Daher werden im folgenden Abschnitt die Vor- und Nachteile erörtert, die sich aus der Nutzung dieses Modells ergeben. Danach wird ein Thema, das das Rendering von Schrift betrifft erörtert, da die Art und Weise, wie das Rendering in der Blocklib gelöst ist, spezielle Nachteile hat. Zuletzt werden mögliche Verbesserungen für die in dieser Arbeit entwickelten nebenläufigen Architektur beschrieben.