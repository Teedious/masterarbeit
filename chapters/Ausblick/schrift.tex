Ein Punkt, der während der Implementierung der nebenläufigen Architektur im Bereich \ac{gui} aufgefallen ist, ist die Art und Weise, wie in der Blocklib Text gerendert wird. \textcite{Vries2020} beschreibt die Nutzung von sogenannten \emph{Bitmap-Fonts} als die typische Rendermethode für Text. Bei dieser Methode werden alle zu zeichnenden Zeichen in einer Textur gespeichert. Ein Beispiel für eine solche Textur ist in Abbildung~\ref{fig:bitmapfont} dargestellt.
\begin{figure}[!htbp]
	\centering
	\includegraphics[width=.8\textwidth]{ExportedFont.png}
	\caption[Bitmap-Font der Schrift Consolas.]{Bitmap-Font der Schrift Consolas, erzeugt mit \enquote{Codehead's Bitmap Font Generator}~\cite{Codehead2015}.}\label{fig:bitmapfont}
\end{figure}
Durch die Textur sind Schriftart und Schriftgröße festgelegt. Möchte man diese ändern, wird eine weitere Textur benötigt. Um Text zu rendern wird für jeden Buchstaben die entsprechende Stelle im Bitmap-Font ausgewählt und gerendert. In der Blocklib wird statt eines Bitmap-Fonts ein Rendersystem von Java, das Abstract Window Toolkit (AWT), genutzt. Damit werden zur Laufzeit mit der \ac{cpu} Texturen des gesamten zu rendernden Texts erzeugt und diese dann auf die \ac{gpu} geladen. Soll der Text angezeigt werden, kann nun die gesamte Textur gerendert werden.

Beide Methoden haben bestimmte Vor- und Nachteile. Mit Bitmap-Fonts ist die Wahl der Schriftart durch die gegebenen Texturen eingeschränkt, zudem ist die Geschwindigkeit stark von der Menge der gerenderten Symbole abhängig. Die Methode der Blocklib hat den Vorteil, dass beliebige Fonts und Schriftgrößen genutzt werden können und wenn eine Textur erstellt wurde, ist das Rendern des gesamten Texts sehr schnell. Große Probleme treten allerdings auf, wenn Text gerendert werden soll, der sich schnell ändert, wie beispielsweise ein Zeitanzeige mit Millisekunden oder eine Konsole, in die Text geschrieben werden kann. Jedes mal, wenn sich der zu rendernde Text ändert muss zuerst auf der \ac{cpu} eine neue entsprechende Textur gerendert werden, dann muss diese auf die \ac{gpu} geladen werden, damit sie schließlich angezeigt werden kann. In den eben genannte Fällen passiert das sehr häufig, was zu einer hohen Auslastung der \ac{cpu} und des \ac{gpu}-Busses (die Leitung, über die Daten an die \ac{gpu} geschickt werden können) führt.

Es ist aber möglich, die Vorteil beider Methoden zu kombinieren. Da Spiele inzwischen häufig verschiedene Schriften und Schriftgrößen nutzen, werden Bibliotheken wie FreeType~\cite{TheFreeTypeProject,Vries2020} genutzt, um zur Laufzeit Texturen von Schriften zu erzeugen, die dann als Bitmap-Font auf die \ac{gpu} geladen und dann genutzt werden können. Dieser Ansatz kann auch in der Blocklib gewählt werden, sodass AWT nicht mehr genutzt wird, um einen bestimmten Text zu rendern, sondern, um einen Bitmap-Font zu erzeugen. Zusätzlich müssen die damit verbundenen nötigen Schriftdaten, wie Breite der einzelnen Zeichen und weitere typografisch wichtige Metriken erzeugt und gespeichert werden. Alternativ wäre es möglich zwei getrennte Systeme anzubieten, ein System, das einen Bitmap-Font für sich schnell ändernde Texte nutzt, und eins, das wie bisher Texturen für lange sich selten ändernde Texte erzeugt, sodass diese gesammelt angezeigt werden können. Das zweite bestehende System kann auch dann weiter genutzt werden, wenn die Bitmap-Fonts dynamisch erzeugt werden.