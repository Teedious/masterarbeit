\appendix
\clearpage
\renewcommand{\sectionmark}[1]{\markboth{\Ifnumbered{section}{Anhang \thesection}{}}{#1}}
\pagenumbering{Roman}
\setcounter{page}{1}


\section{Executor im Concurrency Framework}\label{appendix:concFrameworkExecutor}
\includesvg[width=\textwidth]{ConcurrencyFrameworkExecutor.svg}

\section{\texttt{CompletionStage} im Concurrency Framework}\label{appendix:CompletionStage}
\includesvg[width=\textwidth]{CompletionStage.svg}
Klassendiagramm des Interface \class{CompletionStage}, auf das auf Seite~\pageref{sec:CompletableFuture} verwiesen wird. Die definierten Methoden bieten verschiedene Möglichkeiten der Komposition.

\begin{tblr}{
	colspec={ll},
	row{1}={font=\bfseries}
	}
	Komposition & Methoden \\
	Verkettung & \code{thenApply(...)}, \code{thenAccept(...)}, \code{thenRun(...)}\\
	Zusammenführung & \code{thenCombine(...)}, \code{thenAcceptBoth(...)}, \code{runAfterBoth(...)} \\
	Auswahl & \code{applyToEither(...)}, \code{acceptEither(...)}, \code{runAfterEither(...)} \\
\end{tblr}

\pagebreak
\section{\texttt{Context} vollständiges Klassendiagramm}\label{appendix:context}
{
	\centering
	\includesvg[height=.7\textheight]{Context.svg}
	\todo{Caption}
}

\pagebreak
\section{\texttt{BlocklibExecutorService} Klassendiagramm}\label{appendix:BlocklibExecutorService}
{
	\centering
	\includesvg[width=\textwidth]{ExecuterInterfaces.svg}
	\todo{Caption}
}
\pagebreak
\section{Messungen}
\pgfplotsset{
	height=3cm,
}
\fpsplot{seed-0-hexagon}
\fpsplot{seed-0-cube}
\fpsplot{seed-0-static}
\fpsplot{seed-0-rotate}
\fpsplot{seed-0-walk}
\fpsplot{seed-3-walk}
\fpsplot{seed-10-walk}