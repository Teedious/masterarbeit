\appendix
\clearpage
\renewcommand{\sectionmark}[1]{\markboth{\Ifnumbered{section}{Anhang \thesection}{}}{#1}}
\pagenumbering{Roman}
\setcounter{page}{1}


\section{Executor im Concurrency Framework}\label{appendix:concFrameworkExecutor}
\includesvg[width=\textwidth]{ConcurrencyFrameworkExecutor.svg}

\section{\texttt{CompletionStage} im Concurrency Framework}\label{appendix:CompletionStage}
\includesvg[width=\textwidth]{CompletionStage.svg}
Klassendiagramm des Interface \class{CompletionStage}, auf das auf Seite~\pageref{sec:CompletableFuture} verwiesen wird. Die definierten Methoden bieten verschiedene Möglichkeiten der Komposition.

\begin{tblr}{
	colspec={ll},
	row{1}={font=\bfseries}
	}
	Komposition & Methoden \\
	Verkettung & \code{thenApply(...)}, \code{thenAccept(...)}, \code{thenRun(...)}\\
	Zusammenführung & \code{thenCombine(...)}, \code{thenAcceptBoth(...)}, \code{runAfterBoth(...)} \\
	Auswahl & \code{applyToEither(...)}, \code{acceptEither(...)}, \code{runAfterEither(...)} \\
\end{tblr}

\pagebreak
\section{\texttt{Context} vollständiges Klassendiagramm}\label{appendix:context}
{
	\centering
	\includesvg[height=.7\textheight]{Context.svg}
	\todo{Caption}
}

\pagebreak
\section{\texttt{BlocklibExecutorService} Klassendiagramm}\label{appendix:BlocklibExecutorService}
{
	\centering
	\includesvg[width=\textwidth]{ExecuterInterfaces.svg}
	\todo{Caption}
}
\pagebreak
\section{Messungen}
\subsection{\glsentryshort{fps} Verläufe gesammelt}
\pgfplotsset{
	height=1.5cm,
	every axis title/.append style={fill=white,at={(.0,1)},yshift=-.2cm,draw=black,anchor=south west,nodes={scale=.7}},
}
\settowidth\mytemp{1,}
\fpsplot[title={Szenario 1: Hexagon (0)},ytick={0,200,...,1200},height=3cm,ymax=1200,y label style={at={(ticklabel cs:.5,-\mytemp)}},xticklabels={,,},xlabel={}]{seed-0-hexagon}
\\\fpsplot[title={Szenario 2: Halb-Würfel (0)},xticklabels={,,},xlabel={}]{seed-0-cube}
\\\fpsplot[title={Szenario 3: Welt-Statisch (0)},xticklabels={,,},xlabel={}]{seed-0-static}
\\\fpsplot[title={Szenario 4: Welt-Rotation (0)},xticklabels={,,},xlabel={}]{seed-0-rotate}
\\\fpsplot[title={Szenario 5: Welt-Gehen (0)},xticklabels={,,},xlabel={}]{seed-0-walk}
\\\fpsplot[title={Szenario 5: Welt-Gehen (3)},xticklabels={,,},xlabel={}]{seed-3-walk}
\\\fpsplot[title={Szenario 5: Welt-Gehen (10)},]{seed-10-walk}\\[-.5em]

\pagebreak
\subsection{\glsentryshort{cpu}-Auslastung Verläufe gesammelt}
\pgfplotsset{
	height=1.9cm,
}

\cpuplot[title={Szenario 1: Hexagon (0)},xticklabels={,,},xlabel={}]{seed-0-hexagon}
\\\cpuplot[title={Szenario 2: Halb-Würfel (0)},xticklabels={,,},xlabel={}]{seed-0-cube}
\\\cpuplot[title={Szenario 3: Welt-Statisch (0)},xticklabels={,,},xlabel={}]{seed-0-static}
\\\cpuplot[title={Szenario 4: Welt-Rotation (0)},xticklabels={,,},xlabel={}]{seed-0-rotate}
\\\cpuplot[title={Szenario 5: Welt-Gehen (0)},xticklabels={,,},xlabel={}]{seed-0-walk}
\\\cpuplot[title={Szenario 5: Welt-Gehen (3)},xticklabels={,,},xlabel={}]{seed-3-walk}
\\\cpuplot[title={Szenario 5: Welt-Gehen (10)},]{seed-10-walk}\\[-.5em]

\pagebreak
\subsection{\glsentryshort{gpu}-Auslastung Verläufe gesammelt}
\gpuplot[title={Szenario 1: Hexagon (0)},xticklabels={,,},xlabel={}]{seed-0-hexagon}
\\\gpuplot[title={Szenario 2: Halb-Würfel (0)},xticklabels={,,},xlabel={}]{seed-0-cube}
\\\gpuplot[title={Szenario 3: Welt-Statisch (0)},xticklabels={,,},xlabel={}]{seed-0-static}
\\\gpuplot[title={Szenario 4: Welt-Rotation (0)},xticklabels={,,},xlabel={}]{seed-0-rotate}
\\\gpuplot[title={Szenario 5: Welt-Gehen (0)},xticklabels={,,},xlabel={}]{seed-0-walk}
\\\gpuplot[title={Szenario 5: Welt-Gehen (3)},xticklabels={,,},xlabel={}]{seed-3-walk}
\\\gpuplot[title={Szenario 5: Welt-Gehen (10)},]{seed-10-walk}\\[-.5em]

\pagebreak
\subsection{\glsentryshort{ram}-Auslastung Verläufe gesammelt}
  				\memplot[title={Szenario 1: Hexagon (0)},xticklabels={,,},xlabel={}]{seed-0-hexagon-single-mem.csv}
\\[-1.5em]\memplot[xticklabels={,,},xlabel={}]{seed-0-hexagon-multi-mem.csv}
\\[-1.5em]\memplot[title={Szenario 2: Halb-Würfel (0)},xticklabels={,,},xlabel={}]{seed-0-cube-single-mem.csv}
\\[-1.5em]\memplot[xticklabels={,,},xlabel={}]{seed-0-cube-multi-mem.csv}
\\[-1.5em]\memplot[title={Szenario 3: Welt-Statisch (0)},xticklabels={,,},xlabel={}]{seed-0-static-single-mem.csv}
\\[-1.5em]\memplot[xticklabels={,,},xlabel={}]{seed-0-static-multi-mem.csv}
\\[-1.5em]\memplot[title={Szenario 4: Welt-Rotation (0)},xticklabels={,,},xlabel={}]{seed-0-rotate-single-mem.csv}
\\[-1.5em]\memplot[]{seed-0-rotate-multi-mem.csv}
\\[-1.5em]\memplot[title={Szenario 5: Welt-Gehen (0)},xticklabels={,,},xlabel={}]{seed-0-walk-single-mem.csv}
\\[-1.5em]\memplot[xticklabels={,,},xlabel={}]{seed-0-walk-multi-mem.csv}
\\[-1.5em]\memplot[title={Szenario 5: Welt-Gehen (3)},xticklabels={,,},xlabel={}]{seed-3-walk-single-mem.csv}
\\[-1.5em]\memplot[xticklabels={,,},xlabel={}]{seed-3-walk-multi-mem.csv}
\\[-1.5em]\memplot[title={Szenario 5: Welt-Gehen (10)},xticklabels={,,},xlabel={}]{seed-10-walk-single-mem.csv}
\\[-1.5em]\memplot[]{seed-10-walk-multi-mem.csv}\\[-.5em]