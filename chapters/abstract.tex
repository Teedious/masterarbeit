\begin{abstract}
	\KOMAoptions{parskip=half}
	\noindent In der folgenden Arbeit wird eine nebenläufige Architektur für die in der Programmiersprache Java geschriebene 3D-Spielebibliothek Blocklib, die von Studierenden der Ostbayerischen Technischen Hochschule Regensburg entwickelt wird, entworfen und implementiert.

	Die Anforderungen der Blocklib an eine nebenläufige Architektur werden erörtert und basierend auf den in der Spielindustrie gängigen nebenläufigen Architekturen \glsentrylong{sot} und Jobsystem wird ein hybrides Architekturmodell entworfen, das Elemente beider Architekturen nutzt. Dieses Modell wird schließlich implementiert und verwendet bekannte Techniken wie Double-Buffering, um Wettkampfbedingungen zu verhindern. Ein Großteil der bestehenden nebenläufigen Strukturen wird durch die neue nebenläufige Architektur ersetzt, die nun für die Nutzung durch zukünftige Studierende zur Weiterentwicklung der Blocklib bereitsteht.

	Die Architektur teilt die Blocklib grundlegend in zwei nebenläufige Bereiche auf, ein System zur Simulation der Inhalte der Spielebibliothek und ein System, das für das Rendering der Grafik verantwortlich ist. Des Weiteren wird eine Schnittstelle bereitgestellt, die von der im Java-Umfeld bekannten Schnittstelle \code{ScheduledExecutorService} abgeleitet und somit vertraut ist, die dort gebotene Funktionalität aber um die Möglichkeit der Komposition von Jobs mittels der Schnittstelle \code{CompletionStage} erweitert. Auf diese Weise lassen sich komplexe Aufgaben in aufeinander aufbauende Jobs zerlegen, die automatisch nebenläufig ausgeführt werden.

	In einer Performanceanalyse wird die Leistung der neuen Architektur mit der alten Architektur in fünf verschiedenen Szenarien verglichen, indem Messdaten zu den Metriken Bildwiederholrate, Auslastung der \glsentryshort{cpu} und \glsentryshort{gpu} sowie Speicherverbrauch erhoben werden. Aus den Messungen lässt sich eine gesteigerte Leistung mit einer um \SI{172}{\percent} höheren Bildwiederholrate ermitteln. Die Daten zeigen zudem, dass durch das System eine stärkere Auslastung von \glsentryshort{cpu} und \glsentryshort{gpu} erreicht wird. Aber auch der Speicherverbrauch der Blocklib steigt in der neuen Architektur.
	
	Zuletzt lassen die Messwerte darauf schließen, dass sich die Auslastung und die Bildwiederholrate weiter erhöhen lassen. Das kann erreicht werden, indem die neue Architektur genutzt wird, um mehr Aufgaben in der Simulation der Blocklib nebenläufig durchzuführen. Es wird erwartet, dass dies die Leistung der Spielebibliothek steigern kann, da die gemessenen Werte der Auslastung der \glsentryshort{cpu} darauf schließen lassen, dass diese weiterhin durch mangelnde Nebenläufigkeit einen Flaschenhals darstellt, der das Leistungspotenzial der Blocklib einschränkt.
\end{abstract}