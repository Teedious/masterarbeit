Die Entwicklung einer nebenläufigen Architektur führt zu
 tiefgreifenden Änderungen in einer Codebasis, unabhängig davon, ob die Architektur auf \ac{sot} oder einem Jobsystem basiert. Insofern ist ein Verständnis für die bestehende Architektur der Codebasis nötig. Interessant ist dabei insbesondere wie die Game-Loop, beschrieben in Abschnitt~\ref{sec:gameLoop}, gestaltet ist und an welchen Stellen bereits von Multithreading Gebrauch gemacht wird, was in Abschnitt~\ref{sec:nutzungMultithreading} erläutert wird.

\subsubsection{Game-Loop}\label{sec:gameLoop}
Ein nahezu universelles Designelement von Spielen ist die sogenannte \emph{Game-Loop}~\cite[S.~161~\psqq]{Nystrom2015}. Sie ist für die Aktualisierung des Spiels zuständig und umfasst typischerweise die Aufgaben 
\begin{itemize}
  \item Verarbeitung der Eingaben,
  \item Berechnung des neuen Spielzustands und
  \item (grafische) Ausgabe.
\end{itemize}
Die Blocklib besitzt ebenfalls eine Game-Loop und führt die beschriebenen Aufgaben in der obigen Reihenfolge sequentiell aus. Genauer sind es zwei Game-Loops, eine für die Ausführung als Server (ohne graphische Ausgabe) und eine für einen Client oder Einzelspieler.
Listing~\ref{lst:oldGameLoop} zeigt eine vereinfachte Darstellung der Game-Loop der Einzelspieler-Blocklib.
Die typischen Elemente der Game-Loop sind vorhanden.

\begin{lstlisting}[caption={Vereinfachte Version der Blocklib für Einzelspieler.},label={lst:oldGameLoop},float={htbp}]
while(!shutdown){
	//...
	processInput(delta); // Verarbeitung der Eingaben
	//...
	update(delta); // Berechnung des neuen Spielzustands
	//...
	render(); // grafische Ausgabe
	//...
	window.update(); // Verarbeitung der Eingaben und grafische Ausgabe
}
\end{lstlisting}

Die Methode \code{update(...)} ruft ihrerseits die \code{update(...)}-Methoden der einzelnen Simulationssysteme auf (siehe Listing~\ref{lst:gameUpdate}).
Bei der Berechnung des neuen Spielzustands gibt es in der Blocklib kein zentrales System, das den Zustand des letzten Loopdurchlaufs vorhält. Somit ist es möglich, dass das Verhalten der Blocklib von der Reihenfolge der \code{update(...)}-Methoden abhängt, da der Zustand, auf den bei der Berechnung zugegriffen wird, zeitgleich verändert wird. Das kann anhand eines Beispiels veranschaulicht werden. 

\begin{lstlisting}[caption={Vereinfachte Update-Methode von \classGame{}.}, label={lst:gameUpdate},float={htbp}]
private void update(float delta) {
	//...
	Context con = Context.getInstance();
	//...
	con.getChunkManager().update(delta);
	con.getEffectManager().update(delta);
	//...
	con.getAudioManager().update();
	con.getMainScheduler().update(delta);
	con.getEntityManager().update(delta);
	con.getFluidManager().update(delta);
	//...
}
\end{lstlisting}

\begin{example}
Man nehme an, dass der \classEntityManager{} prüft, welche Chunks geladen sind, um zu ermitteln, wo ein neuer Gegner erschaffen werden soll.
Der \classChunkManager{} bestimmt, welche Chunks abhängig von der aktuellen Kameraposition geladen werden.
Wird nun ein Chunk entfernt, auf dem der \classEntityManager{} einen Gegner erschaffen hat, könnte das zu unvorhergesehenem Verhalten führen, da der \classEntityManager{} erwartet, dass der erschaffene Gegner auf einem existenten Chunk erstellt wurde.
\end{example}

Die (mögliche) Abhängigkeit des Verhaltens von der Reihenfolge der Updates, zeigt, dass sich diese Berechnungen nicht problemlos parallelisieren lassen, da man den gesamten Simulationscode auf mögliche Wettkampfbedingungen prüfen müsste.


\subsubsection{Nutzung von Multithreading}\label{sec:nutzungMultithreading}
Bereits vor den Änderungen in dieser Arbeit werden in der Blocklib Threads genutzt, um nebenläufige \glspl{Anweisung} zu parallelisieren. Darunter fallen insbesondere zum einen Hintergrundaufgaben, wie das Erstellen von Chunks, und zum anderen das Ausführen zeitlich gesteuerter Abläufe, wie die Updates von Flüssigkeiten. Insgesamt gibt es zu Beginn der Arbeit sieben Klassen, die Threads erzeugen und nutzen. Eine Auflistung der Klassen und kurzer Beschreibungen, wie diese Nebenläufigkeit nutzen, ist in Tabelle~\ref{tab:concTasksBlocklib} zu finden.

\begin{table}
	%fixup, because hyphenation of lstinline breaks width measurement
	\settowidth{\mytemp}{\texttt{NioClientNetworkLayer}}
	\renewcommand{\arraystretch}{1.5}
	\begin{tblr}{colspec={p{\mytemp}X}}
		\toprule
		Klasse & Bestehende Nutzung von Nebenläufigkeit \\
		\midrule
		\classChunkStorage{} & Executors werden genutzt, um die Generierung von Chunks und das Meshing von Chunks auszulagern.\\
		\classEventManager{} & Eventhandling wird in Threads ausgelagert, um Deadlocks zu verhindern, wenn die benachrichtigten Beobachter versuchen, auf dieselben kritischen Bereiche zuzugreifen. \\
		\classConnectionInfo{} & Im Server wird ein wiederkehrender Task gestartet, um Ping Nachrichten an die Clients zu senden.\\
		\classNioClientNetworkLayer{} & Ein Thread, der im Hintergrund mit dem Server kommuniziert, wird gestartet.\\
		\classNioServerNetworkLayer{} & Ein Thread, der im Hintergrund mit den Clients kommuniziert, wird gestartet.\\
		\classFluid{} & Ein Thread, der den Zustand der Flüssigkeit berechnet, wird gestartet.\\
		\classAudioManager{} & Die Klasse \classTimer{} wird verwendet, Hintergrundmusik abzuspielen.\\
		\bottomrule 
	\end{tblr}
	\caption[Klassen mit nebenläufigen \glsentryplural{Anweisung} in der Blocklib.]{Klassen mit nebenläufigen \glspl{Anweisung} in der Blocklib.}\label{tab:concTasksBlocklib}
\end{table}

Es gibt keine zentrale Stelle, an der die Erstellung von Threads verwaltet wird. Jede Klasse erstellt und beendet Threads eigenständig. Die Anzahl der genutzten Threads lässt sich somit nicht überblicken. Aus diesem Grund ist es schwierig sicherzustellen, dass die Anzahl der genutzten Threads die Anzahl der Prozessoren nicht überschreitet. Das führt zu unnötigen Kontextwechseln und verschlechtert somit die Leistung. Außerdem können die bestehenden Threads möglicherweise nicht voll ausgelastet werden, da nur einzelne Klassen auf diese Zugriff haben.