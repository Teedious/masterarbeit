Bei der Integration einer Multithreading-Architektur in ein bestehendes \gls{Programm} ist es notwendig, den aktuellen Stand der Software in die Entwurfsüberlegungen miteinzubeziehen. Aus diesem Grund ist die Entwicklung der nebenläufigen Architektur in mehrere Schritte unterteilt.

Zuerst wird die bestehende Architektur der Blocklib bezogen auf die Relevanz für die neue Architektur untersucht. Insbesondere wird die im Nachgang beschriebene Game-Loop der Blocklib sowie die bisherige Nutzung von Multithreading analysiert.

Anschließend werden die in der Analyse ermittelten Anforderungen an die neue Architektur beschrieben und die auf darauf basierenden Designentscheidungen der neuen nebenläufigen Architektur und deren Hintergründe erläutert.