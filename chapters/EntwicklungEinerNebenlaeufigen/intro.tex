Bei der Integration einer Multithreading Architektur in ein bestehendes Programm, ist es notwendig den aktuellen Stand der Software in die Designüberlegungen miteinzubeziehen. Daher ist die Entwicklung der nebenläufigen Architektur in mehrere Schritte unterteilt.

Zuerst wird die bestehende Architektur der Blocklib bezogen auf die Relevanz für die neue Architektur untersucht. Insbesondere wird die im Nachgang beschriebene Game Loop der Blocklib sowie die bisherige Nutzung von Multithreading analysiert.

Daraufhin werden die in der Analyse ermittelten Anforderungen an die neue Architektur beschrieben.

Darauf folgt die Erläuterung der auf Analyse und Anforderungen basierenden Designentscheidungen der neuen nebenläufigen Architektur und deren Hintergründe.

Abschließend werden die Details der Implementierung der Multithreading Architektur in der Blocklib beschrieben.