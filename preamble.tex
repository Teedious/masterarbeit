\usepackage{scrhack}
\usepackage[ngerman]{babel}
\usepackage[T1]{fontenc}
\usepackage[utf8]{inputenc}
\usepackage{lmodern}
\usepackage{microtype}
\usepackage{geometry}
\geometry{a4paper, top=27mm, left=30mm, right=20mm, bottom=35mm, headsep=10mm, footskip=12mm}

\usepackage{amsmath,amssymb,amsthm,mathrsfs,amsfonts,dsfont} 
\usepackage{csquotes}
\usepackage{booktabs}

\usepackage{graphicx}
\graphicspath{ {./img/} }
% \usepackage{layouts}
\usepackage{textcomp} 
\usepackage[pdftex,dvipsnames]{xcolor}  % Coloured text etc.

\usepackage[layout={margin,index},draft]{fixme}
\usepackage{rotating}
\usepackage{siunitx}

\usepackage[backend=biber,minbibnames=2,maxbibnames=2,maxcitenames=1,mincitenames=1,style=alphabetic]{biblatex}
\setcounter{biburlnumpenalty}{9000}
\setcounter{biburlucpenalty}{9000}
\setcounter{biburllcpenalty}{9000}

% new stretchable space between characters
\setlength{\biburlnumskip}{0mu plus 1mu}
\setlength{\biburlucskip}{0mu plus 1mu}
\setlength{\biburllcskip}{0mu plus 1mu}
\renewcommand{\namelabeldelim}{\addnbspace}
\usepackage{tikz}
\usetikzlibrary{calc,positioning, shapes, petri, automata}
\usepackage{pdfpages}
\usepackage{subcaption}
\usepackage{standalone}
\usepackage{multirow,tabularx}
\usepackage[hidelinks]{hyperref}
\usepackage[acronym,shortcuts]{glossaries}


%\usepackage{pgfplots}
%\pgfplotsset{width=\columnwidth,compat=1.14}\usepgfplotslibrary{statistics}
%\pgfplotsset{boxplot/.cd,every median/.style={red}}
%\pgfplotsset{grid style={help lines}}
%\pgfplotsset{minor grid style={very thin, dotted}}
%\pgfplotsset{major grid style={thick}}

\newcommand{\unsim}{\mathord{\sim}}
\newcommand{\e}{\ensuremath{\mathrm{e}}}

\newcommand{\R}{\ensuremath{\mathbb{R}}}
\newcommand{\N}{\ensuremath{\mathbb{N}}}
\newcommand{\F}{\ensuremath{\mathbb{F}}}
\newcommand{\Pot}{\ensuremath{\mathcal{P}}}


\newcommand{\rowvec}[2]{\begin{pmatrix}
#1&#2\\
\end{pmatrix}}

\newcommand{\colvec}[2]{\begin{pmatrix}
#1\\
#2
\end{pmatrix}}

\newcommand{\deactivateGlossaries}
{
    \renewcommand{\makenoidxglossaries}{}
    \renewcommand{\printnoidxglossaries}{}
}

%\deactivateGlossaries

\makenoidxglossaries

\usepackage{listings}
\lstset{basicstyle=\footnotesize, captionpos=b, breaklines=true, showstringspaces=false, tabsize=2, frame=lines, numbers=left, numberstyle=\tiny, xleftmargin=2em, framexleftmargin=2em}
% \makeatletter
% \def\l@lstlisting#1#2{\@dottedtocline{1}{0em}{1em}{\hspace{1,5em} Lst. #1}{#2}}
% \makeatother

\definecolor{javared}{rgb}{0.6,0,0} % for strings
\definecolor{javagreen}{rgb}{0.25,0.5,0.35} % comments
\definecolor{javapurple}{rgb}{0.5,0,0.35} % keywords
\definecolor{javadocblue}{rgb}{0.25,0.35,0.75} % javadoc
\definecolor{gray}{rgb}{0.6,0.6,0.6}
 
\lstset{language=Java,
basicstyle=\ttfamily\footnotesize,
keywordstyle=\color{javapurple}\bfseries,
stringstyle=\color{javared},
commentstyle=\color{javagreen}\itshape\bfseries,
morecomment=[s][\color{javadocblue}]{/**}{*/},
numbers=left,
numberstyle=\tiny\color{gray},
stepnumber=1,
numbersep=10pt,
tabsize=3,
showspaces=false,
showstringspaces=false}

\newcommand{\code}[1]{\texttt{#1}}

\tikzset{
    transV/.style={transition, fill=black, minimum height = 12mm, minimum width = 1.5mm,inner sep = 0mm},
    transH/.style={transition, fill=black, minimum width = 12mm, minimum height = 1.5mm,inner sep = 0mm},
    node distance=1.5
}

\newcommand{\todo}[1]{\fxnote{{\color{red}#1}}}
\newcommand{\TODO}[2]{\fxnote*{{\color{red}#1}}{\underline{\emph{#2}}}}