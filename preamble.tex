\usepackage{scrhack}
\usepackage[ngerman]{babel}
\usepackage[T1]{fontenc}
\usepackage[utf8]{inputenc}
\usepackage{lmodern}
\usepackage{microtype}
\usepackage{geometry}
\geometry{a4paper, top=27mm, left=20mm, right=20mm, bottom=35mm, headsep=10mm, footskip=12mm}
\usepackage{setspace}
\usepackage{amsmath,amssymb,amsthm,mathrsfs,amsfonts}

\usepackage{csquotes}
\usepackage{booktabs}

\usepackage{graphicx}
\graphicspath{ {./img/} }
\usepackage{wrapfig}
% \usepackage{layouts}
\usepackage{textcomp} 
\usepackage[pdftex,dvipsnames]{xcolor}  % Coloured text etc.
\usepackage{svg}
\svgsetup{inkscapelatex=false, inkscapearea=drawing}
\usepackage[layout={margin,index},draft]{fixme}
\usepackage{rotating}
\usepackage{siunitx}

\usepackage[backend=biber,minbibnames=2,maxbibnames=2,maxcitenames=1,mincitenames=1,style=alphabetic]{biblatex}
\setcounter{biburlnumpenalty}{9000}
\setcounter{biburlucpenalty}{9000}
\setcounter{biburllcpenalty}{9000}

% new stretchable space between characters
\setlength{\biburlnumskip}{0mu plus 1mu}
\setlength{\biburlucskip}{0mu plus 1mu}
\setlength{\biburllcskip}{0mu plus 1mu}
\renewcommand{\namelabeldelim}{\addnbspace}

\addbibresource{Recherche/My Collection.bib}
\defbibheading{Literatur}{\section{Literaturverzeichnis}} 
\defbibheading{Quellen}{\subsection*{Quellenverzeichnis}} 

\DeclareSourcemap{
  \maps{
    \map{
      \step[fieldsource=url,
            match=\regexp{\$\\sim\$},
            replace=\regexp{\~}]
    }
  }
}

\usepackage{tikz}
\usetikzlibrary{calc,positioning, shapes, petri, automata}
\usepackage{pdfpages}
\usepackage{subcaption}
\usepackage{standalone}
\usepackage{multirow,tabularx}
\usepackage[hidelinks]{hyperref}
\usepackage[acronym,shortcuts,toc]{glossaries}
\usepackage{mathtools}


\usepackage{pgfplots}
    \pgfplotsset{
        table/search path={performance/measurements},
    }

\newcommand{\unsim}{\mathord{\sim}}
\newcommand{\e}{\ensuremath{\mathrm{e}}}

\newcommand{\R}{\ensuremath{\mathbb{R}}}
\newcommand{\N}{\ensuremath{\mathbb{N}}}
\newcommand{\F}{\ensuremath{\mathbb{F}}}
\newcommand{\Pot}{\ensuremath{\mathcal{P}}}

\DeclareMathOperator{\reachOp}{E}
\newcommand{\E}[1]{\reachOp(#1)}


\newcommand{\rowvec}[2]{\begin{pmatrix}
#1&#2\\
\end{pmatrix}}

\newcommand{\colvec}[2]{\begin{pmatrix}
#1\\
#2
\end{pmatrix}}

\newcommand{\deactivateGlossaries}
{
    \renewcommand{\makenoidxglossaries}{}
    \renewcommand{\printnoidxglossaries}{}
}

%\deactivateGlossaries

\makenoidxglossaries

\usepackage{listings}
\lstset{basicstyle=\footnotesize, captionpos=b, breaklines=true, showstringspaces=false, tabsize=2, frame=lines, numbers=left, numberstyle=\tiny, xleftmargin=2em, framexleftmargin=2em}
% \makeatletter
% \def\l@lstlisting#1#2{\@dottedtocline{1}{0em}{1em}{\hspace{1,5em} Lst. #1}{#2}}
% \makeatother

\definecolor{javared}{rgb}{0.6,0,0} % for strings
\definecolor{javagreen}{rgb}{0.25,0.5,0.35} % comments
\definecolor{javapurple}{rgb}{0.5,0,0.35} % keywords
\definecolor{javadocblue}{rgb}{0.25,0.35,0.75} % javadoc
\definecolor{gray}{rgb}{0.6,0.6,0.6}
 
\lstset{
	language=Java,
	basicstyle=\ttfamily\footnotesize,
	keywordstyle=\color{javapurple}\bfseries,
	stringstyle=\color{javared},
	commentstyle=\color{javagreen}\itshape\bfseries,
	morecomment=[s][\color{javadocblue}]{/**}{*/},
	numbers=left,
	numberstyle=\tiny\color{gray},
	stepnumber=1,
	numbersep=10pt,
	tabsize=3,
	showspaces=false,
	showstringspaces=false
}

\lstdefinestyle{inline}{
	basicstyle=\ttfamily\normalsize,
	keywordstyle=\ttfamily\normalsize,
	breaklines=true,
	breakatwhitespace=true%,
	%literate={A}{}{0\discretionary{}{A}{A}}
  %         {B}{}{0\discretionary{}{B}{B}}
  %         {C}{}{0\discretionary{}{C}{C}}
  %         {D}{}{0\discretionary{}{D}{D}}
  %         {E}{}{0\discretionary{}{E}{E}}
  %         {F}{}{0\discretionary{}{F}{F}}
  %         {G}{}{0\discretionary{}{G}{G}}
  %         {H}{}{0\discretionary{}{H}{H}}
  %         {I}{}{0\discretionary{}{I}{I}}
  %         {J}{}{0\discretionary{}{J}{J}}
  %         {K}{}{0\discretionary{}{K}{K}}
  %         {L}{}{0\discretionary{}{L}{L}}
  %         {M}{}{0\discretionary{}{M}{M}}
  %         {N}{}{0\discretionary{}{N}{N}}
  %         {O}{}{0\discretionary{}{O}{O}}
  %         {P}{}{0\discretionary{}{P}{P}}
  %         {Q}{}{0\discretionary{}{Q}{Q}}
  %         {R}{}{0\discretionary{}{R}{R}}
  %         {S}{}{0\discretionary{}{S}{S}}
  %         {T}{}{0\discretionary{}{T}{T}}
  %         {U}{}{0\discretionary{}{U}{U}}
  %         {V}{}{0\discretionary{}{V}{V}}
  %         {W}{}{0\discretionary{}{W}{W}}
  %         {X}{}{0\discretionary{}{X}{X}}
  %         {Y}{}{0\discretionary{}{Y}{Y}}
  %         {Z}{}{0\discretionary{}{Z}{Z}}
}

\newcommand{\code}[1]{\lstinline[style=inline]!#1!}
%TODO fix overfull hboxes LATER
\newcommand{\class}[1]{\code{#1}}
\newcommand{\const}[1]{\code{#1}}

\lstset{escapeinside={(*}{*)}}

\newcommand{\LSset}[2]{\scriptsize $\begin{aligned}&\{#1\}_L\\&\{#2\}_S\end{aligned}$}


\tikzset{
    transV/.style={transition, fill=black, minimum height = 12mm, minimum width = 1.5mm,inner sep = 0mm},
    transH/.style={transition, fill=black, minimum width = 12mm, minimum height = 1.5mm,inner sep = 0mm},
    node distance=1.5
}

\DeclareSIUnit{\Bit}{bit}
\DeclareSIUnit{\Byte}{B}


\newcommand{\todo}[1]{\fxnote{{\color{red}#1}}}
\newcommand{\TODO}[2]{\fxnote*{{\color{red}#1}}{\underline{\emph{#2}}}}

\newlength{\wrapfigwidth}