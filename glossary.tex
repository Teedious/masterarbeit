\newglossaryentry{nichtAtomar}{
name={nicht-atomar},
plural={nicht-atomare},
user1={nicht-atomaren},
description={Beschreibung einer Anweisung, die selbst aus weiteren Anweisungen besteht.}
}
\newglossaryentry{compositionRoot}{
name={Composition-Root},
description={Position in einer Anwendung, an der Module zusammengesetzt werden.}
}
\newglossaryentry{dependencyInjection}{
name={Dependency-Injection},
description={Designarchitektur, in der Module ihre Abhängigkeiten nicht selbst erzeugen, sondern diese Aufgabe an aufrufende Module übertragen.}
}
\newglossaryentry{servicelocator}{
name={Service-Locator},
description={Designpattern zur dynamischen Lokalisierung von Abhängigkeiten über ein zentrales Modul.}
}
\newglossaryentry{singleton}{
name={Singleton},
description={Designpattern, mit dem ein Objekt global verfügbar gemacht und dessen mehrfache Instanziierung verhindert wird.}
}
\newglossaryentry{testdouble}{
name={Test-Double},
description={Spezielles Modul, das das Verhalten einer Komponente vorspielt, damit das Original beim Testen nicht benötigt wird.}
}
\newglossaryentry{uniform}{
name={Uniform},
description={Globale Variable innerhalb eines \glsuseri{shaderprogram}.}
}
\newglossaryentry{shaderprogram}{
name={Shader-Programm},
plural={Shader-Programme},
user1={Shader-Programms},
description={\gls{Programm}, das auf der \ac{gpu} ausgeführt wird.}
}
\newglossaryentry{Programm}{
name={Programm},
plural={Programme},
user1={Programms},
user2={Programmen},
description={Konkrete Niederschrift eines Algorithmus.}
}
\newglossaryentry{Anweisung}{
name={Anweisung},
plural={Anweisungen},
description={Bezeichnung für die Elemente, aus denen ein \gls{Programm} besteht.}
}
\newglossaryentry{Rechenprozess}{
name={Rechenprozess},
plural={Rechenprozesse},
user1={Rechenprozesses},
user2={Rechenprozessen},
description={Eine Sequenz von \glspl{Aktivitaet}, die ein isoliertes Problem abarbeitet.}
}
\newglossaryentry{Aktivitaet}{
name={Aktivität},
plural={Aktivitäten},
description={Durchlaufener Einzelschritt bei der Ausführung eines \glsuseri{Programm}.}
}





\newacronym[]{fps}{FPS}{Frames per Second (dt. Bilder pro Sekunde)}
\DeclareSIUnit{\fps}{\ac{fps}}
\newacronym[]{sot}{SoT}{System on a Thread}
%Prozessor Hauptprozessor
\newacronym[]{cpu}{CPU}{Hauptprozessor (engl. central processing unit)}
\newacronym[user1={Grafikkarte}]{gpu}{GPU}{Grafikkarte (engl. graphics processing unit)}
\newacronym[]{ram}{RAM}{Hauptspeicher (engl. random access memory)}
\newacronym[]{vao}{VAO}{Vertex-Array-Object}
\newacronym[]{gui}{GUI}{grafische Benutzeroberfläche (engl. grafical user interface)}
