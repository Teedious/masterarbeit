% !TeX spellcheck = de_DE
% !LuaLaTeX
\documentclass[12pt,a4paper,listof=totocnumbered,parskip=half,numbers=noendperiod]{scrreprt}
\usepackage{scrhack}
\usepackage[ngerman]{babel}
\usepackage[T1]{fontenc}
\usepackage[utf8]{inputenc}
\usepackage{lmodern}
\usepackage{microtype}
\usepackage{geometry}
\geometry{a4paper, top=27mm, left=20mm, right=20mm, bottom=35mm, headsep=10mm, footskip=12mm}

\usepackage{amsmath,amssymb,amsthm,mathrsfs,amsfonts}

\usepackage{csquotes}
\usepackage{booktabs}

\usepackage{graphicx}
\graphicspath{ {./img/} }
\usepackage{wrapfig}
% \usepackage{layouts}
\usepackage{textcomp} 
\usepackage[pdftex,dvipsnames]{xcolor}  % Coloured text etc.
\usepackage{svg}
\svgsetup{inkscapelatex=false, inkscapearea=drawing}
\usepackage[layout={margin,index},draft]{fixme}
\usepackage{rotating}
\usepackage{siunitx}

\usepackage[backend=biber,minbibnames=2,maxbibnames=2,maxcitenames=1,mincitenames=1,style=alphabetic]{biblatex}
\setcounter{biburlnumpenalty}{9000}
\setcounter{biburlucpenalty}{9000}
\setcounter{biburllcpenalty}{9000}

% new stretchable space between characters
\setlength{\biburlnumskip}{0mu plus 1mu}
\setlength{\biburlucskip}{0mu plus 1mu}
\setlength{\biburllcskip}{0mu plus 1mu}
\renewcommand{\namelabeldelim}{\addnbspace}

\addbibresource{Recherche/My Collection.bib}
\defbibheading{Literatur}{\chapter{Literaturverzeichnis}} 
\defbibheading{Quellen}{\section*{Quellenverzeichnis}} 

\DeclareSourcemap{
  \maps{
    \map{
      \step[fieldsource=url,
            match=\regexp{\$\\sim\$},
            replace=\regexp{\~}]
    }
  }
}

\usepackage{tikz}
\usetikzlibrary{calc,positioning, shapes, petri, automata}
\usepackage{pdfpages}
\usepackage{subcaption}
\usepackage{standalone}
\usepackage{multirow,tabularx}
\usepackage[hidelinks]{hyperref}
\usepackage[acronym,shortcuts,toc]{glossaries}
\usepackage{mathtools}


%\usepackage{pgfplots}
%\pgfplotsset{width=\columnwidth,compat=1.14}\usepgfplotslibrary{statistics}
%\pgfplotsset{boxplot/.cd,every median/.style={red}}
%\pgfplotsset{grid style={help lines}}
%\pgfplotsset{minor grid style={very thin, dotted}}
%\pgfplotsset{major grid style={thick}}

\newcommand{\unsim}{\mathord{\sim}}
\newcommand{\e}{\ensuremath{\mathrm{e}}}

\newcommand{\R}{\ensuremath{\mathbb{R}}}
\newcommand{\N}{\ensuremath{\mathbb{N}}}
\newcommand{\F}{\ensuremath{\mathbb{F}}}
\newcommand{\Pot}{\ensuremath{\mathcal{P}}}

\DeclareMathOperator{\reachOp}{E}
\newcommand{\E}[1]{\reachOp(#1)}


\newcommand{\rowvec}[2]{\begin{pmatrix}
#1&#2\\
\end{pmatrix}}

\newcommand{\colvec}[2]{\begin{pmatrix}
#1\\
#2
\end{pmatrix}}

\newcommand{\deactivateGlossaries}
{
    \renewcommand{\makenoidxglossaries}{}
    \renewcommand{\printnoidxglossaries}{}
}

%\deactivateGlossaries

\makenoidxglossaries

\usepackage{listings}
\lstset{basicstyle=\footnotesize, captionpos=b, breaklines=true, showstringspaces=false, tabsize=2, frame=lines, numbers=left, numberstyle=\tiny, xleftmargin=2em, framexleftmargin=2em}
% \makeatletter
% \def\l@lstlisting#1#2{\@dottedtocline{1}{0em}{1em}{\hspace{1,5em} Lst. #1}{#2}}
% \makeatother

\definecolor{javared}{rgb}{0.6,0,0} % for strings
\definecolor{javagreen}{rgb}{0.25,0.5,0.35} % comments
\definecolor{javapurple}{rgb}{0.5,0,0.35} % keywords
\definecolor{javadocblue}{rgb}{0.25,0.35,0.75} % javadoc
\definecolor{gray}{rgb}{0.6,0.6,0.6}
 
\lstset{language=Java,
basicstyle=\ttfamily\footnotesize,
keywordstyle=\color{javapurple}\bfseries,
stringstyle=\color{javared},
commentstyle=\color{javagreen}\itshape\bfseries,
morecomment=[s][\color{javadocblue}]{/**}{*/},
numbers=left,
numberstyle=\tiny\color{gray},
stepnumber=1,
numbersep=10pt,
tabsize=3,
showspaces=false,
showstringspaces=false}

\newcommand{\code}[1]{\lstinline[basicstyle=\ttfamily\normalsize,
keywordstyle=\ttfamily\normalsize]!#1!}
\newcommand{\class}[1]{\code{#1}}
\newcommand{\const}[1]{\code{#1}}

\lstset{escapeinside={(*}{*)}}

\newcommand{\LSset}[2]{\scriptsize $\begin{aligned}&\{#1\}_L\\&\{#2\}_S\end{aligned}$}


\tikzset{
    transV/.style={transition, fill=black, minimum height = 12mm, minimum width = 1.5mm,inner sep = 0mm},
    transH/.style={transition, fill=black, minimum width = 12mm, minimum height = 1.5mm,inner sep = 0mm},
    node distance=1.5
}

\newcommand{\todo}[1]{\fxnote{{\color{red}#1}}}
\newcommand{\TODO}[2]{\fxnote*{{\color{red}#1}}{\underline{\emph{#2}}}}

\newlength{\wrapfigwidth}
\newglossaryentry{nichtAtomar}{
name={nicht-atomar},
plural={nicht-atomare},
user1={nicht-atomaren},
description={Beschreibung einer Anweisung, die selbst aus weiteren Anweisungen besteht}
}
\newglossaryentry{compositionRoot}{
name={Composition Root},
description={Position in einer Anwendung an der Module zusammengesetzt werden}
}
\newglossaryentry{dependencyInjection}{
name={Dependency Injection},
description={Designarchitektur, in der Module ihre Abhängigkeiten nicht selbst erzeugen, sondern diese Aufgabe an aufrufende Module übertragen}
}
\newglossaryentry{servicelocator}{
name={Service Locator},
description={Designpattern zur dynamischen Lokalisierung von Abhängigkeiten über eine zentrales Modul}
}
\newglossaryentry{singleton}{
name={Singleton},
description={Designpattern, mit dem ein Objekt global verfügbar gemacht und dessen mehrfache Instanziierung verhindert wird}
}
\newglossaryentry{testdouble}{
name={Test Double},
description={spezielles Modul, das das Verhalten einer Komponente vorspielt, damit das Original beim Testen nicht benötigt wird}
}

\newacronym[]{fps}{FPS}{Frames per Second (dt. Bilder pro Sekunde)}
\DeclareSIUnit{\fps}{\ac{fps}}
\newacronym[]{sot}{SoT}{System on a Thread}
\newacronym[]{cpu}{CPU}{Prozessor (Central Processing Unit)}
\newacronym[user1={Grafikkarte}]{gpu}{GPU}{Grafikkarte (Graphics Processing Unit)}
\newacronym[]{ram}{RAM}{Hauptspeicher (Random Access Memory)}
\begin{document}
\titlehead{\includegraphics[height=2cm]{img/IM_LOGO.pdf}}

\subject{Masterarbeit}

\title{Konzeption und Integration einer Multithreading- und einer Test-API für eine 3D-Spielbibliothek sowie Analyse ihres Einflusses auf die Performance}

\subtitle{}

\author{
Florian Loher \textit{Technical University of Applied
Science Regensburg} \\
florian.loher@st.oth-regensburg.de
}

\date{30. Februar 2022}

\publishers{
    \setlength{\extrarowheight}{.3ex}
    \noindent\begin{tabular}{@{}ll}
        Fakultät: & Informatik und Mathematik\\
        \TODO{Fach}: & Informatik\\
        Abgabe: & \today\\
        Betreuer: & Prof.\ Dr.\ rer.\ nat.\ Carsten Kern
  \end{tabular}
}

\maketitle
\null\thispagestyle{empty}\clearpage
\tableofcontents
\chapter{Einleitung}

\setstretch{1.15}

\section{Blocklib}
\section{Nebenläufige Programmierung}
\section{Testen und Testframeworks}

\chapter{Grundlagen}
Es gibt einige Ansätze zur Modellierung verteilter oder nebenläufiger \glsplural{Programm}. Darunter finden sich auch Petri-Netze~\cite{Murata1989}. Mit einem Petri-Netz kann ein modelliertes \glstext{Programm} mathematisch formal definiert und analysiert werden. Der Begriff Petri-Netz beschreibt eine Familie von verwandten Modellen. Das hier im Folgenden beschriebene Petri-Netz-Modell wird auch Platz-Transitions-Netz (PT-Netz) genannt. Formal lässt sich ein PT-Netz $N$ als 5-Tupel $ N=(P,T,F,W,m_0)$ beschreiben. Dabei ist 
\begin{align*}
	&P  \quad \text{eine endliche Menge von sogenannten \emph{Plätzen},}\\
	&T  \quad \text{eine endliche Menge von sogenannten \emph{Transitionen},}\\
	&F \subseteq (P\times T) \cup (T \times P) \quad \text{die Menge der Relationen zwischen Plätzen und Transitionen,}\\
	&W: F \mapsto \N  \quad \text{eine Gewichtungsfunktion der Relationen und}\\
	&m_0: P \mapsto \N_0   \quad \text{die anfängliche \emph{Markierung} der Plätze~\cite{Murata1989}.}
\end{align*}
Der durch ein PT-Netz beschriebene Graph kann auch grafisch dargestellt werden. Die Plätze $p \in P$ werden dabei als Kreise dargestellt, die Transitionen $ t \in T$ durch schwarz gefüllte Rechtecke, die Relationen $ f \in F$ durch gerichtete Kanten zwischen den Kreisen und Rechtecken, wobei die Elemente der Gewichtungsfunktion $W$ an die jeweiligen Kanten gesetzt werden. Die Markierung wird durch schwarze Punkte, die \emph{Marken} genannt werden, in den Plätzen dargestellt. Abbildung~\ref{fig:petrinet} zeigt ein simples Beispiel eines PT-Netzes, sowohl formal beschrieben (Abbildung~\ref{fig:petrinet:formal}) als auch seine grafische Repräsentation (Abbildung~\ref{fig:petrinet:graph}).
\begin{figure}
	\centering
	\begin{subfigure}[b]{.48\textwidth}
		$\begin{aligned}
			N &= (P,T,F,W,m_0)\\
			P &= \{p_1, p_2\}\\
			T &= \{t_1, t_2\}\\
			F &= \{(p_1, t_1), (p_2, t_2), (t_1, p_2)\}\\
			W &= \{((p_1, t_1),1), ((p_2, t_2),2), ((t_1, p_2), 1)\}\\
			m_0 &= \{(p_1, 1), (p_2, 0)\}
		\end{aligned}$
		\subcaption{Formale Definition des PT-Netzes.}\label{fig:petrinet:formal}
	\end{subfigure}
	\hfill
	\begin{subfigure}[b]{.48\textwidth}
		\begin{tikzpicture}
			\node[place, label=$p_1$, tokens=1] (p1) at (0,0) {};
			\node[transV,label=$t_1$, right = of p1] (t1){};
			\node[place, label=$p_2$, right = of t1] (p2) {};
			\node[transV,label=$t_2$, right = of p2] (t2){};

			\draw 
			(p1) edge[post] node[above] {1} (t1)
			(t1) edge[post] node[above] {2} node[below=1cm] {$N$} (p2)
			(p2) edge[post] node[above] {1} (t2);
		\end{tikzpicture}
		\subcaption{Graph-Repräsentation des PT-Netzes.}\label{fig:petrinet:graph}
	\end{subfigure}
	\caption{Beispiel eines PT-Netzes mit zwei Plätzen und zwei Transitionen.}\label{fig:petrinet}
\end{figure}

\subsubsection{Transitionsregel} Um das Verhalten von \glsuserii{Programm} beschreiben zu können, kann die Markierung eines Petri-Netzes anhand der folgenden sogenannten \emph{Transitions-Regel} geändert werden~\cite{Murata1989}.
\begin{enumerate}
	\item Eine Transition $t_i \in T$ heißt \emph{aktiviert}, wenn für alle $(p,t_i) \in F $ gilt: $ W((p,t_i)) \leq M(p)$ wobei $M$ die aktuelle Markierung des Petri-Netzes ist. Es müssen also in allen Plätzen mit eingehenden Kanten genügend Marken bezüglich der Gewichtungsfunktion existieren. Diese Plätze werden auch \emph{Vorbedingungen} für die Transition genannt. Wenn ein Platz genügend Marken für eine Transition enthält, nennt man die Vorbedingungen (bezüglich dieser Transition) \emph{erfüllt}.
	\item Eine aktivierte Transition kann, muss aber nicht \emph{feuern}.
	\item Feuert eine aktivierte Transition $t_i$, geht eine Markierung $M$ in eine Markierung $M'$ über. Dies erfolgt nach den folgenden Regeln.
	\begin{enumerate}
		\item $P$ wird in die disjunkten Mengen $P_\emptyset, P_\rightarrow, P_\leftarrow, P_\leftrightarrow$ unterteilt, wobei 
		\item $P_\emptyset = \{p \in P | (p,t_i) \notin F \land (t_i,p) \notin F\}$ die unbeteiligten Plätze sind,
		\item $P_\rightarrow = \{p \in P | (p,t_i) \in F \land (t_i,p) \notin F\}$ die Plätze mit eingehenden Kanten zur Transition,
		\item $P_\leftarrow = \{p \in P | (p,t_i) \notin F \land (t_i,p) \in F\}$ die Plätze mit ausgehenden Kanten aus der Transition und 
		\item $P_\leftrightarrow = \{p \in P | (p,t_i) \in F \land (t_i,p) \in F\}$ die Plätze mit eingehenden und ausgehenden Kanten.
		\item Dann gilt $
			M'(p) = \left\{ 
				\begin{aligned}
					& M(p) && \; , \; p \in P_\emptyset\\
					& M(p)-W((p,t_i)) && \; , \; p \in P_\rightarrow\\
					&M(p)+W((t_i,p)) && \; , \; p \in P_\leftarrow\\
					& M(p)-W((p,t_i))+W((t_i,p)) && \; , \; p \in P_\leftrightarrow
				\end{aligned}
				\right\}
		$
	\end{enumerate}
\end{enumerate}
Betrachtet man Abbildung~\ref{fig:petrinet} erneut, lässt sich erkennen, dass Transition $t_1$ aktiviert ist und Transition $t_2$ nicht, da sich in $p_2$ keine Marke befindet.

\subsubsection{Erreichbarkeitsgraph}
Mittels der Transitionsregel lassen sich in einem Petri-Netz $N$ ausgehend von der Anfangsmarkierung $m_0$ gegebenenfalls weitere Markierungen erzeugen. Durch die wiederholte Anwendung der Transitionsregel auf alle daraus entstehenden Markierungen lässt sich ein Graph erstellen, der alle von der Anfangsmarkierung aus erreichbaren Markierungen erhält. Dieser wird der \emph{Erreichbarkeitsgraph} $\E{N}$ des Petri-Netzes genannt.
\begin{figure}
	\begin{tikzpicture}[node distance=5mm,auto,state/.append style={rounded rectangle}]
		\node[state, initial] (q1) {$\{(p_1,1),(p_2,0)\}$};
		\node[state] (q2) [right=of q1] {$\{(p_1,0),(p_2,2)\}$};
		\node[state] (q3) [right=of q2] {$\{(p_1,0),(p_2,1)\}$};
		\node[state] (q4) [right=of q3] {$\{(p_1,0),(p_2,0)\}$};
		\path[->] 
		(q1) edge node {$t_1$} (q2)
		(q2) edge node {$t_2$} node[below=6mm] {Graph mit vollständigen Knotenbezeichnungen}(q3)
		(q3) edge node {$t_2$} (q4);

		\node[state,initial] (q5) [below=1.2cm of q1]{$(1,0)$};
		\node[state] (q6) [below=1.2cm of q2] {$(0,2)$};
		\node[state] (q7) [below=1.2cm of q3] {$(0,1)$};
		\node[state] (q8) [below=1.2cm of q4] {$(0,0)$};
		\path[->] 
		(q5) edge node {$t_1$} (q6)
		(q6) edge node {$t_2$} node[below=6mm] {Graph mit gekürzten Knotenbezeichnungen}(q7)
		(q7) edge node {$t_2$} (q8);
		\end{tikzpicture}
		\caption[Erreichbarkeitsgraph des Beispiel PT-Netzes.]{Erreichbarkeitsgraph des Petri-Netzes aus Abbildung~\ref{fig:petrinet}. Der Graph ist zweimal dargestellt. Oben enthalten die Knoten des Graphen die vollständige Auszeichnung der Markierung, unten wird die Kurzschreibweise für die Knotenbezeichnungen genutzt, in der die Position den Platz kodiert. Der \enquote{start} Pfeil kennzeichnet die anfängliche Markierung.}\label{fig:reachability}
\end{figure}
In Abbildung~\ref{fig:reachability} wird der Erreichbarkeitsgraph des in Abbildung~\ref{fig:petrinet} dargestellten Petri-Netzes gezeigt. Die Knoten des Graphen sind die Markierungen des Petri-Netzes, die durch die Transitionsregel erreicht werden können. Da die Plätze des Petri-Netzes nummeriert sind, kann die Markierung verkürzt geschrieben werden, indem der Index des Tupels der Knotenbezeichnung den Platz beschreibt und die Zahl die Anzahl der Marken des Platzes kennzeichnet. Der obere Bereich der Abbildung stellt den Graph mit den vollständigen Markierungsbezeichnungen dar. Darunter wird die verkürzte Schreibweise genutzt.

Ist die Anzahl der Marken pro Platz in jeder Markierung des Erreichbarkeitsgraphen maximal $1$, kann die Schreibweise weiter verkürzt werden, indem nur die Plätze, genauer gesagt die Indizes der Plätze, die eine Marke enthalten, genannt werden. Abbildung~\ref{fig:1-bpetrinet} zeigt eine Abwandlung des Petri-Netzes aus Abbildung~\ref{fig:petrinet}, das die beschriebene Eigenschaft erfüllt, sowie den zugehörigen Erreichbarkeitsgraphen in verkürzter Schreibweise.
\begin{figure}
	\centering
	\begin{subfigure}[b]{.48\textwidth}
		\centering
		\begin{tikzpicture}
			\node[place, label=$p_1$, tokens=1] (p1) at (0,0) {};
			\node[transV,label=$t_1$, right = of p1] (t1){};
			\node[place, label=$p_2$, right = of t1] (p2) {};
			\node[transV,label=$t_2$, right = of p2] (t2){};

			\draw 
			(p1) edge[post] node[above] {1} (t1)
			(t1) edge[post] node[above] {1} (p2)
			(p2) edge[post] node[above] {1} (t2);
		\end{tikzpicture}
		\subcaption{Abwandlung des Petri-Netzes aus Abbildung~\ref{fig:petrinet}. Die Gewichtungsfunktion ist für jede Relation maximal 1.}\label{fig:1-bpetrinet:graph}
	\end{subfigure}
	\hfill
	\begin{subfigure}[b]{.48\textwidth}
		\centering
		\begin{tikzpicture}[node distance=5mm,auto,state/.append style={rounded rectangle}]
			\node[state, initial] (q1) {$1$};
			\node[state] (q2) [right=of q1] {$2$};
			\node[state] (q3) [right=of q2] {};
			\path[->] 
			(q1) edge node {$t_1$} (q2)
			(q2) edge node {$t_2$} (q3);
		\end{tikzpicture}
		\subcaption{Erreichbarkeitsgraph, bei dem die Zahl(en) in den Knoten die  Indizes der Plätze beschreiben, die eine Markierung besitzen.}\label{fig:1-bpetrinet:reachability}
	\end{subfigure}
\caption{Petri-Netz und zugehöriger Erreichbarkeitsgraph, in dem die Anzahl der Markierungen pro Platz stets maximal 1 ist. }\label{fig:1-bpetrinet}
\end{figure} 
\subsubsection{Weitere Definitionen}
Um die Arbeit mit Petri-Netzen zu vereinfachen, werden nun noch einige weitere Definitionen eingeführt. Es sei ein Petri-Netz $N = (P,T,F,W,m_0)$ gegeben.

\begin{enumerate}
	\item Die Menge der Knoten $v$ mit Kanten zu einem Knoten $k \in P\cup T$ heißt \emph{Vorbereich} des Knotens $k$ und ist definiert als $^\circ k = \{v | (v,k) \in F\}$.
	\item Die Menge der Knoten $n$ mit Kanten von einem Knoten $k \in P\cup T$ heißt \emph{Nachbereich} des Knotens $k$ und ist definiert als $k^\circ  = \{n | (k,n) \in F\}$.
	\item Die Menge der Knoten einer Markierung $m$ mit mindesten $n$ Marken $\{x | m(x)\geq n\}$ wird $M_{\geq n}$ genannt.
	\item Das Maximum einer Markierung $m$, $\deg(m) \coloneqq \max \{m(p)|p\in P\}$, wird ihr \emph{Grad} genannt.
	\item Ein Petri-Netz $N$ heißt \emph{$n$-beschränkt}, wenn $n$ das Maximum der Grade der Markierungen des Erreichbarkeitsgraphen $(V,K)=\E{N}$ ist, also $n = \max\{\deg(m)| m\in V \}$.
	\item Gilt für eine Kante $((a, b),w) \in W$, dass $w = 1$, so kann im dazugehörigen Graphen die Beschriftung entfallen. Kanten ohne Beschriftung haben also ein implizites Gewicht von $1$.
\end{enumerate}



\subsubsection{Erweiterte Petri-Netze}
Um das Konzept von Variablenzugriffen einfacher zu modellieren, führen \textcite{Goel1990} ein erweitertes Petri-Netz-Modell ein. Dabei wird das 5-Tupel des Petri-Netzes um die folgenden Komponenten erweitert:
\begin{enumerate}
	\item eine Menge von Variablen $V$,
	\item eine Funktion $L: T \mapsto \Pot(V)$, die \emph{lesenden Zugriff} auf die Variablen modelliert, und
	\item eine Funktion $S: T \mapsto \Pot(V)$, die \emph{schreibenden Zugriff} auf die Variablen modelliert.
\end{enumerate}
$\Pot(V)$ ist dabei die Potenzmenge von $V$, also die Menge aller Teilmengen von $V$.
Ein Beispiel für ein erweitertes Petri-Netz, das ansonsten identisch zu dem Petri-Netz in Abbildung~\ref{fig:petrinet} ist, ist in Abbildung~\ref{fig:augpetrinet} gegeben. Die formale Definition in Abbildung~\ref{fig:augpetrinet:formal} zeigt die neuen Mengen $V$, $L$ und $S$. Der Graph in Abbildung~\ref{fig:augpetrinet:graph} zeigt die Lese- und Schreibmengen unter Transition $t_1$. Die Lesemenge wird durch ein $L$ gekennzeichnet, die Schreibmenge durch ein $S$. 
\begin{figure}
\centering
	\begin{subfigure}[b]{.49\textwidth}
		$\begin{aligned}
			N &= (P,T,F,W,m_0, V, L, S)\\
			P &= \{p_1, p_2\}\\
			T &= \{t_1, t_2\}\\
			F &= \{(p_1, t_1), (p_2, t_2), (t_1, p_2)\}\\
			W &= \{((p_1, t_1),1), ((p_2, t_2),2), ((t_1, p_2), 1)\}\\
			m_0 &= \{(p_1, 1), (p_2, 0)\}\\
			V &= \{a,b\}\\
			L &= \{(t_1,\{a,b\}), (t_2,\emptyset)\}\\
			S &= \{(t_1,\{a\}), (t_2,\emptyset)\}
		\end{aligned}$
		\caption{Formale Definition eines erweiterten Petri-Netzes.}\label{fig:augpetrinet:formal}
	\end{subfigure}
	\hfill
	\begin{subfigure}[b]{.49\textwidth}
		\begin{tikzpicture}
			\node[place, label=$p_1$, tokens=1] (p1) at (0,0) {};
			\node[transV,label=$t_1$, right = of p1, label=below:{\LSset{a,b}{a}}] (t1){};
			\node[place, label=$p_2$, right = of t1] (p2) {};
			\node[transV,label=$t_2$, right = of p2] (t2){};

			\draw 
			(p1) edge[post] node[above] {1} (t1)
			(t1) edge[post] node[above] {2} node[below=2cm] {$N$} (p2)
			(p2) edge[post] node[above] {1} (t2);
		\end{tikzpicture}
		\caption{Graph-Repräsentation des erweiterten Petri-Netzes.}\label{fig:augpetrinet:graph}
	\end{subfigure}
	\caption[Beispiel eines erweiterten Petri-Netzes.]{Beispiel eines erweiterten Petri-Netzes. In (a) ist die formale Definition gegeben, in (b) der dazugehörige Graph. In Platz $p_2$ wird auf die Variablen $a$ und $b$ lesend und auf Variable $a$ schreibend zugegriffen.}\label{fig:augpetrinet}
\end{figure}
Nachdem nun die formalen Grundlagen eingeführt worden sind, werden in diesem Abschnitt Grundbegriffe des Bereichs Nebenläufigkeit erläutert, die notwendig für das Verständnis der nächsten Kapitel sind. Besonders wichtig sind insbesondere die Begriffe \enquote{Thread}, \enquote{Nebenläufigkeit} und \enquote{Wettkampfbedingung}, da diese Begriffe in der Arbeit intensiv behandelt werden.


\subsubsection{Definition von Thread}
Für die Definition des Begriffs Thread müssen zuerst einige andere Grundbegriffe eingeführt werden. 
Als \gls{Programm} wird in dieser Arbeit die konkrete Niederschrift eines Algorithmus bezeichnet. Die Elemente, aus denen das \gls{Programm} besteht, werden als \glspl{Anweisung} bezeichnet. Bei der Ausführung eines \glsuseri{Programm} werden Einzelschritte durchlaufen, diese bezeichnet man als \glspl{Aktivitaet}. \glspl{Aktivitaet} sind also die Handlungen, die aufgrund der \glspl{Anweisung} der Programms ausgeführt werden. Eine Sequenz von \glspl{Aktivitaet}, die ein isoliertes Problem abarbeitet, wird \emph{Prozess} genannt. Um diese Definition des Prozessbegriffs von späteren Definitionen abzugrenzen, wird er im Folgenden \gls{Rechenprozess} genannt. Die Ausführungseinheit, auf der die Schritte eines \glsuseri{Rechenprozess} durchgeführt werden, wird \emph{Prozessor} genannt~\cite[S.~22]{Herrtwich1989}. In der Regel besitzt ein Rechner mehrere Ausführungseinheiten. Häufige gibt es dabei eine Recheneinheit, die üblicherweise die meisten \glspl{Rechenprozess} ausführt und andere Prozessoren koordiniert, den \ac{cpu}.

Wenn ein \gls{Programm} auf einem Betriebssystem ausgeführt wird, erzeugt das Betriebssystem einen isolierten Adressraum, in dem das \gls{Programm} ausgeführt wird. Ein \gls{Rechenprozess}, der auf diese Weise ausgeführt wird, wird in dieser Arbeit als \emph{(System-)Prozess} bezeichnet. \textcite[S.~125~ff.]{Tanenbaum2016} liefern eine gute Übersicht über Systemprozesse, dort als Prozesse bezeichnet. Systemprozesse können selbst weitere Systemprozesse starten, wenn das Betriebssystem dies zulässt. Diese Systemprozesse besitzen dann ihren eigenen Adressraum. Die Anzahl der Systemprozesse, die auf einem Rechner laufen, ist meist höher als die Anzahl der Prozessoren des Rechners. Somit können nicht alle Prozesse zur gleichen Zeit ausgeführt werden. Damit dennoch alle Prozesse voranschreiten können, wechseln moderne Betriebssysteme die Systemprozesse, die auf den Prozessoren des Rechners ausgeführt werden, in schneller Folge. Dabei muss das Betriebssystem einige zu den Systemprozessen gehörende Daten, den \emph{Prozesskontrollblock}, tauschen, sodass der jeweils gerade ausführende Systemprozess seinen eigenen Prozesskontrollblock zur Verfügung hat. Dieser Tausch wird \emph{Kontextwechsel} genannt~\cite[S.~59]{Tanenbaum2016}.

Das Speichern und Laden der Prozesskontrollblöcken im Rahmen der Kontextwechsel von Systemprozessen benötigt eine nicht zu vernachlässigende Menge an Zeit. Dieser Zeitverbrauch wird auch als \emph{Overhead} bezeichnet. Um den Overhead von Kontextwechseln zu vermindern, bieten moderne Betriebssysteme Systemprozessen die Möglichkeit \glspl{Rechenprozess} zu erzeugen, die denselben Adressraum wie der erzeugende Systemprozess besitzen. Ein auf diese Art erzeugter \gls{Rechenprozess} heißt \emph{Thread}. Da die Menge der thread-eigenen Daten, des \emph{Threadkontrollblocks}, deutlich geringer ist als die des Prozesskontrollblocks, erzeugt der Kontextwechsel zwischen Threads einen geringeren Overhead. Zudem können Threads aufgrund des gemeinsamen Adressraums einfacher auf geteilte Ressourcen zugreifen. Das vereinfacht die Kooperation zwischen diesen \glsuserii{Rechenprozess}~\cite[S.~139~ff.]{Tanenbaum2016}.

\paragraph{Anweisungen und Petri-Netze}
Um formal Eigenschaften eines \glsuseri{Programm} zu beschreiben, ist es möglich dieses mittels eines Petri-Netzes (automatisiert) zu modellieren. Dazu müssen Petri-Netz-Konstrukte genutzt werden, die die \glspl{Anweisung} des \glsuseri{Programm} abbilden können. Für die Analyse nebenläufiger \glspl{Programm} sind nur sogenannte \emph{Rendezvous}-\glspl{Anweisung} sowie Kontrollstrukturen relevant~\cite{Goel1990}. Für diese lassen sich Unter-Petri-Netze definieren, die deren Verhalten modellieren. Eine Anweisung entspricht dann einem bestimmten Unter-Petri-Netz. Eine Auflistung für die Modellierung nebenläufiger \glspl{Programm} benötigter Unter-Petri-Netze ist in \cite{Goel1990} zu finden.  Da die Modellierung von Verzweigungen durch Unter-Petri-Netze in \cite[Abbildung 3.1]{Goel1990} (siehe Abbildung~\ref{fig:supnetifelse:old}) semantisch nicht ganz korrekt ist, wird eine Korrektur in Abbildung~\ref{fig:supnetifelse} dargestellt. 
\begin{figure}
	\begin{subfigure}[t]{.49\textwidth}
		\centering
		\tikzset{external/export next=false}
		\begin{tikzpicture}[node distance=1.35cm,on grid, auto]
			\node[place, label=$p_1$] (p1) {};
			\node[transV, label=\texttt{if}, above right = of p1] (if) {};
			\node[transV, label=\texttt{else},below right = of p1] (else) {};
			\node[place, label=$p_2$, right = of if] (p2) {};
			\node[place, label=$p_3$, right = of else] (p3) {};
			\node[right = of p2] (dots1){$\dots$};
			\node[right = of p3] (dots2){$\dots$};
			\node[place, label=$p_4$, right = of dots1] (p4) {};
			\node[place, label=$p_5$, right = of dots2] (p5) {};
			\node[transV, label={[xshift=2mm]\texttt{end if}}, below right = of p4] (endif) {};
		
			\draw 
			(p1) edge[post] (if)
			(p1) edge[post] (else)
			(if) edge[post] (p2)
			(else) edge[post] (p3)
			(p2) edge[post] (dots1)
			(p3) edge[post] (dots2)
			(dots1) edge[post] (p4)
			(dots2) edge[post] (p5)
			(p4) edge[post] (endif)
			(p5) edge[post] (endif)
			;
		\end{tikzpicture}
		\subcaption{Nachbildung des Unter-Petri-Netzes aus \cite[Abbildung~3.1]{Goel1990}. Da sowohl $p_4$ als auch $p_5$ Vorbedingungen für \texttt{end if} sind, kann diese Transition nicht feuern, nachdem der \texttt{if}- oder \texttt{else}-Pfad durchlaufen wurden.}\label{fig:supnetifelse:old}
	\end{subfigure}
	\hfill
	\begin{subfigure}[t]{.49\textwidth}
		\centering
		\begin{tikzpicture}[node distance=1.35cm,on grid, auto]
			\node[place, label=$p_1$] (p1) {};
			\node[transV, label=\texttt{if}, above right = of p1] (if) {};
			\node[transV, label=\texttt{else},below right = of p1] (else) {};
			\node[right = of if] (dots1){$\dots$};
			\node[right = of else] (dots2){$\dots$};
			\node[place, label=$p_2$, right = of dots1] (p2) {};
			\node[place, label=$p_3$, right = of dots2] (p3) {};
			\node[transV, label=$t_1$, right = of p2] (t1) {};
			\node[transV, label=$t_2$, right = of p3] (t2) {};
			\node[place, label=$p_4$, below right = of t1] (p4) {};
			\node[transV, label=\texttt{end if}, right = of p4] (endif) {};
		
		
		
			\draw 
			(p1) edge[post] (if)
			(p1) edge[post] (else)
			(if) edge[post] (dots1)
			(else) edge[post] (dots2)
			(dots1) edge[post] (p2) 
			(dots2) edge[post] (p3) 
			(p2) edge[post] (t1)
			(p3) edge[post] (t2)
			(t1) edge[post] (p4)
			(t2) edge[post] (p4)
			(p4) edge[post] (endif)
			;
		\end{tikzpicture}
		\subcaption{Semantisch korrigiertes Unter-Petri-Netz. Die Transition \texttt{end if} besitzt nur die Vorbedingungen $p_4$. Somit kann sie feuern nachdem einer der beiden Pfade durchlaufen wurde.}\label{fig:supnetifelse:new}
	\end{subfigure}

	\caption[Semantische Korrektur des Unter-Petri-Netzes zur Modellierung von Verzweigungen.]{Semantische Korrektur des Unter-Petri-Netzes zur Modellierung von Verzweigungen nach \cite[Abbildung~3.1]{Goel1990}. In a) wird das ursprüngliche Unter-Petri-Netz gezeigt, in b) das korrigierte.}\label{fig:supnetifelse}
\end{figure}

Die in Abbildung~\ref{fig:supnetifelse:new} gezeigten Plätze $p_1,p_2,p_3,p_4$ und Transitionen $t_1,t_2$ sind Hilfselemente, die die semantische Korrektheit der Transitionen \code{if}, \code{else} und \code{end if} sicherstellen. Eine Markierung kann exklusiv nur von \code{if} oder von \code{else} zum Feuern verwendet werden. Durch die Transitionen $t_1$ und $t_2$ entsteht eine Markierung in $p_5$ wodurch \code{end if} feuern kann, egal ob anfangs \code{if} oder \code{else} gefeuert hat. $p_2$ und $p_3$ existierten, damit das Petri-Netz ein bipartiter Graph bleibt, sich also immer Plätze und Transitionen \enquote{abwechseln}. Die Modellierung eines \code{switch}-Statements erfolgt analog, indem die Anzahl der Pfade zwischen $p_1$ und $p_4$ angepasst wird.

\subsubsection{Nebenläufigkeit}\label{sec:nebenl}
Im vorherigen Abschnitt wurde beschrieben, dass verschiedene \glspl{Rechenprozess} unabhängig von einander \enquote{gleichzeitig} ablaufen können. Im Folgenden wird \enquote{gleichzeitig} präziser definiert. Ein \gls{Programm} wird häufig als \emph{Sequenz} von \glspl{Anweisung} verstanden. In der Definition von \gls{Programm} in dieser Arbeit wird bewusst auf das Wort Sequenz verzichtet, denn bei vielen \glsuserii{Programm} kann es bei bestimmten \glspl{Anweisung} irrelevant sein, in welcher Reihenfolge oder ob sie sogar gleichzeitig ausgeführt werden. Man betrachte das folgende \gls{Programm} in Listing~\ref{lst:squareSeqEx}, das die Quadratsumme zweier Ganzzahlen (im Folgenden auch Quadratsumme genannt) ausgibt.

\begin{lstlisting}[caption={[Beispiel eines \glsentryuseri{Programm}, das sequentiell die Summe von Quadraten zweier Ganzzahlen berechnet.]Beispiel eines \glsuseri{Programm} das die Summe von Quadraten zweier Ganzzahlen berechnet. Die Berechnung der Quadratzahlen wird nacheinander in einer fest definierten Sequenz durchgeführt.}, label={lst:squareSeqEx},float={!htbp}]
printSummedSquare(int a, int b){
  x <- a*a (*\label{lst:squareSeqEx:aa}*)
  y <- b*b (*\label{lst:squareSeqEx:bb}*)
  print(x+y) (*\label{lst:squareSeqEx:print}*)
}
\end{lstlisting}
Es ist leicht zu erkennen, dass es irrelevant ist, ob zuerst \code{x} oder \code{y} berechnet wird oder die Berechnungen simultan stattfinden. Einzig der Ausgabebefehl in Zeile~\ref{lst:squareSeqEx:print} darf erst ausgeführt werden, nachdem \code{x} und \code{y} ermittelt wurden. Zuzulassen, dass die Ausführreihenfolge in dem \gls{Programm} nicht definiert ist, führt dazu, dass das \gls{Programm} präziser der Problembeschreibung \enquote{gib die Quadratsumme zweier Ganzzahlen aus} entspricht, und ermöglicht auch eine potenziell schnellere Berechnung, da die Quadrate gleichzeitig berechnet werden können.

Paare oder Gruppen von \glspl{Anweisung}, die gleichzeitig oder in beliebiger Reihenfolge ausgeführt werden, heißen \emph{nebenläufig}. Da die Ausführungsreihenfolge der Anweisungen in Zeile~\ref{lst:squareSeqEx:aa} und Zeile~\ref{lst:squareSeqEx:bb} egal ist, dürfen diese Anweisungen auch nebenläufig definiert sein.

Eine Implementierung, die obige Problemstellung der Quadratsumme besser abbildet, könnte unter Nutzung von Nebenläufigkeit wie Listing~\ref{lst:squareConcEx} aussehen. Dabei wird auf die Schreibweise von \textcite[S.~16]{Herrtwich1989} zurückgegriffen.
\begin{lstlisting}[caption={[Beispiel eines \glsuseri{Programm} mit nebenläufigem Code in einem \code{conc}-Block.]Beispiel eines \glsuseri{Programm} mit nebenläufigem Code in einem \code{conc}-Block. Das \gls{Programm} gibt die Summe von zwei Quadratzahlen aus, wobei die Berechnung der Quadratzahlen nebenläufig stattfindet.}, label={lst:squareConcEx},float={!htbp}]
printSummedSquare(int a, int b){
  conc 
    x <- a*a ||
    y <- b*b
  conc end
  print(x+y)
}
\end{lstlisting}
In einem \code{conc}-Block (von englisch \emph{concurrent} -- nebenläufig), definiert durch \code{conc} und \code{conc end}, sind alle \glspl{Anweisung}, die durch \code{||} getrennt sind, als nebenläufig zu verstehen. 

Wie man an dem Beispiel in Listing~\ref{lst:squareConcEx} erkennen kann, beschäftigt sich Nebenläufigkeit besonders mit der Struktur und der Möglichkeit der Zusammensetzung voneinander unabhängiger \glspl{Anweisung}. Somit ist es Aufgabe der nebenläufigen Programmierung, Probleme oder Aufgaben in unabhängige Teile zu zerlegen und zu strukturieren~\cite{Pike2012,Hettel2016}. Blöcke von \glspl{Anweisung}, die (bezüglich der Aufgabe) nebenläufig sein könnten, aber als Anweisungssequenz definiert sind (siehe Zeile 2 und 3 in Listing~\ref{lst:squareSeqEx}), werden in dieser Arbeit als \emph{pseudo-sequentialisiert} bezeichnet.

Bei der Definition nebenläufiger Anweisungen ist Vorsicht geboten, da sichergestellt werden muss, dass zwei Anweisungen auch tatsächlich nebenläufig sein dürfen. Eine \gls{Anweisung} kann beispielsweise durch einen Compiler in mehrere \glspl{Anweisung} geteilt werden oder selbst schon aus mehreren \glspl{Anweisung} bestehen (man denke zum Beispiel an Funktionsaufrufe). Solche \glspl{Anweisung} werden \gls{nichtAtomar} genannt. Eine Menge von nebenläufigen \glspl{Anweisung}, von denen mindestens eine \gls{nichtAtomar} ist, kann \emph{verzahnt} ausgeführt werden. Eine Ausführung von \glspl{Anweisung} ist verzahnt, wenn zwischen der Ausführung der Teile einer \glsuseri{nichtAtomar} \gls{Anweisung} andere \glspl{Anweisung} ausgeführt werden. Abbildung \ref{fig:concAnweisungen} zeigt zur Veranschaulichung die verschiedenen Möglichkeiten, wie zwei nebenläufige \glspl{nichtAtomar} \glspl{Anweisung} im Zeitverlauf ausgeführt werden können, sowohl auf einem als auch auf zwei Prozessoren. Dadurch kann es passieren, dass Anweisungen, die zuerst den Anschein haben, nebenläufig sein zu können, nicht nebenläufig sein dürfen, weil eine verzahnte Ausführung der zugehörigen Aktivitäten ein unerwünschtes Ergebnis liefern würde. 

\begin{figure}[hbt]
	\centering
\newlength\aOne
\newlength\aTwo
\newlength\aThree
\newlength\bOne
\newlength\bTwo
\pgfmathsetlength{\aOne}{2cm}
\pgfmathsetlength{\aTwo}{2.25cm}
\pgfmathsetlength{\aThree}{2.5cm}
\pgfmathsetlength{\bOne}{2.75cm}
\pgfmathsetlength{\bTwo}{3cm}
\begin{subfigure}{\textwidth}
	\centering
\begin{tikzpicture}

	\draw[->] (0,0) -- (13,0) node[right] {Zeit};
	
	\node[draw,fill=red, anchor=south west, minimum width = \aOne] at (0, .5) (A1) {A Teil1};

	\node[draw,fill=red, anchor=west, minimum width = \aTwo] at (A1.east) (A2) {A Teil2};
	\node[draw,fill=red, anchor=west, minimum width = \aThree] at (A2.east) (A3) {A Teil3};
	
	\node[draw,fill=cyan, anchor=south west, minimum width = \bOne] at (0, 1.5) (B1) {B Teil1};
	\node[draw,fill=cyan, anchor=west, minimum width = \bTwo] at (B1.east) (B2) {B Teil2};

	\node[anchor=east] at (B1.west) {Prozessor 1};
	\node[anchor=east] at (A1.west) {Prozessor 2};
\end{tikzpicture}
\subcaption{Gleichzeitige Ausführung der \glspl{Anweisung} A und B auf zwei Prozessoren}
\end{subfigure}
\\[1.5em]
\begin{subfigure}{\textwidth}
	\centering
\begin{tikzpicture}

	\draw[->] (0,0) -- (13,0) node[right] {Zeit};
	
	\node[draw,fill=cyan, anchor=south west, minimum width = \bOne] at (0, .5) (B1) {B Teil1};
	\node[draw,fill=cyan, anchor=west, minimum width = \bTwo] at (B1.east) (B2) {B Teil2};
	\node[draw,fill=red, anchor=west, minimum width = \aOne] at (B2.east) (A1) {A Teil1};
	\node[draw,fill=red, anchor=west, minimum width = \aTwo] at (A1.east) (A2) {A Teil2};
	\node[draw,fill=red, anchor=west, minimum width = \aThree] at (A2.east) (A3) {A Teil3};

	\node[anchor=east] at (B1.west) {Prozessor 1};
\end{tikzpicture}
\subcaption{Sequenzielle Ausführung der \glspl{Anweisung} A und B auf einem Prozessor}
\end{subfigure}
\\[1.5em]
\begin{subfigure}{\textwidth}
	\centering
\begin{tikzpicture}

	\draw[->] (0,0) -- (13,0) node[right] {Zeit};
	
	\node[draw,fill=cyan, anchor=south west, minimum width = \bOne] at (0, .5) (B1) {B Teil1};
	\node[draw,fill=red, anchor=west, minimum width = \aOne] at (B1.east) (A1) {A Teil1};
	\node[draw,fill=red, anchor=west, minimum width = \aTwo] at (A1.east) (A2) {A Teil2};
	\node[draw,fill=cyan, anchor=west, minimum width = \bTwo] at (A2.east) (B2) {B Teil2};
	\node[draw,fill=red, anchor=west, minimum width = \aThree] at (B2.east) (A3) {A Teil3};

	\node[anchor=east] at (B1.west) {Prozessor 1};
\end{tikzpicture}
\subcaption{Verzahnte Ausführung der \glspl{Anweisung} A und B auf einem Prozessor}
\end{subfigure}

\caption[Mögliche Ausführungen nebenläufiger \glsentryplural{Anweisung}.]{Mögliche Ausführungen nebenläufiger \glspl{Anweisung} nach~\cite{Herrtwich1989}.}\label{fig:concAnweisungen}
\end{figure}

Werden nebenläufige \glspl{Anweisung} parallel ausgeführt, spricht man auch von \emph{Multiprocessing}. Handelt es sich bei den Prozessen um Threads, wird der Begriff \emph{Multithreading} genutzt.

\paragraph{Nebenläufigkeit und Petri-Netze}
Um eine eindeutige Bedeutung des hier genutzten Begriffs der Nebenläufigkeit festzulegen, wird dieser nun unter Verwendung des Petri-Netz-Formalismus definiert. Es ist wichtig anzumerken, dass der hier beschriebene Begriff von dem üblichen Verständnis der Nebenläufigkeit in Petri-Netzen abweicht. 

Normalerweise werden Petri-Netze als inhärent nebenläufig verstanden, da durch die Transitionsregel zu jeder Zeit jede beliebige Transition feuern kann, deren Vorbedingungen erfüllt sind. Der hier verwendete Begriff bezieht sich allerdings darauf, dass \glspl{Anweisung} unabhängig voneinander, ohne sich gegenseitig zu beeinflussen, ausführbar sind. Das Feuern einer Transition kann aber den Vorbereich einer anderen Transition verändern, wodurch sie beeinflusst wird.

In dieser Arbeit heißen zwei Transitionen $t_1 \neq t_2$ eines Petri-Netzes $N$ genau dann nebenläufig, wenn ${}^\circ t_1 \cap {}^\circ t_2 = \emptyset$ gilt und eine Markierung $m_\text{conc}$ im Erreichbarkeitsgraphen $\E{N}$ existiert, in der sowohl $t_1$ als auch $t_2$ aktiviert sind. Die Vorbereiche der Transitionen dürfen sich also nicht überschneiden.

\subsubsection{Folgen von Nebenläufigkeit}\label{sec:nebenl-folgen}
Wie in Abschnitt \ref{sec:nebenl} beschrieben, muss ein \gls{Programm} keine Sequenz von \glspl{Anweisung} sein, sondern kann auch nebenläufige \glspl{Anweisung} enthalten. Diese Möglichkeit hat eine Reihe von Folgen, die es bei der nebenläufigen Programmierung zu beachten gilt.
\paragraph{Nichtdeterminismus}
Enthält ein \gls{Programm} nebenläufige \glspl{Anweisung}, ist die Reihenfolge der daraus resultierenden \glspl{Aktivitaet} nicht definiert und kann sich bei jedem Programmdurchlauf ändern. Die Ausführung ist als direkte Folge der Nebenläufigkeit~\cite[S.~17~f.]{Herrtwich1989} \emph{nichtdeterministisch}. Erwartet man von einem \gls{Programm} \emph{Determiniertheit}\footnote{Es kann durchaus sein, dass Determiniertheit in einem \gls{Programm} nicht gewünscht ist. Man betrachte beispielsweise ein \gls{Programm}, das einen echten Zufallsgenerator beschreibt. Hier wäre Determiniertheit ein direkter Widerspruch zur Aufgabe des \glsuseri{Programm}.}, also die Eigenschaft, dass gleiche Eingaben immer zu den gleichen Ausgaben führen, ist Nichtdeterminismus in der Regel zu vermeiden. Determiniertheit kann aber auch bei einem Verzicht auf Determinismus sichergestellt werden. Liefert ein \gls{Programm} in jeder beliebigen Ausführreihenfolge dasselbe Ergebnis, ist es trotz Nichtdeterminismus weiterhin determiniert, weil die Ausführreihenfolge für das Ergebnis keine Rolle spielt~\cite[S.~18~f.]{Herrtwich1989}. 
\paragraph{Nichtreproduzierbarkeit}
Die Nachvollziehbarkeit des Programmablaufs wird erschwert, da das Wissen und die Kontrolle über die Ausführreihenfolge abgegeben wird~\cite[S.~20]{Herrtwich1989}. Tritt beispielsweise ein Fehler auf, kann die Ausführreihenfolge, die zu dem Fehler geführt hat, im Nachhinein nicht ermittelt werden. Dasselbe Problem ergibt sich, wenn ein nichtdeterminiertes \gls{Programm} eine Lösung ausgibt und nachvollzogen werden soll, welche Ausführreihenfolge zu diesem Ergebnis geführt hat. Insbesondere das Testen und Debugging von Software wird dadurch erschwert, da es unmöglich ist bei jedem Test dieselben Bedingungen herzustellen, sodass beispielsweise Tests Fehler nicht verlässlich aufzeigen, weil diese nur bei bestimmten Ausführreihenfolgen auftreten~\cite[S.~20]{Herrtwich1989}. Somit muss ein Test entweder alle möglichen Ausführreihenfolgen simulieren oder es muss anderweitig sichergestellt werden, dass die Ausführreihenfolge der nebenläufigen \glspl{Anweisung} keine Rolle spielt.
\paragraph{Wettkampfbedingungen}
Ressourcen (zum Beispiel Drucker, Variablen und Dateien) und insbesondere Daten spielen in \glsuserii{Programm} eine zentrale Rolle. Greift eine Gruppe von nebenläufigen \glspl{Anweisung} auf dieselben (geteilten) Daten zu, ist die Reihenfolge der Zugriffe, wie alle anderen \glspl{Aktivitaet} der nebenläufigen \glspl{Anweisung}, beliebig. Wenn mindestens eine der \glspl{Anweisung} schreibend auf die geteilten Daten zugreift (diese also verändert), hängt der Zustand der Ausgabe des \glsuseri{Programm} im Allgemeinen von der Reihenfolge der Ausführung ab. Diese Situation wird \emph{Wettkampfbedingung} (engl. Race Condition) genannt~\cite{Hettel2016}. 

Erwähnenswert ist hier, dass Wettkampfbedingungen nur auftreten können, wenn die geteilten Daten nebenläufig \emph{geändert} werden. Ausschließlich lesende Zugriffe sind unkritisch, da die Daten unabhängig von der Ausführreihenfolge immer identisch sind. Wenn Daten von \glspl{Anweisung} jederzeit in einem Zustand aufgefunden werden, der korrekt ist, werden sie als \emph{threadsicher} bezeichnet.

Formal können Wettkampfbedingungen auch mittels erweiterter Petri-Netze definiert werden. Dazu betrachtet man, ob Variablen des erweiterten Petri-Netzes sich in den Lese- und Schreibmengen von nebenläufigen Transitionen überschneiden. Diese Variablen werden \emph{kritische} Variablen genannt. Die Menge der kritischen Variablen, die zur Existenz von Wettkampfbedingungen führen, ist formal gegeben durch
\begin{align*}
	V_\text{krit} = \;\bigcup_{\mathclap{\substack{s,t\, \in T\\s \neq t\\s, t \text{ nebenläufig}}}} \;\left( \strut{S(s) \cap (S(t) \cup L(t))}\right).
\end{align*}

Diese Definition ergibt sich wie folgt. Es werdenalle Paare von Transitionen $s$ und $t$ betrachtet, wobei $s$ und $t$ nebenläufig und (deswegen) nicht identisch sind. Dann werden der Menge $V_\text{krit}$ die Variablen hinzugefügt, die in der Schreibmenge der einen Transition $S(s)$ und ebenfalls in der Schreib- oder Lesemenge der anderen Transition $(S(t) \cup L(t))$ enthalten sind.

Ein Paar von Transitionen $(t_1, t_2)$ eines erweiterten Petri-Netzes \emph{erzeugt} eine Wettkampfbedingung, wenn die Transitionen dazu führen, dass eine Variable in die Menge der kritischen Variablen aufgenommen wird. Eine Markierung $m$ des Erreichbarkeitsgraphen $\E{N}$ eines Petri-Netzes $N$ enthält eine Wettkampfbedingung, wenn ein Paar von Transitionen $(t_1, t_2)$ existiert, das eine Wettkampfbedingung erzeugt, und $t_1$ und $t_2$ in $m$ aktiviert sind.

\begin{figure}
	\centering
	\begin{tikzpicture}[node distance=2cm,on grid, auto]
		\node[place, label=$p_1$] (p1) {};
		\node[transV, label=$t_1$, right = of p1] (t1) {};
		\node[place, label=$p_2$, tokens=1, above right = of t1] (p2) {};
		\node[place, label=$p_3$, tokens=1, below right = of t1] (p3) {};
		\node[transV,label=$t_2$, right = of p2,label=below:{\LSset{a,{\color{red}b}}{a}} ] (t2){};
		\node[transV,label=$t_3$, right = of p3,label=below:{\LSset{b}{{\color{red}b}}}] (t3){};
		\node[place, label=$p_4$, right = of t2] (p4) {};
		\node[place, label=$p_5$, right = of t3] (p5) {};
		\node[transV, label=$t_4$, below right = of p4] (t4) {};
		\node[place, label=$p_6$, right = of t4] (p6) {};
	
	
	
		\draw 
		(p1) edge[post] (t1)
		(t1) edge[post] (p2)
		(t1) edge[post] (p3)
		(p2) edge[post] (t2)
		(p3) edge[post] (t3)
		(t2) edge[post] (p4)
		(t3) edge[post] (p5)
		(p4) edge[post] (t4)
		(p5) edge[post] (t4)
		(t4) edge[post] (p6)
		;
	\end{tikzpicture}
	\caption[Ein erweitertes Petri-Netz, dessen Markierung eine Wettkampfbedingung enthält.]{Ein erweitertes Petri-Netz, dessen Markierung eine Wettkampfbedingung enthält. Die Schreibmenge von $t_3$ enthält die Variable $b$, diese ist allerdings auch in der Lesemenge von $t_2$ enthalten. Die problematische Variable {\color{red}$b$} ist rot markiert. Das Auftauchen von $b$ in der Lesemenge von $t_3$ stellt kein Problem dar.}\label{fig:wettkampfpetri}
\end{figure}

Um das Konzept zu veranschaulichen, ist in Abbildung~\ref{fig:wettkampfpetri} ein erweitertes Petri-Netz gezeigt, dessen Markierung eine Wettkampfbedingung enthält. Die Plätze $p_2$ und $p_3$ enthalten je eine Markierung. Daher müssen nun die Transitionen betrachtet werden, die aktiviert sind. Diese sind $t_2$ und $t_3$. Wie in der Abbildung rot markiert ist, überschneiden sich die Lese- und Schreibmengen der Transitionen, weil beide die Variable $b$ enthalten. Dadurch gilt $V_\text{krit} = \{b\} \neq \emptyset$ und die gezeigte Markierung enthält eine Wettkampfbedingung.

Ein erweitertes Petri-Netz enthält eine Wettkampfbedingung, wenn mindestens eine Markierung seines Erreichbarkeitsgraphen eine Wettkampfbedingung enthält.

Aufgabe der nebenläufigen Programmierung ist es also, \glspl{Anweisung} so zu strukturieren, dass alle von nebenläufigen \glspl{Anweisung} genutzten Daten threadsicher sind oder Daten, die nicht threadsicher sind, explizit synchronisiert werden. Über die Modellierung von Petri-Netzen ausgedrückt, gilt es also die Struktur eines \glsuseri{Programm} so zu definieren, dass in dem Petri-Netz, welches das \gls{Programm} modelliert, keine Wettkampfbedingungen existieren.

\subsubsection{Synchronisierung}
Um das Auftreten von Wettkampfbedingungen beim schreibenden Zugriff von nebenläufigen \glspl{Anweisung} auf geteilte Ressourcen zu vermeiden, muss sichergestellt werden, dass der Endzustand nach der Ausführung der \glspl{Anweisung} unabhängig von der Ausführreihenfolge der daraus resultierenden \glspl{Aktivitaet} ist. Dieser Vorgang wird \emph{Synchronisierung} genannt~\cite[S.~4]{Maurer2019}.

\paragraph{Synchronisierungsanforderungen} Nach \textcite[S.~132~ff.]{Herrtwich1989} gibt es allgemein zwei Arten von Anforderungen an Synchronisierungsmechanismen, je nachdem welche Art von Zugriff auf geteilte Ressourcen stattfindet. Sie unterscheiden dabei zwischen \emph{kausal abhängigen} Relationen und \emph{kausal unabhängigen} Relationen zwischen nebenläufigen \glspl{Anweisung}. Ist \gls{Anweisung} $B$ kausal abhängig von \gls{Anweisung} $A$, so dürfen die \glspl{Aktivitaet} von $B$ erst ausgeführt werden, nachdem die \glspl{Aktivitaet} von $A$ abgeschlossen sind. Beispielsweise werden in der Blocklib Chunks im Hintergrund geladen. Benötigt eine \gls{Anweisung} eines Spielelements einen noch nicht geladenen Chunk, so muss die Ausführung der zugehörigen \gls{Aktivitaet} solange aufgeschoben werden, bis der Chunk geladen ist. Bei kausal unabhängigen Relationen muss sichergestellt werden, dass der Zugriff auf die Ressourcen nicht zur gleichen Zeit stattfindet.  Beispielsweise wird die Liste der zu zeichnenden Objekte in der Blocklib von verschiedenen Systemen angepasst. Hier muss also darauf geachtet werden, dass diese Anpassungen zeitlich getrennt stattfinden. Sonst könnten beispielsweise hinzugefügte Elemente verloren gehen, wenn zwei Threads parallel Elemente hinzufügen, weil ein Thread möglicherweise die Änderungen des anderen überschreibt.

\paragraph{Synchronisierungsmethoden} Es gibt mehrere Möglichkeiten, die Synchronisierung zwischen nebenläufigen \glspl{Anweisung} zu erreichen. \textcite{Michael1996} unterteilen Algorithmen diesbezüglich in zwei Kategorien: \emph{blockierende} und \emph{nicht-blockierende}. Blockierende Synchronisierungsalgorithmen definieren sogenannte \emph{kritische Bereiche}. Ein kritischer Bereich ist eine Menge von \glspl{Anweisung} von deren \glspl{Aktivitaet} nur eine gleichzeitig ausgeführt werden darf. Sind mehrere auszuführende \glspl{Aktivitaet} Teil eines kritischen Bereichs, muss durch \emph{Sperren} (engl. Locks) sichergestellt werden, dass nur eine der \glspl{Aktivitaet} gleichzeitig ausgeführt wird. Möchte ein \gls{Rechenprozess} einen kritischen Bereich betreten, muss er sich die Sperre zunächst aneignen. Ist die Sperre bereits im Besitz eines anderen \glsuseri{Rechenprozess}, muss der \gls{Rechenprozess} warten, bis seine Aneignung der Sperre erfolgreich ist. Dadurch führen solche Synchronisierungsmethoden laut \textcite{Michael1996} möglicherweise dazu, dass \glspl{Rechenprozess} durch andere \glspl{Rechenprozess} beliebig lang beim Zugriff auf geteilte Daten aufgehalten werden können.

Nicht-blockierende Algorithmen dagegen garantieren, dass mindestens einer der beteiligten \glspl{Rechenprozess} mit einer endlichen Anzahl an Schritten endet. Zudem konnte \textcite{Herlihy1991} bereits 1991 zeigen, dass beispielsweise unter Verwendung von atomaren \emph{Compare-and-Swap}-Instruktionen Datenstrukturen implementiert werden können, auf die nicht-blockierend von einer unbegrenzten Zahl an nebenläufigen \glspl{Aktivitaet} zugegriffen werden kann. Durch die Atomarität der genutzten \glspl{Anweisung} wird sichergestellt, dass die geteilten Ressourcen nicht in einen unvorhergesehenen Zustand übergehen. Die eben genannte Instruktion Compare-and-Swap ist eine \gls{Anweisung}, in der atomar der Wert einer Variablen ausgelesen und mit einem übergebenen Wert verglichen wird (Compare). Danach wird, weiterhin atomar, der Inhalt der Variablen mit einem zweiten übergebenen Wert genau dann überschrieben (Swap), wenn die verglichenen Werte identisch waren. Die Instruktion gibt dann den vorherigen Wert zurück. Eine andere Variante der Instruktion (\emph{Compare-and-Set}) gibt zurück, ob der Wert überschrieben wurde oder nicht. Die Atomarität dieser gesamten Instruktion muss durch die Hardware sichergestellt werden, das ist softwareseitig nicht möglich. Alle gängigen \acp{cpu} stellen Instruktionen dieser Art zur Verfügung.

\section{Multithreading in Java}
In Java können Threads über ein einheitliches Interface verwaltet werden. Zur Entkopplung vom Betriebssystem übernimmt die Java Runtime die Aufrufe der betriebssystemspezifischen Funktionen zur Erzeugung und Verwaltung eines Threads~\cite[S.~3]{Friesen2015}. Das einheitliche Interface wird durch die Klasse \class{Thread} repräsentiert, die einen Java Thread von einem Thread des  Betriebssystems entkoppelt. 

Um die Ausführung eines Java Threads zu starten, wird dessen Methode \code{start(): void} aufgerufen~\cite[S.~8]{Friesen2015}. Ohne eine Angabe der auszuführenden Anweisungen ist allerdings noch nicht definiert, welche Aktivitäten der Thread bei der Ausführung durchführen soll. Java enthält das Interface \class{Runnable}, das eine einzige Methode \code{run(): void} definiert, die weder Parameter noch Rückgabewert besitzt. Die Klasse \class{Thread} besitzt Konstruktoren die als Argument ein Objekt erwarten, das \class{Runnable} implementiert, wird der Java Thread gestartet, so wird in der Ausführung die \code{run()} Methode des übergebenen Objekts aufgerufen~\cite[S.~3]{Friesen2015}.

Alternativ kann eine Klasse von \class{Thread} erben, da \class{Thread} selbst \class{Runnable} implementiert, und die Definition von \code{run()} überschreiben~\cite[S.~335]{Rauber2006}. Da die Klasse von \class{Thread} erbt, besitzen Objekte der Klasse ebenfalls die Methode \code{start()}. Wird diese aufgerufen, so wird wie bei \class{Thread} ein Betriebssystem Thread gestartet, der nun die überschriebene Definition von \code{run()} ausführt. Das Erben von von \class{Thread} ist im Normalfall nicht empfohlen, da dieses Design einige Probleme verusacht. Beispielsweise kann die Thread-Unterklasse von keiner anderen Klasse mehr erben~\cite[S.~335]{Rauber2006}, zudem sind die Definition der Anweisungen mit der Definition der Ausführung höchstmöglich gekoppelt, was dem grundlegenden Prinzip der \emph{losen Kopplung} entgegensteht. Entscheidet man sich also für dieses Design, sollte dies eine gute Begründung haben.

Um die Ausführung mehrerer Java Threads zu koordinieren, bietet Java einige Synchronisationsmechanismen. Diese werden nun vorgestellt

\subsection{Synchonisierung in Java}

\subsection{Executors}\label{sec:executor}
\section{Multithreading in Spielen}
Computerspiele nutzen Multithreading seit dem Aufkommen von Mehrkern-Prozessoren, um eine bessere Performance zu erreichen~\cite{Davies2006}. Es gibt verschiedene für Spiele geeignete Multithreading-Architekturen, die jeweils eigene Vor- und Nachteile mit sich bringen. Im Folgenden werden zwei weit verbreitete Multithreading Architekturen betrachtet.

\subsubsection{\glsentrylong{sot}}
\begin{figure}
	\centering
	\begin{tikzpicture}[scale=1.1]
			\fill[lightgray]  (0,0) rectangle (11,1);
			\fill[lightgray] (0,-1.5) rectangle (11,-0.5);
			\fill[lightgray]  (0,1.5) rectangle (11,2.5);
			
			\node at (-1,2) {Thread 1};
			\node at (-1,0.5) {Thread 2};
			\node at (-1,-1) {Thread 3};
			
			
			\fill [orange] (0.5,2.5) rectangle (4.5,1.5);
			\fill [orange] (5,2.5) rectangle (9,1.5);
			\fill [magenta] (0.5,1) rectangle (3.5,0);
			\fill [magenta] (5,1) rectangle (8,0);
			
			
			\node at (2.5,2) {Simulation $n$};
			\node at (7,2) {Simulation $n+1$};
			\node at (2.5,3.5) {Frame $n$};
			\node at (7,3.5) {Frame $n+1$};
			
			\draw  (0.5,3) rectangle (4.5,-2);
			\draw  (5,3) rectangle (9,-2);
			
			\node at (2,0.5) {Render $n-1$};
			\node at (6.5,0.5) {Render $n$};
			\node at (4,0.5) {?};
			\node at (8.5,0.5) {?};
			\node at (2.5,-1) {?};
			\node at (7,-1) {?};
	\end{tikzpicture}
	\caption{Systeme laufen auf eigenen Threads. Es wird die Auslastung während zwei Frames $n$ und $n+$ gezeigt. Einige Threads sind dadurch nicht zu \SI{100}{\percent} ausgelastet, was durch Fragezeichen (?) gekennzeichnet ist. In diesem Fall kann Thread 3 gar nicht genutzt werden. Thread 2 übernimmt das Rendering, durch Pipelining wird dabei immer der Zustand der Simulation des letzten Frames dargestellt. Die Abbildung ist von einer Grafik aus \cite[S.~14]{Tatarchuk2014} inspiriert.}\label{fig:sot}
\end{figure}
Eine einfache Architektur vergibt, wie in Abbildung~\vref{fig:sot} gezeigt, für einzelne Systeme (wie Simulation oder Rendering) eigene Threads, die diese exklusiv nutzen~\cite{Davies2006,Tatarchuk2014,Genova2015,Hodgman2016}. \textcite{Tatarchuk2014} bezeichnet diese Architektur als \ac{sot}. Durch Pipelining kann das Rendering System dann beispielsweise auf den Spielzustand des vorherigen Frames für die Darstellung zugreifen. Diese Architektur bietet einige Vor- und Nachteile, die nun erörtert werden.
\begin{itemize}
	\item[$+$] Die Architektur bietet den Vorteil, dass die Ausführung innerhalb der Systeme seriell ist, was die Entwicklung wie in Abschnitt~\ref{sec:nebenl-folgen} beschrieben einfacher macht als nebenläufige Entwicklung.
	\item[$+$] Solange die Systeme voneinander separiert sind, gibt es keine Probleme die mit Nebenläufigkeit einhergehen, da beispielsweise Wettkampfbedingungen nur auftreten können, wenn nebenläufig auf geteilte Daten zugegriffen wird. Da keine Wettkampfbedingungen verhindert werden müssen, entfällt blockierende Synchronisierung und Deadlocks sind ausgeschlossen.
	\item[$-$] Auf heterogenen Plattformen, die beispielsweise unterschiedliche Prozessoreinheiten besitzen, kann es zu Performance Problemen kommen. Die Architektur muss eventuell für jede Plattform anders aufgebaut sein. Es gibt also eine gewisse Abhängigkeit von der Hardware.
	\item[$-$] Synchronisierung der Systeme erfordert einen großen Speicheraufwand, da alle Daten (beispielsweise mit einem Double Buffer) zwischengespeichert werden müssen.
\end{itemize}

Die Nutzung von Threads pro System bietet sich an, wenn nur für eine Plattform, beispielsweise PC, entwickelt wird und es keine starken Speicherplatzrestriktionen gibt. Zudem ist die Implementierung einfacher als die der folgenden Architektur.

\subsubsection{Jobsystem}\label{sec:gamesJobsystem}

\begin{figure}
	\centering
	\begin{tikzpicture}[scale=1.1]
		\fill[lightgray]  (0,0) rectangle (11,1);
		\fill[lightgray] (0,-1.5) rectangle (11,-0.5);
		\fill[lightgray]  (0,1.5) rectangle (11,2.5);
		
		\node at (-1,2) {Thread 1};
		\node at (-1,0.5) {Thread 2};
		\node at (-1,-1) {Thread 3};
	
		\foreach \i in {0,3.5,7}{
		\fill [orange,draw=lightgray] ($(\i,0) + (0.5, 1.5)$) rectangle ($(\i,0) +(1.5, 2.5)$);
		\fill [orange,draw=lightgray] ($(\i,0) + (0.5,-1.5)$) rectangle ($(\i,0) +(1.5,-0.5)$);
		\fill [orange,draw=lightgray] ($(\i,0) + (1.5, 1.5)$) rectangle ($(\i,0) +(2.5, 2.5)$);
		\fill [orange,draw=lightgray] ($(\i,0) + (0.5, 0.0)$) rectangle ($(\i,0) +(1.5, 1.0)$);
		
		\fill [magenta,draw=lightgray] ($(\i,0) + (2.5, 1.5)$) rectangle ($(\i,0) + (3.5, 2.5)$);
		\fill [magenta,draw=lightgray] ($(\i,0) + (2.5, 0.0)$) rectangle ($(\i,0) + (3.5, 1.0)$);
		\fill [magenta,draw=lightgray] ($(\i,0) + (2.5,-1.5)$) rectangle ($(\i,0) + (3.5,-0.5)$);
		
		\draw  ($(\i,0) + (0.5,3)$) rectangle ($(\i,0) + (3.5,-2)$);
		
		\node[font=\footnotesize] at ($(\i,0) + (1,2)$) {Sim 1};
		\node[font=\footnotesize] at ($(\i,0) + (1,0.5)$) {Sim 2};
		\node[font=\footnotesize] at ($(\i,0) + (1,-1)$) {Sim 3};
		\node[font=\footnotesize] at ($(\i,0) + (2,2)$) {Sim 4};
		\node[font=\footnotesize] at ($(\i,0) + (3,2)$) {Ren 1};
		\node[font=\footnotesize] at ($(\i,0) + (3,0.5)$) {Ren 2};
		\node[font=\footnotesize] at ($(\i,0) + (3,-1)$) {Ren 3};
	
		}
	\node at (2,3.5) {Frame $n$};
	\node at (5.5,3.5) {Frame $n+1$};
	\node at (9,3.5) {Frame $n+2$};
	\end{tikzpicture}
	\caption{Die Systeme definieren Jobs, die frei auf die vorhandenen Threads aufgeteilt werden. Sim steht für Simulation und Ren für Rendering. Die Abbildung ist schematisch, die Jobs müssen nicht exakt die gleiche Länge haben, auch wenn kurze Jobs mit ähnlichen Laufzeiten ideal sind. Im Vergleich zu Abbildung~\vref{fig:sot} lässt sich eine höhere Auslastung der Threads erkennen, was wiederum höhere FPS zur Folge hat.}\label{fig:jobt}
\end{figure}
Eine etwas komplexere Architektur liefert die Implementierung eines sogenannten \emph{Job-} oder \emph{Task-Systems}~\cite{Davies2006,Tatarchuk2014,Genova2015,Hodgman2016}. Ein Schema der Architektur ist in Abbildung~\vref{fig:jobt} zu sehen. Anstatt Aufgaben fest auf einzelne Threads zu verteilen, gibt es \emph{Jobs} und ein System, das die Jobs auf Threads verteilt. Konzeptuell entspricht die Architektur sehr dem in Abschnitt~\ref{sec:executor} beschriebenen \class{Executor} in Java. 

Die idealerweise kurzen Jobs werden an das System übergeben. Das System verwaltet die Threads, deren Anzahl idealerweise der Anzahl der Hardwarethreads entspricht, da dann die Leistung des Prozessors optimal ausgenutzt werden kann und die Anzahl der Kontextwechsel minimiert wird. Ist ein Thread im Leerlauf, wird ein Job an den Thread zur Ausführung übergeben.

\begin{figure}
	\centering
	\begin{tikzpicture}
		\node[fill=orange] (sim1) {Sim 1};
		\node[fill=orange,below=of sim1] (sim2)  {Sim 2};
		\node[fill=orange,below=of sim2] (sim3)  {Sim 3};

		\node[fill=orange,right=of sim2] (sim4)  {Sim 4};

		\node[fill=magenta,right=of sim4] (ren2)  {Ren 2};
		\node[fill=magenta,above=of ren2] (ren1)  {Ren 1};
		\node[fill=magenta,below=of ren2] (ren3)  {Ren 3};

		
		\draw[->](sim1.east) -- (sim4.north west);
		\draw[->](sim2.east) -- (sim4.west);
		\draw[->](sim3.east) -- (sim4.south west);
		\draw[->](sim4.north east) -- (ren1.west);
		\draw[->](sim4.east) -- (ren2.west);
		\draw[->](sim4.south east) -- (ren3.west);
	\end{tikzpicture}
	\caption{Beispiel eines Job Graphen, der der Ausführung in Abbildung~\vref{fig:jobt} zugrunde liegen könnte. Die farbigen Knoten stellen Jobs dar, wobei orange Knoten Simulationsjobs und magentafarbene Knoten Renderingjobs sind. Die gerichteten Kanten zeigen die mögliche Ausführungsreihen folge an. So muss beispielsweise Sim 4 nach Sim 3 ausgeführt werden. Gibt es keinen Pfad von Job $A$ zu Job $B$ so können beide parallel ausgeführt werden.}\label{fig:jobdependencies}
\end{figure}
Das System kann zudem so implementiert werden, dass es Abhängigkeiten zwischen einzelnen Jobs gibt, das heißt, dass ein Job beispielsweise erst ausgeführt wird, wenn ein anderer Job abgeschlossen ist. Daraus bildet sich ein Job Graph, der die nebenläufige Ausführung der Jobs beschreibt. Es hat sich als hilfreich erwiesen, feste Synchronisationspunkte zu definieren, damit auf systemübergreifende Daten zugegriffen werden kann. Das kann im Jobsystem wie in Abbildung~\vref{fig:jobdependencies} durch Jobs repräsentiert werden, die Abhängigkeiten zu vielen vorangehenden Jobs sammeln und dann als Abhängigkeit für die folgenden Jobs dienen. Betrachten wir nun einige Vor- und Nachteile eines Jobsystems.
\begin{itemize}
	\item[$+$]  Da die Durchführung von Jobs nicht an bestimmte Threads gekoppelt ist, kann Hardware mit mehr Prozessoreinheiten sofort voll ausgenutzt werden. Wenn die Anzahl der Prozessoren zur Laufzeit ermittelt wird, funktioniert das sogar dynamisch. Ist eine Prozessoreinheiten vorhanden, wird diese genutzt, gibt es $100$ Prozessorkerne, werden $100$ genutzt. Durch ein Jobsystem kann also jede beliebige in Hardware vorhandene Parallelität ausgenutzt werden. Dadurch werden ohne zusätzlichen Programmieraufwand verschiedenste Plattformen unterstützt.
	\item[$+$] Ein Jobsystem bietet die Möglichkeit, Simulation und Rendering in einem Frame sequentiell durchzuführen, ohne alle Nebenläufigkeit zu verlieren. Durch das Jobsystem können die Systeme intern parallelisiert werden. Werden Simulation und Rendering sequentialisiert, ist es nicht nötig des Spielzustand zwischenzuspeichern, da die Simulation während des Renderings bereits abgeschlossen ist und somit Wettkampfbedingungen ausgeschlossen sind (da das Rendering nur lesend auf Daten zugreift). Damit benötigt man weniger Speicher als bei einer Architektur, die auf einen Zwischenspeicher angewiesen ist. Spiele, die eh schon viel Speicher verbrauchen und auf speicherbeschränkten Systemen ausgeführt werden sollen, können auf diese Art von Nebenläufigkeit guten Gebrauch machen. 
	
	Wie in Abbildung~\vref{fig:jobt} zu sehen, führt dies aber auch dazu, dass die Threads gegebenenfalls nicht vollständig ausgelastet sind, da beispielsweise alle noch auszuführenden Jobs auf die Beendigung eines noch nicht abgeschlossenen Jobs warten müssen.
	\item[$+$] Auch mit einem Jobsystem gibt es weiterhin die Möglichkeit mittels eines Zwischenspeichers Pipelining zu betreiben und so die in Abbildung~\vref{fig:jobt} sichtbaren Auslastungslücken zu schließen.
	\item[$+$] Wird ein neues System hinzugefügt, beispielsweise ein System zur Verwaltung der Audioausgabe, muss die grundlegende Architektur nicht angepasst werden, um das neue System nebenläufig in das Spiel zu integrieren. Das System kann einfach Jobs erzeugen und diese zur Ausführung bringen.
	\item[$-$] Das Jobsystem basiert auf Jobs. Diese müssen erstellt werden. Soll also ein bestehendes Spiel ein Jobsystem nutzen, muss der gesamte Code so umgeschrieben werden, das bestehende Abläufe in kleine Jobs aufgeteilt werden und dann mit den korrekten Abhängigkeiten nebenläufig ausgeführt werden. Es muss also tief in die bestehenden Systeme eingegriffen werden. Dieses Problem ergibt sich bei der \ac{sot} Architektur nicht, da hier die Systeme intern weiterhin sequentiell sind.
	\item[$-$] Typischerweise haben die Nutzer des Jobsystems keine Möglichkeit zu entscheiden, auf welchem Thread ein Job ausgeführt werden soll. Das ist zum einen so gewollt und zum anderen für gewöhnlich kein Problem. Insbesondere beim Rendering mittels der Grafikschnittstelle OpenGL führt das allerdings zu Problemen, da die OpenGL API nur von einem speziellen Thread aus aufgerufen werden darf. Unter der Nutzung von OpenGL ist es also nicht möglich, das Rendering nebenläufig durchzuführen.
\end{itemize}

Entwickler von Computerspielen haben zuerst \ac{sot} genutzt, da diese Architektur einfach ist~\cite{Genova2015,Tatarchuk2014} und Computer zu der Zeit ihrer Nutzung noch wenige Kerne hatten. Vor über 15 Jahren hat die Jobsystem Architektur begonnen an Gewicht gewinnen~\cite{Davies2006} und Entwickler nutzen seither vermehrt diese Architektur in ihren neuen Spielen~\cite{Tatarchuk2014,Genova2015,Gyrling2015,Hodgman2016}.

Diese Entscheidung liegt vermutlich darin begründet, dass die Vorteile von \ac{sot}  und die Nachteile des Jobsystems für moderne Spiele weniger relevant sind. Entwickler sind inzwischen mit dem Jobsystem Konzept vertraut und dieses kann ebenfalls gut vor Wettkampfbedingungen schützen, wenn die Abhängigkeiten zwischen Jobs korrekt sind. Bei der Entwicklung neuer Spiele, kann die Nutzung des Jobsystems sofort berücksichtigt werden. Dadurch sind später keine tiefgreifenden Änderungen nötig, da die Integration von Anfang an stattgefunden hat. Spielentwickler nutzen statt OpenGL meist DirectX, das seit 2009 mit Version 11 Multithreading unterstützt~\cite{White2018}. Auch Vulkan, das inzwischen vermehrt in der Spielentwicklung Einzug erhält, unterstützt Multithreading~\cite{Schott2016}. Daher können auch die Aufrufe des Rendering über das Jobsystem nebenläufig geschehen.

\begin{figure}
	\includegraphics[width=\textwidth]{Destiny.png}
	 \caption{Darstellung eines Job Graphen für das Spiel Destiny. Die farbigen Umrahmungen zeigen die verschiedenen Stufen, in die der Graph durch Synchronisationspunkte unterteilt ist. Die Abbildung ist eine Aggregation von vier Abbildungen aus \cite[S.~39~ff.]{Tatarchuk2014}.}\label{fig:destiny-jobgraph}
\end{figure}


Das Spiel Destiny, das 2014 erschienen ist, besitzt beispielsweise eine Engine mit einem Jobsystem in dem das Rendering sequentiell nach der Simulation erfolgt. Abbildung~\vref{fig:destiny-jobgraph} zeigt den Job Graph des Spiels, wobei veranschaulicht wird, dass der Graph verschiedene Phasen enthält (\enquote{Game simulation}, \enquote{Extract game state}, \enquote{Prepare GPU-friendly data}, \enquote{Submit to GPU}), die durch synchronisierende Jobs sequentialisiert sind. Dadurch muss der Spielzustand nicht zwischengespeichert werden. Wie in der Abbildung zu erkennen ist, gibt es in den einzelnen Phasen große Mengen an Jobs, die nebenläufig ausgeführt werden können, sodass das Spiel von einer großen Menge an Prozessoreinheiten profitieren kann.

Im folgenden Kapitel wird nun der Zustand der Blocklib analysiert, um die bereits vorhandene Nutzung von Multithreading zu verstehen und eine Grundlage für das Design der neuen nebenläufigen Architektur und die Anforderungen an diese zu bilden.


\chapter{Entwicklung einer nebenläufigen Architektur}

\section{Analyse der Blocklib}\label{sec:blocklibAnalyse}
Die Entwicklung einer nebenläufigen Architektur führt zu
 tiefgreifenden Änderungen in einer Codebasis, unabhängig davon, ob die Architektur auf \ac{sot} oder einem Jobsystem basiert. Insofern ist ein Verständnis für die bestehende Architektur der Codebasis nötig. Interessant ist dabei insbesondere wie die Game-Loop, beschrieben in Abschnitt~\ref{sec:gameLoop}, gestaltet ist und an welchen Stellen bereits von Multithreading Gebrauch gemacht wird, was in Abschnitt~\ref{sec:nutzungMultithreading} erläutert wird.

\subsubsection{Game-Loop}\label{sec:gameLoop}
Ein nahezu universelles Designelement von Spielen ist die sogenannte \emph{Game-Loop}~\cite[S.~161~\psqq]{Nystrom2015}. Sie ist für die Aktualisierung des Spiels zuständig und umfasst typischerweise die Aufgaben 
\begin{itemize}
  \item Verarbeitung der Eingaben,
  \item Berechnung des neuen Spielzustands und
  \item (grafische) Ausgabe.
\end{itemize}
Die Blocklib besitzt ebenfalls eine Game-Loop und führt die beschriebenen Aufgaben in der obigen Reihenfolge sequentiell aus. Genauer sind es zwei Game-Loops, eine für die Ausführung als Server (ohne graphische Ausgabe) und eine für einen Client oder Einzelspieler.
Listing~\ref{lst:oldGameLoop} zeigt eine vereinfachte Darstellung der Game-Loop der Einzelspieler-Blocklib.
Die typischen Elemente der Game-Loop sind vorhanden.

\begin{lstlisting}[caption={Vereinfachte Version der Blocklib für Einzelspieler.},label={lst:oldGameLoop},float={htbp}]
while(!shutdown){
	//...
	processInput(delta); // Verarbeitung der Eingaben
	//...
	update(delta); // Berechnung des neuen Spielzustands
	//...
	render(); // grafische Ausgabe
	//...
	window.update(); // Verarbeitung der Eingaben und grafische Ausgabe
}
\end{lstlisting}

Die Methode \code{update(...)} ruft ihrerseits die \code{update(...)}-Methoden der einzelnen Simulationssysteme auf (siehe Listing~\ref{lst:gameUpdate}).
Bei der Berechnung des neuen Spielzustands gibt es in der Blocklib kein zentrales System, das den Zustand des letzten Loopdurchlaufs vorhält. Somit ist es möglich, dass das Verhalten der Blocklib von der Reihenfolge der \code{update(...)}-Methoden abhängt, da der Zustand, auf den bei der Berechnung zugegriffen wird, zeitgleich verändert wird. Das kann anhand eines Beispiels veranschaulicht werden. 

\begin{lstlisting}[caption={Vereinfachte Update-Methode von \classGame{}.}, label={lst:gameUpdate},float={htbp}]
private void update(float delta) {
	//...
	Context con = Context.getInstance();
	//...
	con.getChunkManager().update(delta);
	con.getEffectManager().update(delta);
	//...
	con.getAudioManager().update();
	con.getMainScheduler().update(delta);
	con.getEntityManager().update(delta);
	con.getFluidManager().update(delta);
	//...
}
\end{lstlisting}

\begin{example}
Man nehme an, dass der \classEntityManager{} (Eine Klasse, die unter anderem alle Charaktere wie Gegner oder den Spieler verwaltet) prüft, welche Chunks geladen sind, um zu ermitteln, wo ein neuer Gegner erschaffen werden soll.
Der \classChunkManager{} bestimmt, welche Chunks abhängig von der aktuellen Kameraposition geladen werden.
Wird nun ein Chunk entfernt, auf dem der \classEntityManager{} einen Gegner erschaffen hat, könnte das zu unvorhergesehenem Verhalten führen, da der \classEntityManager{} erwartet, dass der erschaffene Gegner auf einem existenten Chunk erstellt wurde.
\end{example}

Die (mögliche) Abhängigkeit des Verhaltens von der Reihenfolge der Updates, zeigt, dass sich diese Berechnungen nicht problemlos parallelisieren lassen, da man den gesamten Simulationscode auf mögliche Wettkampfbedingungen prüfen müsste.


\subsubsection{Nutzung von Multithreading}\label{sec:nutzungMultithreading}
Bereits vor den Änderungen in dieser Arbeit werden in der Blocklib Threads genutzt, um nebenläufige \glspl{Anweisung} zu parallelisieren. Darunter fallen insbesondere zum einen Hintergrundaufgaben, wie das Erstellen von Chunks, und zum anderen das Ausführen zeitlich gesteuerter Abläufe, wie die Updates von Flüssigkeiten. Insgesamt gibt es zu Beginn der Arbeit sieben Klassen, die Threads erzeugen und nutzen. Eine Auflistung der Klassen und kurzer Beschreibungen, wie diese Nebenläufigkeit nutzen, ist in Tabelle~\ref{tab:concTasksBlocklib} zu finden.

\begin{table}
	%fixup, because hyphenation of lstinline breaks width measurement
	\settowidth{\mytemp}{\texttt{NioClientNetworkLayer}}
	\renewcommand{\arraystretch}{1.5}
	\begin{tblr}{colspec={p{\mytemp}X}}
		\toprule
		Klasse & Bestehende Nutzung von Nebenläufigkeit \\
		\midrule
		\classChunkStorage{} & Executors werden genutzt, um die Generierung von Chunks und das Meshing von Chunks auszulagern.\\
		\classEventManager{} & Eventhandling wird in Threads ausgelagert, um Deadlocks zu verhindern, wenn die benachrichtigten Beobachter versuchen, auf dieselben kritischen Bereiche zuzugreifen. \\
		\classConnectionInfo{} & Im Server wird ein wiederkehrender Task gestartet, um Ping Nachrichten an die Clients zu senden.\\
		\classNioClientNetworkLayer{} & Ein Thread, der im Hintergrund mit dem Server kommuniziert, wird gestartet.\\
		\classNioServerNetworkLayer{} & Ein Thread, der im Hintergrund mit den Clients kommuniziert, wird gestartet.\\
		\classFluid{} & Ein Thread, der den Zustand der Flüssigkeit berechnet, wird gestartet.\\
		\classAudioManager{} & Die Klasse \classTimer{} wird verwendet, Hintergrundmusik abzuspielen.\\
		\bottomrule 
	\end{tblr}
	\caption[Klassen mit nebenläufigen \glsentryplural{Anweisung} in der Blocklib.]{Klassen mit nebenläufigen \glspl{Anweisung} in der Blocklib.}\label{tab:concTasksBlocklib}
\end{table}

Es gibt keine zentrale Stelle, an der die Erstellung von Threads verwaltet wird. Jede Klasse erstellt und beendet Threads eigenständig. Die Anzahl der genutzten Threads lässt sich somit nicht überblicken. Aus diesem Grund ist es schwierig sicherzustellen, dass die Anzahl der genutzten Threads die Anzahl der Prozessoren nicht überschreitet. Das führt zu unnötigen Kontextwechseln und verschlechtert somit die Leistung. Außerdem können die bestehenden Threads möglicherweise nicht voll ausgelastet werden, da nur einzelne Klassen auf diese Zugriff haben.

\section{Anforderungen der Blocklib an eine Threading API}\label{sec:anforderungen}
Aus der in Abschnitt~\ref{sec:blocklibAnalyse} beschriebenen Analyse lassen sich einige Anforderungen ableiten, die von der nebenläufigen Architektur erfüllt werden müssen. Neben den aus der Analyse erwachsenen Anforderungen, lassen sich weitere Anforderungen definieren, die für die Blocklib in Zukunft nützlich sind.

Die Schnittstelle der nebenläufigen Architektur, im Folgenden auch \emph{API} genannt, muss sich in das bestehende Ökosystem der Codebasis der Blocklib integrieren, sodass deren Nutzung möglichst einfach von der Hand geht und angenommen wird.

\subsubsection{Kontrolle über Threads an einer Stelle}
Um die Nutzung der API einfach zu gestalten, ist es notwendig, die Verwaltung und Kontrolle über die Java-Threads, die nebenläufige \glspl{Anweisung} ausführen, hinter der API zu verstecken. Der Lebenszyklus eines Threads ist für die asynchrone Abarbeitung von Tasks irrelevant, kann allerdings gegebenenfalls komplex sein, wenn beispielsweise dynamisch Threads erzeugt oder beendet werden oder ermittelt wird, wann ein Thread welche Tasks abarbeitet. Diese Komplexität darf nicht zu den Nutzern der API vordringen.

Zudem ist es sinnvoll, diese Verwaltung an einer Stelle zu bündeln, um den Überblick über die Anzahl der Java-Threads nicht zu verlieren, damit durch die Auswahl einer geeigneten Zahl Kontextwechsel möglichst vermieden werden können. Diese Übersicht und Kontrolle lässt sich am besten an einer Stelle gebündelt verwirklichen.

Des Weiteren wird so Redundanz in der Codebasis verringert, da der Code zur Verwaltung der Threads nur an einer Stelle benötigt wird. Gäbe es hier mehrere Orte, würde sich beispielsweise eventuell der Code, der für die Generierung eines Java-Threads zuständig ist, an mehreren Stellen wiederholen. Dies ginge dann mit den üblichen Problemen von Redundanz einher. Mit der Zeit unterscheiden sich die verschiedenen Codeteile, da vergessen wird Änderungen an allen Orten durchzuführen und Änderungen sind aufwendiger, da sie an allen Stellen durchgeführt werden müssen.

\subsubsection{Verteilung von Berechnungen an Worker-Threads}
Wie in Abschnitt~\ref{sec:nutzungMultithreading} beschrieben, werden für die Generierung von Chunks Threads genutzt. Die Blocklib benötigt also eine Möglichkeit \glspl{Anweisung} asynchron an einen Executor zu übergeben. Der Executor kann in Zukunft auch für Übergabe anderer asynchroner Aufgaben genutzt werden. So kann die Blocklib die vorhandene Parallelität in der Hardware voll ausnutzen. Das Ziel ist, durch Worker-Threads immer möglichst viele Tasks parallel ausführen zu können, wenn genügend Tasks vorhanden sind.

\subsubsection{Einfaches starten von Hintergrundtasks und einmaligen Aufgaben}\label{sec:reqBackgroundTasks}
Einige Systeme der Blocklib, wie beispielsweise das System zu Berechnung von Flüssigkeitsbewegungen, führen Aufgaben durch, die periodisch im Hintergrund ablaufen. So wird Im Flüssigkeitssystem für jede Flüssigkeit nach einem eigenen Zeitintervall geprüft, wohin die Flüssigkeit fließt. Es muss eine einfache Möglichkeit geben solche periodischen Aufgaben an die Threading API zu übergeben, die sich dann um die Ausführung zur korrekten Zeit kümmert.

Damit verwandt ist, eine Aufgabe zu definieren, die nach dem Ablauf eines festgelegten Intervalls einmalig ausgeführt wird. Man überlege sich beispielsweise eine Art Falle, die durch das Betreten eines bestimmten Blocks in der Blocklib ausgelöst wird und dann nach ein paar Sekunden zuschnappt. Ein solches System gibt es in der Blocklib zwar noch nicht, ließe sich dann durch die API allerdings einfach realisieren.

Die Klasse ChunkStorage erzeugt für jeden zu ladenden Chunk einen Task für die Generierung und einen Task, um das zugehörige Mesh zu erzeugen. In vielen Situationen, wie dem Start der Blocklib, Teleportation der Kamera durch einen Konsolen-Befehl oder schnelle Bewegung der Kamera, wird dadurch eine sehr große Anzahl dieser Tasks erzeugt. Beim Start der Blocklib werden bei einer geladenen Anzahl von $15$ Chunks in  $x$-, $y$- und $z$- Richtung bis zu $2\cdot15^3=6750$  Tasks auf einmal erzeugt. Für Abarbeitung dieser Tasks benötigen die meisten Rechner mehr Zeit, als in einem Frame zur Verfügung steht (möchte man eine Bildwiederholgeschwindigkeit von mindestens 60~\ac{fps} erreichen, stehen pro Frame maximal \SI{16,67}{\milli\second} zur Verfügung). Daher müssen diese Hintergrundtasks anders behandelt werden, sodass deren Ausführung nicht die Durchführung von Tasks behindert, die für die Berechnung eines Frames notwendigerweise durchgeführt werden müssen. Die Besonderheit der Blocklib, dass sich die Anzahl der Tasks  dynamisch von Frame zu Frame stark unterscheiden kann, muss also berücksichtigt werden.

\section{Design der Threading API}
Bei der Nutzung von Multithreading in Spielen gibt es zwei prominente Ansätze, zum einen die Nutzung eines separaten Threads, der das Rendern übernimmt, und zum anderen den Einsatz einer Jobarchitektur.

Die Idee der Nutzung eines Renderthreads ergibt sich daraus, dass Simulation des Spiels und Anzeige zwei unabhängige Bereiche sind, die also nebenläufig ausgeführt werden können. Da Rendering sehr rechenintensiv ist, ist es also sinnvoll diese Aktivitäten in einen eigenen Thread auszulagern, damit sie den gesamten Frame für die Ausführung nutzen können.

Der Einsatz eines separaten Renderthreads ist dabei auch in der Jobarchitektur vorgesehen.

Da es sich bei der Blocklib, mit knapp $60000$ Zeilen Code und über $900$ Java Dateien, um ein großes bereits bestehendes System handelt, ist es unrealistisch, in kurzer Zeit das gesamte Programm so umzustrukturieren, dass es vollständig eine Jobarchitektur nutzt. Diese Umstrukturierung erfordert tiefes Verständnis für jedes zu ändernde System. Zudem müssen viele Bereiche grundlegend geändert werden, um die Abhängigkeiten der Systeme zu verringern beziehungsweise auszuschließen.

Die Blocklib enthält beispielsweise eine Klasse \class{Context}, die als eine Art \gls{singleton}-Implementierung eines \glspl{servicelocator}~\cite[S.~301~ff.]{Nystrom2015} dient. Da \glspl{singleton} ähnlich wie globale Variablen von überall aufgerufen werden können, wird der Code dadurch potenziell schwieriger nachvollziehbar~\cite[S.~108]{Nystrom2015}. Dadurch und durch die ebenfalls resultierende Kopplung unterschiedlicher Komponenten würden bei einer naiven Umsetzung der Jobarchitektur unvorhersehbar Wettkampfbedingungen auftreten.
Man erinnere sich, dass Wettkampfbedingungen auftreten, wenn nebenläufige Aktivitäten auf dieselben Ressourcen zugreifen. Durch den \class{Context} ist es nun schwierig, einen Überblick zu haben von wo aus auf welche Ressourcen zugegriffen wird. Die Wahrscheinlichkeit, dass also Zugriffe existieren, die zu Wettkampfbedingungen führen, ist also sehr hoch, solange dabei nicht sorgsam vorgegangen wird.

\subsection{Renderthread und Job-API}
Da es sich bei der Blocklib, mit knapp $60000$ Zeilen Code und über $900$ Java Dateien, um ein großes bereits bestehendes System handelt, ist es unrealistisch, in kurzer Zeit das gesamte Programm so umzustrukturieren, dass es vollständig eine Jobarchitektur nutzt. Diese Umstrukturierung erfordert tiefes Verständnis für jedes zu ändernde System. Zudem müssen viele Bereiche grundlegend geändert werden, um die Abhängigkeiten der Systeme zu verringern beziehungsweise auszuschließen.

Die Blocklib enthält beispielsweise eine Klasse \class{Context}, die als eine Art Singletonimplementierung eines Service Locators~\cite[S.~301~ff.]{Nystrom2015} dient. Da Singletons ähnlich wie globale Variablen von überall aufgerufen werden können, wird der Code dadurch potenziell schwieriger nachvollziehbar~\cite[S.~108]{Nystrom2015}. Dadurch und durch die ebenfalls resultierende Kopplung unterschiedlicher Komponenten würden bei einer naiven Umsetzung der Jobarchitektur unvorhersehbar Wettkampfbedingungen auftreten.
Man erinnere sich, dass Wettkampfbedingungen auftreten wenn nebenläufige Aktivitäten auf die selben Ressourcen zugreifen. Durch den \class{Context} ist es nun schwierig, einen Überblick zu haben von wo aus auf welche Ressourcen zugegriffen wird. Die Wahrscheinlichkeit, dass also Zugriffe existieren, die zu Wettkampfbedingungen führen, ist also sehr hoch, solange dabei nicht sorgsam vorgegangen wird.

\subsection{Design des Renderthreads}\label{sec:desgignRenderthread}
\begin{figure}
	\centering
	\begin{tikzpicture}[scale=1.1]
		\fill[lightgray]  (0,0) rectangle (11,1);
		\fill[lightgray] (0,-1.5) rectangle (11,-0.5);
		\fill[lightgray]  (0,1.5) rectangle (11,2.5);
		
		\node[anchor=east] at (0,2) {Thread 1};
		\node[anchor=east] at (0,0.5) {Thread 2};
		\node[anchor=east] at (0,-1) {Renderthread};
		
		
		\fill [orange,draw=lightgray] (0.5,1.5) rectangle node[black,font=\footnotesize] {Sim 1} (1.5,2.5);
		\fill [orange,draw=lightgray] (0.5,0) rectangle node[black,font=\footnotesize] {Sim 2} (1.5,1);
		\fill [orange,draw=lightgray] (1.5,1.5) rectangle node[black,font=\footnotesize] {Sim 3} (2.5,2.5);
		\fill [orange,draw=lightgray] (2.5,1.5) rectangle node[black,font=\footnotesize] {Sim 4} (3.5,2.5);
		\fill [orange,draw=lightgray] (1.5,0) rectangle node[black,font=\footnotesize] {Sim 5} (2.5,1);
		\fill [orange,draw=lightgray] (3.5,1.5) rectangle node[black,font=\footnotesize] {Sim 6} (4.5,2.5);
		\fill [orange,draw=lightgray] (2.5,0) rectangle node[black,font=\footnotesize] {Sim 7} (3.5,1);
		\fill [orange,draw=lightgray] (3.5,0) rectangle node[black,font=\footnotesize] {Sim 8} (4.5,1);
		
		\fill [orange,draw=lightgray] ($(4.5,0)+(0.5,1.5)$) rectangle node[black,font=\footnotesize] {Sim 1} ($(4.5,0)+(1.5,2.5)$);
		\fill [orange,draw=lightgray] ($(4.5,0)+(0.5,0)$) rectangle node[black,font=\footnotesize] {Sim 2} ($(4.5,0)+(1.5,1)$);
		\fill [orange,draw=lightgray] ($(4.5,0)+(1.5,1.5)$) rectangle node[black,font=\footnotesize] {Sim 3} ($(4.5,0)+(2.5,2.5)$);
		\fill [orange,draw=lightgray] ($(4.5,0)+(2.5,1.5)$) rectangle node[black,font=\footnotesize] {Sim 5} ($(4.5,0)+(3.5,2.5)$);
		\fill [orange,draw=lightgray] ($(4.5,0)+(1.5,0)$) rectangle node[black,font=\footnotesize] {Sim 4} ($(4.5,0)+(2.5,1)$);
		\fill [orange,draw=lightgray] ($(4.5,0)+(3.5,1.5)$) rectangle node[black,font=\footnotesize] {Sim 6} ($(4.5,0)+(4.5,2.5)$);
		\fill [orange,draw=lightgray] ($(4.5,0)+(2.5,0)$) rectangle node[black,font=\footnotesize] {Sim 7} ($(4.5,0)+(3.5,1)$);
		\fill [orange,draw=lightgray] ($(4.5,0)+(3.5,0)$) rectangle node[black,font=\footnotesize] {Sim 8} ($(4.5,0)+(4.5,1)$);
	
		\fill [magenta] (0.5,-0.5) rectangle node[black]{Render $n-1$} (4.2,-1.5);
		\fill [magenta] (5,-0.5) rectangle node[black]{Render $n$} (8.7,-1.5);
		
		\node at (2.5,3.5) {Frame $n$};
		\node at (7,3.5) {Frame $n+1$};
		
		\draw  (0.5,3) rectangle (4.5,-2);
		\draw  (5,3) rectangle (9,-2);
	\end{tikzpicture}
	\caption{Darstellung des Designs der Multithreading Architektur der Blocklib. Es existiert ein gesonderter Renderthread, der einen großen Teil der Rechenzeit während eines Frames nutzt.}\label{fig:optimalArchitecture}
	\todo{Caption}
\end{figure}

Da der Performancegewinn, in Form von \ac{fps}, bei der Nutzung eines Renderthreads als hoch zu erwarten ist, soll dieser in die Blocklib integriert werden. Um die Anforderungen von Kapitel~\ref{sec:anforderungen} zu Erfüllen, wird ein Jobsystem implementiert. Da die Blocklib OpenGL als Grafik Schnittstelle nutzt und OpenGL, wie in Abschnitt~\ref{sec:gamesJobsystem} beschrieben, Multithreading nicht unterstützt, kann das Rendering selbst nicht nebenläufig durchgeführt werden.

Der Renderthread kann einerseits als Teil des Jobsystems designt werden, andererseits gibt es die Möglichkeit den Renderthread von diesem zu trennen. Ist der Renderthread Teil des Jobsystems, kann dieser voll zur Bearbeitung von Jobs mitgenutzt werden. Da üblicherweise die Anzahl der Threads der Anzahl der Prozessorkerne entspricht, kann man so automatisch immer die Leistung aller Prozessorkerne nutzen. Trennt man den Thread dagegen ab, entsteht die Problematik zu entscheiden, wann welche Anzahl von Threads genutzt wird, um möglichst alle Kerne zu nutzen, aber gleichzeitig zu verhindern, dass sich die Threads in der Ausführung gegenseitig behindern. Andererseits gestaltet sich die Implementierung eines getrennten Renderthreads als deutlich einfacher und intuitiver~\cite{Tatarchuk2014}, im Gegensatz zur Jobsystemintegration.

Nach Messungen ergibt sich, dass das Rendering in etwa \SI{50}{\percent}\todo{messen} der Zeit eines Frames benötigt. Daher bietet es sich an, den Renderthread zu separieren. Der prinzipiell mögliche Performancegewinn durch die Integration in das Jobsystem ist ausgeschlossen, da das Rendering selbst sie gesamte Rechenleistung des Kerns jeden Frame beansprucht. Da das Rendering im sequentialisierten Fall circa \SI{50}{\percent} der Zeit benötigt, entspricht das annähernd \SI{100}{\percent}, sobald Simulation und Rendering in zwei Threads nebenläufig ausgeführt werden. Wird die Simulation selbst nebenläufig durchgeführt verstärkt sich dieser Effekt, wodurch die Framezeit, durch das Rendering bestimmt wird. Damit lässt sich auch die Anzahl der Threads des Jobsystems bestimmen, indem die Anzahl der Jobthreads um eins verringert wird ohne dadurch Prozessorkerne ohne Arbeit zu verursachen.

Da der Renderthread auf Daten des Spiels zugreift, um die sichtbaren Elemente zu zeichnen, muss sichergestellt werden, dass diese Daten keinen Wettkampfbedingungen unterliegen. In der Spieleentwicklung ist es üblich einen \emph{Spielzustand} (engl. Game State) zu definieren, der während der Simulation in jedem Frame angepasst wird. Der Teil des Zustandes auf den der Renderthread zugreift muss also konstant sein. Während der Simulation wird dieser allerdigs verändert. Eine Möglichkeit, dieses Problem zu beheben, besteht darin einen \emph{Double Buffer}~\cite[S.~143]{Nystrom2015} zu nutzen, um den gesamten Spielzustand zwischenzuspeichern~\cite{Tatarchuk2014}. Die Blocklib ist nicht mit diesem Hintergrund entwickelt worden, weswegen es keine einfache Möglichkeit gibt, den gesamten Spielzustand auf diese Weise zwischenzuspeichern. Die Objekte, die den Spielzustand darstellen, sind über die Blocklib hinweg verteilt. Um dennoch einen Renderthread nutzen zu können, müssen die Objekte identifiziert werden, die für das Rendering benötigt werden. Für diese Objekte muss dann ein geeigneter Double Buffer erzeugt werden, sodass die Daten aus Renderthread-Sicht konstant sind.

Somit ist eine nebenläufige Architektur, wie sie in Abbildung~\ref{fig:optimalArchitecture} dargestellt wird, erstrebenswert. Es gibt einen Renderthread, der die von der Simulation im vorherigen Frame berechneten Objekte zeichnet. Alle anderen verfügbaren Hardwarethreads können in dem Jobsystem für die Simulation genutzt werden. 

\subsection{Design des Jobsystems}\label{sec:desgignJobsystem}
\begin{figure}
	\centering
	\begin{tikzpicture}[scale=1.1]
		\fill[lightgray]  (0,0) rectangle (11,1);
		\fill[lightgray] (0,-1.5) rectangle (11,-0.5);
		\fill[lightgray]  (0,1.5) rectangle (11,2.5);
		
		\node[anchor=east] at (0,2) {Thread 1};
		\node[anchor=east] at (0,0.5) {Thread 2};
		\node[anchor=east] at (0,-1) {Renderthread};
		
		
		\fill [orange,draw=lightgray] (0.5,1.5) rectangle node[black] {Simulation $n$} (4.5,2.5);
		\fill [orange,draw=lightgray] (0.5,0) rectangle node[black,font=\footnotesize] {Job 1} (1.5,1);
		\fill [orange,draw=lightgray] (1.5,0) rectangle node[black,font=\footnotesize] {Job 2} (2.5,1);
		\fill [orange,draw=lightgray] (2.5,0) rectangle node[black,font=\footnotesize] {Job 3} (3.5,1);
	
		\fill [orange,draw=lightgray] (5,1.5) rectangle node[black] {Simulation $n+1$} (9,2.5);
		\fill [orange,draw=lightgray] (5,0) rectangle node[black,font=\footnotesize] {Job 1} (6,1);
		\fill [orange,draw=lightgray] (6,0) rectangle node[black,font=\footnotesize] {Job 2} (7,1);
		\fill [orange,draw=lightgray] (7,0) rectangle node[black,font=\footnotesize] {Job 3} (8,1);
	
		\fill [magenta] (0.5,-0.5) rectangle node[black]{Render $n-1$} (4.2,-1.5);
		\fill [magenta] (5,-0.5) rectangle node[black]{Render $n$} (8.7,-1.5);
		
		\node at (2.5,3.5) {Frame $n$};
		\node at (7,3.5) {Frame $n+1$};
		
		\draw  (0.5,3) rectangle (4.5,-2);
		\draw  (5,3) rectangle (9,-2);
	\end{tikzpicture}
	\caption{Design der Threadingarchitektur der Blocklib}\label{fig:plannedArchitecture}
	\todo{Caption}
\end{figure}

Aufgrund des Umfangs der Blocklib und der Unübersichtlichkeit an einigen Stellen kann das Jobsystem nicht, wie in Abbildung~\ref{fig:optimalArchitecture} gezeigt, umfassend integriert werden. Um die Anforderungen von Kapitel~\ref{sec:anforderungen} zu erfüllen, wird daher ein Jobsystem implementiert, das nur an ausgewählten Stellen mit der Blocklib konsolidiert wird und ansonsten für die zukünftige Nutzung bereitsteht.

Da das Jobsystem keine vollständige Integration in die Blocklib erfährt, verändert sich die Architektur konzeptuell leicht. Diese Änderung ist in Abbildung~\ref{fig:plannedArchitecture} zu sehen. Anstatt die gesamte Simulation in viele kleine Jobs zu zerlegen, bleibt eine sequentialisierte Simulation bestehen, die dann aber die Möglichkeit besitzt, weitere nebenläufige Jobs zu starten. Mit dieser Architektur ist es möglich, die Blocklib inkrementell zu der in Abbildung~\ref{fig:optimalArchitecture} gezeigten Architektur umzuwandeln, indem in zukünftigen Arbeiten immer mehr pseudosequentialisierte Anweisungen der Simulation als nebenläufige Jobs definiert werden, bis sie vollständig aus Jobs besteht.

Java bietet mit dem Interface \class{ExecutorService} bereits eine gute Schnittstellendefinition für ein Jobsystem. Daher baut das Design des Jobsystems der Blocklib auf diesem Interface auf. 

\begin{figure}
	\includesvg[width=\textwidth]{GrobesDesign.svg}
	\caption{Struktur }\label{fig:GrobesDesign}
	\todo{Caption}
\end{figure}

In Abbildung~\ref{fig:GrobesDesign} ist die Struktur des Designs für das Jobsystem der Blocklib dargestellt. Es wird ein Interface \class{BlocklibExecutorService} definiert, das von dem Interface \class{ScheduledExecutorService} der Java Bibliothek abgeleitet ist. Das Interface wird so erweitert, dass die verschiedenen \code{submit(...)} Methoden jeweils Objekte vom Typ \class{CompletableFuture} zurückgeben. Die \code{schedule(...)} Methoden geben jeweils ein \class{ScheduledCompletableFuture} Objekt zurück, das später noch näher erläutert wird.

Das Interface \class{BlocklibExecutorService} wird durch die Klasse \class{BlocklibExecutor} implementiert. Sie nutzt zur Durchführung von Jobs den von der Java Bibliothek definierten \class{ScheduledThreadPoolExecutor}. Um die für die Rückgabewerte nötigen \class{CompletableFuture} Objekte zu erzeugen, wird eine Klasse \class{CompletableFutueWrapper} erstellt. Eine vollständige Auflistung der von den drei Interfaces bereitgestellten Methoden ist in Anhang~\ref{appendix:BlocklibExecutorService} zu finden. 

Die API des Jobsystems bietet verschiedene Methoden für zum Starten von nebenläufigen Anweisungen. Mittels der \code{submit(...)} Methoden können Anweisungen definiert werden, die sobald wie möglich ausgeführt werden. Da diese Methoden ein \class{CompletableFuture} Objekt zurückgeben, lassen sich über die dort definierten Methoden einfach nachgelagerten nebenläufige Anweisungen definieren, die abhängig von der Vollendung der ursprünglichen Anweisung ausgeführt werden. Mit den \code{schedule(...)} Methoden wird analog dazu eine Möglichkeit geboten Anweisungen zu definieren, die nach Ablauf eines bestimmten Zeitintervalls nebenläufig ausgeführt werden. Mittels \code{scheduleAtFixedRate(...)} und \code{scheduleWithFixedDelay(...)} können periodisch durchzuführende Anweisungen zur Ausführung gebracht werden.

\section{Implementierung der Threading API}
Wie das Design, ist auch die Implementierung der nebenläufigen Architektur prinzipiell in zwei Bereiche unterteilt, die Implementierung des Renderthreads und die Implementierung der Job API.

Um einen Renderthread in die Architektur der Blocklib zu integrieren, muss besonders darauf geachtet werden, dass Wettkampfbedingungen vermieden werden. Wie in Abschnitt~\ref{sec:desgignRenderthread} kann dies unter anderem durch die Nutzung von Double Buffern gelingen. Folgend wird die Implementierung des Renderthreads beschrieben. Im darauffolgenden Abschnitt wird die Implementierung des Jobsystems beschrieben.
\subsection{Implementierung des Renderthreads}
Da ein Jobsystem implementiert wird, bietet sich der \code{main}-Thread als Renderthread an. Alle anderen Rechenprozesse werden über das Jobsystem auf anderen Threads ausgeführt. Bei der Implementierung der für einen Renderthread nötigen Funktionalität müssen in der Blocklib viele verschiedene Klassen angepasst werden. Der Übersichtlichkeit halber werden die Anpassungen in Bereiche eingeteilt. 

In Abschnitt~\ref{sec:loader} wird beschrieben, wie das Laden von Daten auf die \ac{gpu} geändert werden muss, nachdem dieser Vorgang aus Sicht der Simulation nun wegen der Nebenläufigkeit des Renderthreads asynchron ist. Danach erläutert Abschnitt~\ref{sec:statelessRendering} den Wechsel von einer Renderarchitektur,die sich zu zeichnende Elemente merkt, zu einer zustandslosen Architektur. Abschnitt~\ref{sec:saveRenderState} beschreibt, wie darauf aufbauend der Zustand der zu zeichnenden Elemente unter Verwendung von Double-Buffers zwischengespeichert wird, damit er während des Renderings unverändert bleibt. In Abschnitt~\ref{sec:adjustGameLoop} wird schließlich beschrieben, wie der Aufbau der Game-Loop an die neue Architektur angepasst wird.

\subsubsection{Laden von Daten auf die \glsentryuseri{gpu}}\label{sec:loader}
In der Blocklib werden häufig Daten an die \ac{gpu} übergeben. Beispielsweise muss bei der Generierung eines Chunks ein Gitternetz konstruiert werden, das die Form des Terrains im Chunk beschreibt, und dann an die \ac{gpu} gesendet werden. Dafür wird die OpenGL API genutzt. Da alle Aufrufe an OpenGL auf einem Thread ausgeführt werden müssen, ist dafür zu sorgen, dass Anweisungen zwischengespeichert werden, die mit der \ac{gpu} interagieren, aber zum Beispiel durch die Chunkgenerierung angestoßen werden. Dann können die zwischengespeicherten Anweisungen vom Renderthread ausgelesen und ausgeführt werden. 

Um Anweisungen zum Laden von Daten auf die \ac{gpu} zwischenzuspeichern, muss die dafür zuständige Klasse \class{Loader} angepasst werden. Die Klasse wird um eine Warteschlange erweitert, die Objekte vom Typ \class{Runnable} enthält. Anstatt die Ladeanweisungen direkt auszuführen, werden sie als Tasks der Warteschlange übergeben und dort als \class{Runnable}-Objekte gespeichert. Die Warteschlange nutzt intern die von Java bereitgestellte Klasse \class{ConcurrentLinkedQueue}. Die Warteschlange bietet damit nicht-blockierende Synchronisierung, wodurch mehrere Threads gleichzeitig Elemente der Warteschlange hinzufügen können, ohne dass das zu Wettkampfbedingungen führen kann. In der Game-Loop der Blocklib wird diese Warteschlange zu Beginn vom Renderthread abgearbeitet, damit danach sowohl die Simulation als auch der Renderer darauf Zugriff haben. 

\begin{figure}
	\includesvg{Loader.svg}
	\caption{Klassendiagramm der wichtigsten Klassen, die am Laden und Entfernen von Daten auf und von der \ac{gpu} beteiligten sind. Zum Laden wird der Loader aufgerufen, dieser erstellt einen Task, der eine der drei Transmitter-Klassen aufruft. Die so erstellten Tasks werden an \class{DoubleBufferedAsyncQueue} übergeben, die ein \class{CompletableFuture}-Objekt zurückgibt.}\label{fig:loaderDiagram}
\end{figure}
Ein Diagramm der wichtigsten beteiligten Klassen ist in Abbildung~\vref{fig:loaderDiagram} gezeigt. Im Klassendiagramm sind die drei Klassen \class{VAOTransmitter}, \class{TextureTransmitter} und \class{DataTransmitter} abgebildet. Die Klasse \class{Loader} erzeugt Tasks, die dann Methoden dieser Klassen aufrufen, je nachdem, welche Art von Daten an die \ac{gpu} übertragen oder von dor gelöscht werden soll. Die Klasse \class{VAOTransmitter} ermöglicht den Umgang mit sogenannten \acp{vao}. Das ist eine Datenstruktur, die alle relevanten Informationen zur Darstellung eines Modells enthält, beispielsweise eines Chunks oder des Spielercharakters. Mit der Klasse \class{TextureTransmitter} lassen sich Texturen übertragen, die beispielsweise das Aussehen von Blöcken definieren. Der \class{DataTransmitter} wird genutzt, um die Daten an die \ac{gpu} zu senden, die für die Berechnung von Niederschlag, wie Regen und Schnee, benötigt werden.


Die Abarbeitung aller Ladevorgänge der Klasse \class{Loader} muss synchronisiert sein, da die Simulation sonst beispielsweise versuchen könnte ein Objekt rendern zu lassen, das noch nicht geladen wurde. Nachdem die Daten eines Objekts auf die \ac{gpu} geladen worden sind, kann das Objekt dann gerendert werden.

\subsubsection{Zustandsloses Rendering}\label{sec:statelessRendering}
In der Blocklib implementieren alle zu zeichnenden Elemente das Interface \class{Renderable}. Dieses definiert die Funktionalität, die zum Zeichnen eines Elements notwendig ist, sowie die Methoden \code{show()} und \code{hide()}. Das Rendersystem ist so aufgebaut, dass ein Element nach einem Aufruf von \code{show()} solange gezeichnet wird, bis \code{hide()} aufgerufen wird. Dies wird erreicht, indem eine Datenstruktur im \class{MasterRenderer} alle Renderables speichert. Diese zustandsbehaftete Zeichenmethode birgt Vor- und Nachteile.

\begin{itemize}
	\item[$+$] Da \code{show()} nur einmal aufgerufen werden muss, können auch Systeme ohne Updatemethode das Zeichnen von Elementen veranlassen.
	\item[$+$] Da die Elemente gespeichert sind, müssen sie nicht jedes Mal neu hinzugefügt werden. Das verringert den Rechenbedarf.
	\item[$-$] Werden Renderables häufig ausgetauscht, müssen die alten Elemente jedes Mal entfernt werden.
	\item[$-$] Da die Einführung eines parallelen Renderthreads ansonsten zu Race Conditions führen würde, muss diese Datenstruktur aus Sicht des Renderthreads während des Zeichnens unverändert sein. Wie beschrieben, wird dazu ein Double Buffer eingesetzt. Bei einem zustandsbehafteten Double Buffer müssen beim Swap die Elemente des einen Buffers in den anderen Buffer kopiert werden. Das erfordert Zeit, die nicht parallelisiert werden kann, da das Wechseln des Buffers synchronisiert sein muss.
	\item[$-$] Die Existenz von paarweise aufzurufenden Funktionen birgt die Gefahr, dass der zweite Aufruf vergessen wird. Wie bei \code{malloc()} und \code{free()} in C entsteht durch einen fehlenden Aufruf von \code{hide()} ein Speicherleck. Zudem wird dann die Anzahl der zu zeichnenden Elemente immer größer und der Rendervorgang wird verlangsamt. Des Weiteren werden möglicherweise Renderables gezeichnet, die nicht gezeichnet werden sollen.
\end{itemize}

Um den Zeichenaufwand zu verringern, implementiert die Klasse \class{ChunkManager} das sogenannte \emph{Frustum Culling}. Es wird berechnet, welche Chunks sich im Sichtfeld der Kamera befinden. Nur diese Chunks sollen gezeichnet werden. Dazu entfernt der ChunkManager jeden Frame alle Chunks mittels \code{hide()} aus der Datenstruktur der zu zeichnenden Elemente und fügt nur die als sichtbar ermittelten Chunks wieder ein. Die Chunks machen mit etwa 
% 334*2/(334*2+4+73+142)
75 \% bis
% 529*2/(529*2+4+73+142)
82 \% einen Großteil aller Renderables aus. Es wird also bereits ein Großteil der zu zeichnenden Elemente in jedem Frame neu hinzugefügt. Des Weiteren gibt es in der Blocklib bis jetzt keine Klasse, die Renderables zu der Datenstruktur hinzufügt, aber keine Updatemethode besitzt. Somit werden die Gefahr von vergessenen \code{hide()} Aufrufen und der Kopieraufwand des Double Buffers als wichtiger eingeschätzt als die oben beschriebenen Vorteile.

Die Datenstruktur ist nun wie folgt implementiert. Die Methode \code{hide()} entfällt ersatzlos. Stattdessen wird nach dem Swap des Double Buffers der Renderables, der Buffer für die nächsten zu zeichnenden Elemente geleert, sodass Renderables automatisch nicht mehr gezeichnet werden, wenn sie nicht mehr hinzugefügt werden. Die Methode \code{show()} wird zu \code{draw()} umbenannt, um zu signalisieren, dass es sich um einen einmaligen Vorgang handelt. Alle bisherigen Aufrufe von \code{show()} werden durch \code{draw()} ersetzt. Die Update-Methoden aller Klassen, die \class{Renderable}-Objekte anzeigen lassen, nutzen die neue \code{draw()}-Methode, sodass die Objekte jeden Frame zu der Liste der zu zeichnenden Objekte hinzugefügt werden und das Verhalten zum vorherigen Stand identisch ist.

\subsubsection{Zwischenspeicherung des gerenderten Spielzustands}\label{sec:saveRenderState}
Aufbauend auf die Änderungen des vorherigen Abschnitts ist nun die Zwischenspeicherung der \class{Renderable}-Objekte vereinfacht, da diese nur für ein Bild gespeichert werden müssen. 

Neben den Informationen, die in den  \class{Renderable}-Objekten gespeichert ist, umfasst der für das Rendering relevante Spielzustand noch weitere Daten. Die Zwischenspeicherung der Renderables erfolgt auf zwei Arten. Zum einen muss gespeichert werden, welche Elemente im aktuellen Frame gezeichnet werden, zum andern muss sichergestellt werden, dass sich der Inhalt der zu zeichnenden \class{Renderable}-Objekte nicht verändert. Die \ac{gui} wird auf eine andere Weise gezeichnet, als andere Elemente, da die ihr zugrundeliegende Baumstruktur die Anordnung der Renderables bestimmt. Die Blocklib besitzt eine spezielle Klasse \class{UIRenderer}, die das Rendering der \ac{gui} durchführt. Abbildung~\vref{fig:renderDiagram} zeigt die wichtigsten Klassen für die Ausführung des Rendering und die Zwischenspeicherung des gerenderten Spielzustands.
\begin{figure}
	\centering
	\includesvg{Rendering.svg}
	\caption{Klassendiagramm Rendering}\label{fig:renderDiagram}
\end{figure}
Alle in der Abbildung gezeigten Klassen implementieren das Interface \class{DoubleBuffer}, wie in Abbildung~\vref{fig:renderInterfaceDiagram} dargestellt.
\begin{figure}
	\centering
	\includesvg[width=\textwidth]{Rendering-Interface.svg}
	\caption{Klassendiagramm Rendering}\label{fig:renderInterfaceDiagram}
\end{figure}



Die Blocklib enthält sogenannte \emph{globale} \glspl{uniform}. Ein \gls{uniform}~\cite[S.~45~ff.]{Vries2020} ist eine globale Variable die in einem \gls{shaderprogram} definiert wird. Ein \gls{shaderprogram} ist ein Programm, das auf der \ac{gpu} ausgeführt wird~\cite[S.~32~f.]{Vries2020}. Die in der Blocklib als globale \glspl{uniform} bezeichneten Variablen sind \glspl{uniform}, die alle \glspl{shaderprogram} nutzen können, also auch über die \glspl{shaderprogram} hinweg global sind. Da sich in den globalen \glspl{uniform} Informationen wie die Position und Blickrichtung der Kamera gespeichert sind, müssen auch diese Daten während des Renderings unverändert bleiben. Die Simulation ändert beispielsweise die Position der Kamera, wenn eine Bewegungstaste gedrückt wird. Daher müssen die globalen \glspl{uniform} ebenfalls zwischengespeichert werden.

Die globalen \glspl{uniform} sind in der Klasse \class{GlobalUniforms} als statische Attribute gespeichert.


\subsubsection{Anpassung der Game-Loop}\label{sec:adjustGameLoop}
In der Blocklib implementieren alle zu zeichnenden Elemente das Interface \class{Renderable}. Dieses definiert die Funktionalität, die zum Zeichnen eines Elements notwendig ist, sowie die Methoden \code{show()} und \code{hide()}. Das Rendersystem ist so aufgebaut, dass ein Element nach einem Aufruf von \code{show()} solange gezeichnet wird, bis \code{hide()} aufgerufen wird. Dies wird erreicht, indem eine Datenstruktur im \class{MasterRenderer} alle Renderables speichert. Diese zustandsbehaftete Zeichenmethode birgt Vor- und Nachteile.

\todo{itemize mit + und - ?}
\begin{itemize}
	\item[$+$] Da \code{show()} nur einmal aufgerufen werden muss, können auch Systeme ohne Updatemethode das Zeichnen von Elementen veranlassen.
	\item[$+$] Da die Elemente gespeichert sind, müssen sie nicht jedes Mal neu hinzugefügt werden. Das verringert den Rechenbedarf.
	\item[$-$] Werden Renderables häufig ausgetauscht, müssen die alten Elemente jedes Mal entfernt werden.
	\item[$-$] Da die Einführung eines parallelen Renderthreads ansonsten zu Race Conditions führen würde, muss diese Datenstruktur aus Sicht des Renderthreads während des Zeichnens unverändert sein. Wie beschrieben, wird dazu ein Double Buffer eingesetzt. Bei einem zustandsbehafteten Double Buffer müssen beim Swap die Elemente des einen Buffers tatsächlich in den anderen Buffer kopiert werden. Das erfordert Zeit, die nicht parallelisiert werden kann, da das Wechseln des Buffers synchronisiert sein muss.
	\item[$-$] Die Existenz von paarweise aufzurufenden Funktionen birgt die Gefahr, dass der zweite Aufruf vergessen wird. Wie bei \code{malloc()} und \code{free()} in C entsteht durch einen fehlenden Aufruf von \code{hide()} ein Speicherleck. Zudem wird dann die Anzahl der zu zeichnenden Elemente immer größer und der Rendervorgang wird verlangsamt. Des weiteren werden möglicherweise Renderables gezeichnet, die nicht gezeichnet werden sollen.
\end{itemize}

Um den Zeichenaufwand zu veringern, implementiert die Klasse \class{ChunkManager} das sogenannte \emph{Frustum Culling}. Es wird berechnet, welche Chunks sich im Sichtfeld der Kamera befinden. Nur diese Chunks sollen gezeichnet werden. Dazu entfernt der ChunkManager jeden Frame alle Chunks mittels \code{hide()} aus der Datenstruktur der zu zeichnenden Elemente und fügt nur die als sichtbar ermittelten Chunks wieder ein. Die Chunks machen mit etwa 
% 334*2/(334*2+4+73+142)
75 \% bis
% 529*2/(529*2+4+73+142)
82 \% einen Großteil aller Rendereables aus. Es wird also bereits ein Großteil der zu zeichnenden Elemente in jedem Frame neu hinzugefügt. Des weiteren gibt es in der Blocklib bis jetzt keine Klasse, die Renderables zu der Datenstruktur hinzufügt, aber keine Updatemethode besitzt. Somit werden die Gefahr von vergessenen \code{hide()} Aufrufen und der Kopieraufwand des Double Buffers als wichtiger eingeschätzt, als die oben beschriebenen Vorteile.

Die Datenstruktur ist nun wie folgt implementiert. Die Methode \code{hide()} entfällt ersatzlos. Stattdessen wird nach dem Swap des Double Buffers der Renderables, der Buffer für die nächsten zu zeichnenden Elemente geleert, sodass Renderables automatisch nicht mehr gezeichnet werden, wenn sie nicht mehr hinzugefügt werden. Die Methode \code{show()} wird zu \code{draw()} umbenannt, um zu signalisieren, dass es sich um einen einmaligen Vorgang handelt. Alle bisherigen Aufrufe von \code{show()} werden durch \code{draw()} in Updatemethoden ersetzt, sodass das Verhalten zum vorherigen Stand identisch ist.
\paragraph{Schattenflackern}
Bei der parallelen Ausführung des Renderthreads ist bei der Darstellung der Schatten ein Flackern zu erkennen. Die Tatsache, dass das Flackern in irregulären Intervallen auftritt, lässt auf das Vorhandensein einer Race Condition schließen. Diese Vermutung kann dadurch bestätigt werden, dass das Flackern nichtmehr auftritt, sobald \code{update()} und \code{render()} sequentialisiert werden.

Da Variablen zeitgleich von verschiedenen Threads bearbeitet werden, ist die Suche nach dem Ursprung des Flackerns schwierig. Da der Fehler nur bei nebenläufiger Ausführung auftritt, müssen die problematischen Stellen auf \code{update()} und \code{render()} verteilt sein. Da der Fehler visuell sichtbar ist, lässt sich das Problem über die Anweisungen in \code{render()} nachverfolgen. 

Eine naive Herangehensweise mittels Debugging ist nahezu unmöglich, da das Flackern zufällig auftritt. Zuerst muss also eine Möglichkeit gefunden werden, um das Auftreten des Flackerns programmatisch zu erkennen.

In der \class{Configuration}-Klasse gibt es die Variable \const{SHOW_SHADOW_MAP_FRAME}. Ist diese auf \code{true} gesetzt, wird in der linken oberen Ecke des Fensters ein Feld angezeigt, das die sogenannte \class{ShadowMap} zeigt. Abbildung~\ref{fig:ShadowMap} zeigt einen Screenshot der Blocklib mit \code{SHOW_SHADOW_MAP_FRAME = true}. Die \class{ShadowMap} wird genutzt, um zu berechnen, an welchen Stellen Schatten gerendert werden.
\begin{figure}
	\caption{Screenshot}\label{fig:ShadowMap}
\end{figure}
\textcite{Ebbinger2018} beschreibt, wie genau das Rendering von Schatten in der Blocklib umgesetzt ist. Wie der Name der Klasse \class{ShadowMap} vermuten lässt, wird für die Berechnung der Schatten das sogenannte \emph{Shadow Mapping} verwendet. Dazu wird eine zweite Kamera implementiert, die die Spielwelt aus der Sicht einer Lichtquelle, im Fall der Blocklib die Sonne, betrachtet. Die so gerenderte Sicht wird dann genutzt, um zu berechnen ob ein Fragment das erste Hindernis aus Sicht der Lichtquelle ist. Ist dies der Fall, so wird das Fragment beleuchtet, ansonsten ist etwas anderes vor dem Fragment und es befindet sich im Schatten.

Wird die \class{ShadowMap} mitangezeigt, so kann man erkennen, dass diese sich zeitgleich mit dem Flackern verschiebt. Die Position der \class{ShadowMap} wird über die Klasse \class{ShadowBounds} bestimmt. Die Position muss in zwei Fällen geändert werden, zum einen, wenn die Sonne sich bewegt, also, wenn der Tag-Nacht-Zyklus aktiv ist, zum anderen, wenn sich die Spielerkamera bewegt, da die Darstellung von Schatten immer mit dem Sichtfeld des Spielers übereinstimmen muss.

Die Ausgabe der Position der \class{ShadowBounds} während des Spiels bei deaktiviertem Tag-Nacht-Zyklus und stillstehender Spielerkamera bestätigt, dass diese Position sich tatsächlich zeitgleich mit dem Auftreten des Flackerns verändert. Das Auftreten der Wettkampfbedingung lässt sich also erkennen, indem eine Änderung der Position der \class{ShadowBounds} in aufeinanderfolgenden Frames erkannt wird.

Nun gilt es die Ursache der Positionsänderung ausfindig zu machen. Die Position der \class{ShadowBounds} hängt selbst von mehreren Variablen ab. Verfolgt man den Verlauf der Änderungen, finden sich die folgenden Abhängigkeiten:

\begin{tabular}{ll}
	\class{ShadowMap} &$\to$ \code{ShadowBounds.update}\\
	& $\to$ \class{LightViewMatrix}\\
	& $\to$ \code{DayNightLighting.getSunUp()} \\
	& $\to$ \code{DayNightLighting.position}\\
	& $\to$ \code{DayNightLightig.updateLightPosition(float, boolean)}
\end{tabular}

\code{DayNightLightig.updateLightPosition(float, boolean)} wird nicht im Renderthread ausgeführt, sondern während \code{update()} in einem anderen Thread. Der relevante Abschnitt der Methode ist 
\begin{lstlisting}[]
private void updateLightPosition(float progress, boolean day) {
	// ...
	position = new Vector3f(direction.x, direction.y, direction.z);
	position.scale(SUN_HEIGHT); (*\label{lst:updateLightPosition:scale}*)
	// let sun stay relative to the player
	position = Vector3f.add(position, Context.getInstance().getCamera().getPosition(), new Vector3f());
}
\end{lstlisting}
Wird also der Wert von \code{DayNightLighting.position} im Renderthread ausgelesen, während die Methode in Ausführung ist, beispielsweise während Zeile~\ref{lst:updateLightPosition:scale}, so enthält \code{DayNightLighting.position} einen vollkommen falschen Wert. Hier existiert also die gesuchte Wettkampfbedingung.
\subsection{Implementierung des Jobsystems}
Bei der Implementierung des Jobsystems liegt die Aufgabe insbesondere darin, die Verkettung von Jobs durch \class{CompletableFuture} zu ermöglichen. Diese Verkettung ist nämlich die Voraussetzung dafür, dass die Simulation später Schritt für Schritt in Jobs zerlegt werden kann, die dann aufeinander aufbauen und einen Jobgraphen bilden. 

Während der Implementierung ist eine weitere Anforderung zum Vorschein gekommen, die aufgrund des Zeitpunkts, zu dem sie aufgetaucht ist, nicht vollständig gelöst ist. Dabei handelt es sich um die in Abschnitt~\ref{sec:reqBackgroundTasks} beschriebene Anforderung, die Ausführung von Hintergrundtasks zu ermöglichen, ohne die Ausführung anderer Tasks zu blockieren. 

Den Zugriffspunkt auf das Jobsystem bietet das neu definierte Interface \class{BlocklibExecutorService}, das von der Klasse \class{BlocklibExecutor} implementiert wird. Ein gekürztes Klassendiagramm ist in Abbildung~\vref{fig:diag-BlocklibExecutor} dargestellt.

\begin{figure}[!htb]
	\centering
	\includesvg[width=\textwidth]{BlocklibExecutor-shortened.svg}
	\caption{Klassendiagramm von \class{BlocklibExecutor}, das nur die \emph{nicht} von dem Interface \class{BlocklibExecutorService} definierten Attribute und Methoden zeigt.}\label{fig:diag-BlocklibExecutor}
\end{figure}

Von der Klasse \class{BlocklibExecutor} soll dauerhaft nur eine Instanz existieren, damit die Verwaltung der nebenläufigen Threads von einer Stelle aus stattfinden. Es steht in Aussicht, dass die Nutzung globaler Objekte wie \class{Context} in Zukunft verringert wird. Das wird noch genauer im Ausblick in Abschnitt~\vref{sec:EtablierungEinerKompositionroot} erläutert. Daher wird bei hier nicht auf das Designpattern \gls{singleton} zurückgegriffen sondern das in den Grundlagen beschriebene lokale Singleton-Design implementiert. Um sicherzustellen, dass von der Klasse nur eine Instanz erzeugt wird, enthält \class{BlocklibExecutor} ein statisches Attribut \var{instanced: AtomicBoolean}. Beim Aufruf des Konstruktors wird mit der Methode \code{ensureUniqueInstance()} geprüft, ob bereits eine Instanz der Klasse erzeugt worden ist. \var{instanced} besitzt den Typ \code{AtomicBoolean}, um sicherzustellen, dass diese Prüfung nicht-blockierend synchronisiert ist. Existiert bereits eine Instanz, wird eine \class{UnsupportedOperationException} geworfen, die darauf hinweist.

Die Klasse besitzt Referenzen auf zwei Instanzen von \class{ScheduledThreadPoolExecutor}. Damit wird die Anforderung, unterschiedliche Prioritäten von Jobs zu unterstützen behelfsmäßig unterstützt. Im Gegenzug wird jedoch darauf verzichtet, die Anzahl der Java Threads genau an die Anzahl der Hardwarethreads anzupassen. Für jeden der beiden \class{ScheduledThreadPoolExecutor} werden $\text{Anzahl(Hardwarethreads)} - 1$ Java-Threads erzeugt. Damit entsteht zwar ein Wettstreit um die Belegung der Hardwarethreads, aber nur, wenn in beiden Systemen auch genügend Jobs gleichzeitig vorhanden sind. Aktuell ist das in der Regel nicht der Fall, da die Simulation noch weitgehend sequentiell als ein Job auf einem Thread ausgeführt wird und die konsolidierten Jobs sehr unregelmäßig sind. Die Konsolidierung der bereits vorhandenen Nebenläufigkeit in das Jobsystem wird in Abschnitt~\ref{sec:Konsolidierung} noch genauer erläutert. 

In den folgenden Abschnitten wird die Implementierung des Jobsystems erläutert indem die unterschiedlichen Aufgabenbereich jeweils näher beleuchtet werden. Zuerst wird in Abschnitt~\ref{sec:Koordinierung} näher beschrieben, wie die Koordinierung das Jobsystems aufgebaut ist und wie andere Module auf das Jobsystem zugreifen können. In Abschnitt~\ref{sec:Verkettung} wird beschrieben, wie das System die Komposition von Jobs ermöglicht. Zuletzt wird in Abschnitt~\ref{sec:Konsolidierung} die oben bereits erwähnte Konsolidierung von Blocklib und Jobsystem beschrieben.

\subsubsection{Koordinierung durch \texttt{BlocklibExecutor}}\label{sec:Koordinierung}

Wie bereits beschrieben dient die Klasse \class{BlocklibExecutor} als Zugriffspunktpunkt auf das neu implementierte Jobsystem, indem sie das Interface \class{BlocklibExecutorService} implementiert. Um der Bibliothek möglichst früh zur Verfügung zu stehen, wird \class{BlocklibExecutor} im Konstruktor der \class{Game}-Klasse instanziiert, die als Einstiegspunkt für die Blocklib dient. Wie die Abbildung~\vref{fig:diag-BlocklibExecutor} zeigt, besitzt die Klasse \class{BlocklibExecutor} eine statische Methode \code{defaultConfig()}, die eine \class{BlocklibExecutor}-Instanz in einer Standardkonfiguration erzeugt.

Aktuell bedeutet das, dass die oben erwähnten \class{ScheduledThreadPoolExecutor} erzeugt werden und jeweils $\text{Anzahl(Hardwarethreads)} - 1$ Java-Threads besitzen. Um die Threads beispielsweise beim Debugging zuordnen zu können, werden die Threads über die neue Klasse \class{NamedThreadFactory} erzeugt, die es ermöglicht den erzeugten Threads ein Namenspräfix zu geben. Dadurch besitzen alle Job-Threads den Namen \emph{Jobs-thread-\{x\}} und alle Hintergrund-Threads den Namen \emph{Background-thread-\{x\}}, wobei \{x\} jeweils durch eine Zahl von 1 bis $\text{Anzahl(Hardwarethreads)} - 1$ ersetzt wird.

Die in \class{Game} erzeugte \class{BlocklibExecutor}-Instanz wird dann unter anderem an den Konstruktor der \class{Context}-Klasse übergeben, die momentan für die Instanziierung der meisten Module der Blocklib verantwortlich ist. Hier wird der \class{BlocklibExecutor} an die Konstruktoren der Module übergeben, die das Jobsystem nutzen.

Die Implementierung von priorisierten Jobs wird aktuell durch die Nutzung zweier getrennter \class{Executor}-Instanzen gelöst. Das lässt sich durch das Design des Interface \class{BlocklibExecutorService} leicht anzupassen, da dieses für die Definition von Prioritäten den Aufzählungstyp \class{TaskPriority} nutzt. \class{TaskPriority} enthält aktuell nur die Werte \const{NORMAL} und \const{BACKGROUND}, kann aber sofort erweitert werden, sobald die Implementierung des Executors mehr Prioritäten zulässt. 

Ziel für die Zukunft ist es, dass statt der durch Java bereitgestellten \class{ScheduledThreadPoolExecutor} Klasse ein vollkommen eigenständiger \class{ExecutorService} entwickelt wird, der das Interface \class{BlocklibExecutorService} implementiert. Dieser benötigt eine zweistufige Prioritäts-Warteschlange. Die erste Priorisierung ergibt sich aus den Anforderungen von \class{ScheduledExectutorService}. Danach werden die Tasks nach ihrem Zeitplan sortiert, Tasks die sofort ausgeführt werden sollen müssen vor Tasks einsortiert werden, die erst in einer Sekunde ausgeführt werden sollen. Bei Tasks, die den selben Zeitplan besitzen müssen dann nach ihrer Priorität (\class{TaskPriority}) sortiert werden. Bei gleicher Priorität kann eine beliebige Sortierung verwendet werden. Die Einfüge-Reihenfolge bietet sich aber an, sodass hinzugefügte Tasks auch zuerst ausgeführt werden (engl. first-in first-out (FIFO)).

Bei eingehenden Tasks prüft der \class{BlocklibExecutor} die Priorität und wählt danach den entsprechenden Executor aus. Dann muss der Task noch so verpackt werden, dass eine Verkettung von Tasks möglich ist.

\subsubsection{Komposition von Tasks}\label{sec:Verkettung}
Für die Verpackung von Tasks, sodass diese komponiert werden können, ist die Klasse \class{CompletableFutureWrapper} zuständig. Wie in Abbildung~\vref{fig:wrapper} gezeigt, erzeugt die Klasse dafür Klassen, die \class{Callable} implementieren und Klassen, die von \class{CompletableFuture} abgeleitet sind.
\begin{figure}
	\centering
	\includesvg[width=.9\textwidth]{Wrapper.svg}
	\caption{}\label{fig:wrapper}
\end{figure}
Die Klasse \class{CompletingCallable} enthält die übergebene \class{TaskPriority} und eine Sequenznummer für eine zukünftige Prioritäts-Warteschlange. Zudem wird bei der Konstruktion der ursprüngliche Task und ein \class{CompletableFuture}-Objekt übergeben. 
\begin{lstlisting}[caption={\code{call()}-Methode von \class{CompletingCallable}. Das enthaltene \class{Callable} wird ausgeführt und dann das \class{CompletableFuture} abgeschlossen.}, label={lst:CompletingCall},float]
	public V call() throws Exception {
		try {
			var result = innerCallable.call(); (*\label{lst:CompletingCall:call}*)
			completableFuture.complete(result); (*\label{lst:CompletingCall:complete}*)
			return result;
		} catch (Exception e) {
			completableFuture.completeExceptionally(e); (*\label{lst:CompletingCall:exceptionally}*)
			throw e;
		}
	}
\end{lstlisting}
Wie in Listing~\vref{lst:CompletingCall} zu sehen, führt das \class{CompletingCallable} in seiner \code{call()}-Methode von die \code{call()}-Methode des ihm übergebenen \class{Callable} \var{innerCallable} aus und speichert den Rückgabewert (Zeile~\ref{lst:CompletingCall:call}). Ist der Aufruf erfolgreich, wird das übergebene \class{CompletableFuture} mit dem Rückgabewert abgeschlossen (Zeile~\ref{lst:CompletingCall:complete}). Tritt ein Fehler auf, so wird das  \class{CompletableFuture} mit dem entstandenen Fehler abgeschlossen (Zeile~\ref{lst:CompletingCall:exceptionally}).

Auf diese Weise kann das übergebene \class{CompletableFuture} genutzt werden, um nach Abschluss des inneren \class{Callable} weitere Tasks anzuhängen.
Da das Interface \class{ExecutorService} die Übergabe von \class{Callable} aber auch \class{Runnable} unterstützt, werden aus \class{Runnable}-Objekten mit der Klasse \class{RunnableCallable} in \class{Callable}-Objekte verpackt.

\subsubsection{Konsolidierung mit dem bestehenden System}\label{sec:Konsolidierung}



\section{Integration und Performanceanalyse}

\chapter{Fazit}
\chapter{Ausblick}
\section{Etablierung einer Composition Root}
\textcite[S.~76]{Seemann2012} definiert die Composition Root als (vorzugsweise) genau eine Position in einer Anwendung an der Module zusammengesetzt werden.

\subsection{}
\section{Dynamisches Rendern von Schrift}
\section{Thread Affinity}
\section{Jobsystem mit PriorityQueue}
\section{Konsolidieren des Jobsystems in der Simulation}
\section{Verstärktes Batching in der Simulation}
\printnoidxglossaries

\printbibliography[title={Literaturverzeichnis},heading=bibintoc,notkeyword=online]

\printbibliography[title={Quellenverzeichnis},heading=bibintoc,keyword=online] 

\appendix
\clearpage
\pagenumbering{Roman}
\setcounter{page}{1}

\section{Executor im Concurrency Framework}\label{appendix:concFrameworkExecutor}
\includesvg[width=\textwidth]{ConcurrencyFrameworkExecutor.svg}
\todo{Caption}

\section{\texttt{Context} vollständiges Klassendiagramm}\label{appendix:context}
{
	\centering
	\includesvg[height=.7\textheight]{Context.svg}
	\todo{Caption}
}

\section{\texttt{BlocklibExecutorService} Klassendiagramm}\label{appendix:BlocklibExecutorService}
{
	\centering
	\includesvg[width=\textwidth]{ExecuterInterfaces.svg}
	\todo{Caption}
}

\end{document}