% !TeX spellcheck = de_DE
% !LuaLaTeX
\documentclass[parskip=half,twoside,BCOR=2cm,11pt,DIV=10]{scrreprt}
\usepackage{scrhack}
\usepackage[ngerman]{babel}
\usepackage[T1]{fontenc}
\usepackage[utf8]{inputenc}
\usepackage{lmodern}
\usepackage{microtype}
\usepackage{geometry}
\geometry{a4paper, top=27mm, left=20mm, right=20mm, bottom=35mm, headsep=10mm, footskip=12mm}

\usepackage{amsmath,amssymb,amsthm,mathrsfs,amsfonts}

\usepackage{csquotes}
\usepackage{booktabs}

\usepackage{graphicx}
\graphicspath{ {./img/} }
\usepackage{wrapfig}
% \usepackage{layouts}
\usepackage{textcomp} 
\usepackage[pdftex,dvipsnames]{xcolor}  % Coloured text etc.
\usepackage{svg}
\svgsetup{inkscapelatex=false, inkscapearea=drawing}
\usepackage[layout={margin,index},draft]{fixme}
\usepackage{rotating}
\usepackage{siunitx}

\usepackage[backend=biber,minbibnames=2,maxbibnames=2,maxcitenames=1,mincitenames=1,style=alphabetic]{biblatex}
\setcounter{biburlnumpenalty}{9000}
\setcounter{biburlucpenalty}{9000}
\setcounter{biburllcpenalty}{9000}

% new stretchable space between characters
\setlength{\biburlnumskip}{0mu plus 1mu}
\setlength{\biburlucskip}{0mu plus 1mu}
\setlength{\biburllcskip}{0mu plus 1mu}
\renewcommand{\namelabeldelim}{\addnbspace}

\addbibresource{Recherche/My Collection.bib}
\defbibheading{Literatur}{\chapter{Literaturverzeichnis}} 
\defbibheading{Quellen}{\section*{Quellenverzeichnis}} 

\DeclareSourcemap{
  \maps{
    \map{
      \step[fieldsource=url,
            match=\regexp{\$\\sim\$},
            replace=\regexp{\~}]
    }
  }
}

\usepackage{tikz}
\usetikzlibrary{calc,positioning, shapes, petri, automata}
\usepackage{pdfpages}
\usepackage{subcaption}
\usepackage{standalone}
\usepackage{multirow,tabularx}
\usepackage[hidelinks]{hyperref}
\usepackage[acronym,shortcuts,toc]{glossaries}
\usepackage{mathtools}


%\usepackage{pgfplots}
%\pgfplotsset{width=\columnwidth,compat=1.14}\usepgfplotslibrary{statistics}
%\pgfplotsset{boxplot/.cd,every median/.style={red}}
%\pgfplotsset{grid style={help lines}}
%\pgfplotsset{minor grid style={very thin, dotted}}
%\pgfplotsset{major grid style={thick}}

\newcommand{\unsim}{\mathord{\sim}}
\newcommand{\e}{\ensuremath{\mathrm{e}}}

\newcommand{\R}{\ensuremath{\mathbb{R}}}
\newcommand{\N}{\ensuremath{\mathbb{N}}}
\newcommand{\F}{\ensuremath{\mathbb{F}}}
\newcommand{\Pot}{\ensuremath{\mathcal{P}}}

\DeclareMathOperator{\reachOp}{E}
\newcommand{\E}[1]{\reachOp(#1)}


\newcommand{\rowvec}[2]{\begin{pmatrix}
#1&#2\\
\end{pmatrix}}

\newcommand{\colvec}[2]{\begin{pmatrix}
#1\\
#2
\end{pmatrix}}

\newcommand{\deactivateGlossaries}
{
    \renewcommand{\makenoidxglossaries}{}
    \renewcommand{\printnoidxglossaries}{}
}

%\deactivateGlossaries

\makenoidxglossaries

\usepackage{listings}
\lstset{basicstyle=\footnotesize, captionpos=b, breaklines=true, showstringspaces=false, tabsize=2, frame=lines, numbers=left, numberstyle=\tiny, xleftmargin=2em, framexleftmargin=2em}
% \makeatletter
% \def\l@lstlisting#1#2{\@dottedtocline{1}{0em}{1em}{\hspace{1,5em} Lst. #1}{#2}}
% \makeatother

\definecolor{javared}{rgb}{0.6,0,0} % for strings
\definecolor{javagreen}{rgb}{0.25,0.5,0.35} % comments
\definecolor{javapurple}{rgb}{0.5,0,0.35} % keywords
\definecolor{javadocblue}{rgb}{0.25,0.35,0.75} % javadoc
\definecolor{gray}{rgb}{0.6,0.6,0.6}
 
\lstset{language=Java,
basicstyle=\ttfamily\footnotesize,
keywordstyle=\color{javapurple}\bfseries,
stringstyle=\color{javared},
commentstyle=\color{javagreen}\itshape\bfseries,
morecomment=[s][\color{javadocblue}]{/**}{*/},
numbers=left,
numberstyle=\tiny\color{gray},
stepnumber=1,
numbersep=10pt,
tabsize=3,
showspaces=false,
showstringspaces=false}

\newcommand{\code}[1]{\lstinline[basicstyle=\ttfamily\normalsize,
keywordstyle=\ttfamily\normalsize]!#1!}
\newcommand{\class}[1]{\code{#1}}
\newcommand{\const}[1]{\code{#1}}

\lstset{escapeinside={(*}{*)}}

\newcommand{\LSset}[2]{\scriptsize $\begin{aligned}&\{#1\}_L\\&\{#2\}_S\end{aligned}$}


\tikzset{
    transV/.style={transition, fill=black, minimum height = 12mm, minimum width = 1.5mm,inner sep = 0mm},
    transH/.style={transition, fill=black, minimum width = 12mm, minimum height = 1.5mm,inner sep = 0mm},
    node distance=1.5
}

\newcommand{\todo}[1]{\fxnote{{\color{red}#1}}}
\newcommand{\TODO}[2]{\fxnote*{{\color{red}#1}}{\underline{\emph{#2}}}}

\newlength{\wrapfigwidth}

\begin{document}
\titlehead{\includegraphics[height=2cm]{img/IM_LOGO.pdf}}

\subject{Masterarbeit}

\title{Konzeption und Integration einer Multithreading- und einer Test-API für eine 3D-Spielbibliothek sowie Analyse ihres Einflusses auf die Performance}

\subtitle{}

\author{
Florian Loher \textit{Technical University of Applied
Science Regensburg} \\
florian.loher@st.oth-regensburg.de
}

\date{30. Februar 2022}

\publishers{
    \setlength{\extrarowheight}{.3ex}
    \noindent\begin{tabular}{@{}ll}
        Fakultät: & Informatik und Mathematik\\
        \TODO{Fach}: & Informatik\\
        Abgabe: & \today\\
        Betreuer: & Prof.\ Dr.\ rer.\ nat.\ Carsten Kern
  \end{tabular}
}

\maketitle
\null\thispagestyle{empty}\clearpage
\tableofcontents
\chapter{Einleitung}
\section{blocklib}
\section{Nebenläufige Programmierung}
\section{Testen und Testframeworks}

\chapter{Analyse der blocklib}


\chapter{Entwicklung einer Threading API}

\section{Grundlagen zu Threading}
\subsection{Grundlegende Konzepte und Probleme bei Nebenläufigkeit}
Die Begriffe Nebenläufigkeit und Parallelität lassen sich leicht verwechseln. Es ist allerdings wichtig, deren Unterschiede zu kennen, um zu verstehen welches Thema genau behandelt wird und aus welchem Blickwinkel. Beide Konzepte beschäftigen sich mit der Ausführung von (beliebigen) Aufgaben. Wenn Aufgaben zeitlich unabhängig von einander ausgeführt werden können, bezeichnet man diese als nebenläufig. Parallelität bezeichnet die tatsächliche gleichzeitige Ausführung mehrerer Aufgaben. Als Spieleprogrammierer liegt der Fokus also eher auf Fragen bezüglich Nebenläufigkeit, da Parallelität plattformabhängig ist und meist durch Betriebssysteme abstrahiert wird. Parallelität bezieht also auf die Frage: \enquote{Wie kann ich verschiedene Aufgaben gleichzeitig durchführen?}, und Nebenläufigkeit auf die Frage: \enquote{Wie muss ich mein Programm strukturieren, damit ich von Parallelität profitieren kann?}. 

Bei der Beschäftigung mit Nebenläufigkeit treten Probleme auf, die bei rein sequenziellen Aufgaben nicht auftreten können. Einige dieser Probleme werden im Folgenden beschrieben.

\paragraph{Race Conditions} Wenn Aufgaben auf geteilte Ressource zugreifen, sind sie nicht mehr ohne Weiteres nebenläufig. Greift mindestens eine Aufgabe auf die Ressource in verändernder Weise zu, kann das Verhalten der Aufgaben nicht mehr vorhergesagt werden. Das Ergebnis hängt von der genauen Ausführreihenfolge ab. Diese ist einem Programm nicht zugänglich und kann sich bei jeder Ausführung unterscheiden. Solche Situationen, in denen die Ausführreihenfolge Einfluss auf das Ergebnis des Programmlaufs hat, werden Race Conditions genannt.

Um korrekte Nebenläufigkeit zu garantieren, müssen diese verhindert werden.
\subsection{Threading in Java}
\section{Anforderungen der blocklib an eine Threading API}
\subsection{Kontrolle über Threads an einer Stelle}
\subsection{Verteilung von Berechnungen an Worker Threads}
\subsection{Verhinderung von Livenessproblemen durch Archtektur}
\subsection{Einfaches starten von Hintergrundtasks und einmaligen Aufgaben}
\section{Design der Threading API}
\section{Implementierung der Threading API}
\section{Integration und Performanceanalyse}

\chapter{Entwicklung einer Test API}
\section{Grundlagen zu Testframeworks}
\subsection{Testen mittels OpenGL}
\subsection{JUnit}
\section{Anforderungen der blocklib an eine Test API}
\subsection{Verfügbarkeit von sinnvollen Highlevel Testfunktionen}
\subsection{Automatisierung der Testung (CI?)}
\subsection{Gute Dokumentation der Testmethodik}
\section{Design der Threading API}
\section{Implementierung der Threading API}
\section{Integration und Performanceanalyse}

\chapter
{Fazit und Ausblick}

\end{document}