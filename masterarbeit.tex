% !TeX spellcheck = de_DE
% !LuaLaTeX
\documentclass[12pt,a4paper,listof=totocnumbered,parskip=half]{scrreprt}
\usepackage{scrhack}
\usepackage[ngerman]{babel}
\usepackage[T1]{fontenc}
\usepackage[utf8]{inputenc}
\usepackage{lmodern}
\usepackage{microtype}
\usepackage{geometry}
\geometry{a4paper, top=27mm, left=20mm, right=20mm, bottom=35mm, headsep=10mm, footskip=12mm}

\usepackage{amsmath,amssymb,amsthm,mathrsfs,amsfonts}

\usepackage{csquotes}
\usepackage{booktabs}

\usepackage{graphicx}
\graphicspath{ {./img/} }
\usepackage{wrapfig}
% \usepackage{layouts}
\usepackage{textcomp} 
\usepackage[pdftex,dvipsnames]{xcolor}  % Coloured text etc.
\usepackage{svg}
\svgsetup{inkscapelatex=false, inkscapearea=drawing}
\usepackage[layout={margin,index},draft]{fixme}
\usepackage{rotating}
\usepackage{siunitx}

\usepackage[backend=biber,minbibnames=2,maxbibnames=2,maxcitenames=1,mincitenames=1,style=alphabetic]{biblatex}
\setcounter{biburlnumpenalty}{9000}
\setcounter{biburlucpenalty}{9000}
\setcounter{biburllcpenalty}{9000}

% new stretchable space between characters
\setlength{\biburlnumskip}{0mu plus 1mu}
\setlength{\biburlucskip}{0mu plus 1mu}
\setlength{\biburllcskip}{0mu plus 1mu}
\renewcommand{\namelabeldelim}{\addnbspace}

\addbibresource{Recherche/My Collection.bib}
\defbibheading{Literatur}{\chapter{Literaturverzeichnis}} 
\defbibheading{Quellen}{\section*{Quellenverzeichnis}} 

\DeclareSourcemap{
  \maps{
    \map{
      \step[fieldsource=url,
            match=\regexp{\$\\sim\$},
            replace=\regexp{\~}]
    }
  }
}

\usepackage{tikz}
\usetikzlibrary{calc,positioning, shapes, petri, automata}
\usepackage{pdfpages}
\usepackage{subcaption}
\usepackage{standalone}
\usepackage{multirow,tabularx}
\usepackage[hidelinks]{hyperref}
\usepackage[acronym,shortcuts,toc]{glossaries}
\usepackage{mathtools}


%\usepackage{pgfplots}
%\pgfplotsset{width=\columnwidth,compat=1.14}\usepgfplotslibrary{statistics}
%\pgfplotsset{boxplot/.cd,every median/.style={red}}
%\pgfplotsset{grid style={help lines}}
%\pgfplotsset{minor grid style={very thin, dotted}}
%\pgfplotsset{major grid style={thick}}

\newcommand{\unsim}{\mathord{\sim}}
\newcommand{\e}{\ensuremath{\mathrm{e}}}

\newcommand{\R}{\ensuremath{\mathbb{R}}}
\newcommand{\N}{\ensuremath{\mathbb{N}}}
\newcommand{\F}{\ensuremath{\mathbb{F}}}
\newcommand{\Pot}{\ensuremath{\mathcal{P}}}

\DeclareMathOperator{\reachOp}{E}
\newcommand{\E}[1]{\reachOp(#1)}


\newcommand{\rowvec}[2]{\begin{pmatrix}
#1&#2\\
\end{pmatrix}}

\newcommand{\colvec}[2]{\begin{pmatrix}
#1\\
#2
\end{pmatrix}}

\newcommand{\deactivateGlossaries}
{
    \renewcommand{\makenoidxglossaries}{}
    \renewcommand{\printnoidxglossaries}{}
}

%\deactivateGlossaries

\makenoidxglossaries

\usepackage{listings}
\lstset{basicstyle=\footnotesize, captionpos=b, breaklines=true, showstringspaces=false, tabsize=2, frame=lines, numbers=left, numberstyle=\tiny, xleftmargin=2em, framexleftmargin=2em}
% \makeatletter
% \def\l@lstlisting#1#2{\@dottedtocline{1}{0em}{1em}{\hspace{1,5em} Lst. #1}{#2}}
% \makeatother

\definecolor{javared}{rgb}{0.6,0,0} % for strings
\definecolor{javagreen}{rgb}{0.25,0.5,0.35} % comments
\definecolor{javapurple}{rgb}{0.5,0,0.35} % keywords
\definecolor{javadocblue}{rgb}{0.25,0.35,0.75} % javadoc
\definecolor{gray}{rgb}{0.6,0.6,0.6}
 
\lstset{language=Java,
basicstyle=\ttfamily\footnotesize,
keywordstyle=\color{javapurple}\bfseries,
stringstyle=\color{javared},
commentstyle=\color{javagreen}\itshape\bfseries,
morecomment=[s][\color{javadocblue}]{/**}{*/},
numbers=left,
numberstyle=\tiny\color{gray},
stepnumber=1,
numbersep=10pt,
tabsize=3,
showspaces=false,
showstringspaces=false}

\newcommand{\code}[1]{\lstinline[basicstyle=\ttfamily\normalsize,
keywordstyle=\ttfamily\normalsize]!#1!}
\newcommand{\class}[1]{\code{#1}}
\newcommand{\const}[1]{\code{#1}}

\lstset{escapeinside={(*}{*)}}

\newcommand{\LSset}[2]{\scriptsize $\begin{aligned}&\{#1\}_L\\&\{#2\}_S\end{aligned}$}


\tikzset{
    transV/.style={transition, fill=black, minimum height = 12mm, minimum width = 1.5mm,inner sep = 0mm},
    transH/.style={transition, fill=black, minimum width = 12mm, minimum height = 1.5mm,inner sep = 0mm},
    node distance=1.5
}

\newcommand{\todo}[1]{\fxnote{{\color{red}#1}}}
\newcommand{\TODO}[2]{\fxnote*{{\color{red}#1}}{\underline{\emph{#2}}}}

\newlength{\wrapfigwidth}
\addbibresource{Recherche/My Collection.bib}
\defbibheading{Literatur}{\chapter{Literaturverzeichnis}} 
\defbibheading{Quellen}{\section*{Quellenverzeichnis}} 
\newglossaryentry{nichtAtomar}{
name={nicht-atomar},
plural={nicht-atomare},
user1={nicht-atomaren},
description={Beschreibung einer Anweisung, die selbst aus weiteren Anweisungen besteht}
}
\newglossaryentry{compositionRoot}{
name={Composition Root},
description={Position in einer Anwendung an der Module zusammengesetzt werden}
}
\newglossaryentry{dependencyInjection}{
name={Dependency Injection},
description={Designarchitektur, in der Module ihre Abhängigkeiten nicht selbst erzeugen, sondern diese Aufgabe an aufrufende Module übertragen}
}
\newglossaryentry{servicelocator}{
name={Service Locator},
description={Designpattern zur dynamischen Lokalisierung von Abhängigkeiten über eine zentrales Modul}
}
\newglossaryentry{singleton}{
name={Singleton},
description={Designpattern, mit dem ein Objekt global verfügbar gemacht und dessen mehrfache Instanziierung verhindert wird}
}
\newglossaryentry{testdouble}{
name={Test Double},
description={spezielles Modul, das das Verhalten einer Komponente vorspielt, damit das Original beim Testen nicht benötigt wird}
}

\newacronym[]{fps}{FPS}{Frames per Second (dt. Bilder pro Sekunde)}
\DeclareSIUnit{\fps}{\ac{fps}}
\newacronym[]{sot}{SoT}{System on a Thread}
\newacronym[]{cpu}{CPU}{Prozessor (Central Processing Unit)}
\newacronym[user1={Grafikkarte}]{gpu}{GPU}{Grafikkarte (Graphics Processing Unit)}
\newacronym[]{ram}{RAM}{Hauptspeicher (Random Access Memory)}
\begin{document}
\titlehead{\includegraphics[height=2cm]{img/IM_LOGO.pdf}}

\subject{Masterarbeit}

\title{Konzeption und Integration einer Multithreading- und einer Test-API für eine 3D-Spielbibliothek sowie Analyse ihres Einflusses auf die Performance}

\subtitle{}

\author{
Florian Loher \textit{Technical University of Applied
Science Regensburg} \\
florian.loher@st.oth-regensburg.de
}

\date{30. Februar 2022}

\publishers{
    \setlength{\extrarowheight}{.3ex}
    \noindent\begin{tabular}{@{}ll}
        Fakultät: & Informatik und Mathematik\\
        \TODO{Fach}: & Informatik\\
        Abgabe: & \today\\
        Betreuer: & Prof.\ Dr.\ rer.\ nat.\ Carsten Kern
  \end{tabular}
}

\maketitle
\null\thispagestyle{empty}\clearpage
\tableofcontents
\chapter{Einleitung}
\section{blocklib}
\section{Nebenläufige Programmierung}
\section{Testen und Testframeworks}

\chapter{Grundlagen}
Es gibt einige Ansätze zur Modellierung verteilter oder nebenläufiger \glsplural{Programm}. Darunter finden sich auch Petri-Netze~\cite{Murata1989}. Mit einem Petri-Netz kann ein modelliertes \glstext{Programm} mathematisch formal definiert und analysiert werden. Der Begriff Petri-Netz beschreibt eine Familie von verwandten Modellen. Das hier im Folgenden beschriebene Petri-Netz-Modell wird auch Platz-Transitions-Netz (PT-Netz) genannt. Formal lässt sich ein PT-Netz $N$ als 5-Tupel $ N=(P,T,F,W,m_0)$ beschreiben. Dabei ist 
\begin{align*}
	&P  \quad \text{eine endliche Menge von sogenannten \emph{Plätzen},}\\
	&T  \quad \text{eine endliche Menge von sogenannten \emph{Transitionen},}\\
	&F \subseteq (P\times T) \cup (T \times P) \quad \text{die Menge der Relationen zwischen Plätzen und Transitionen,}\\
	&W: F \mapsto \N  \quad \text{eine Gewichtungsfunktion der Relationen und}\\
	&m_0: P \mapsto \N_0   \quad \text{die anfängliche \emph{Markierung} der Plätze~\cite{Murata1989}.}
\end{align*}
Der durch ein PT-Netz beschriebene Graph kann auch grafisch dargestellt werden. Die Plätze $p \in P$ werden dabei als Kreise dargestellt, die Transitionen $ t \in T$ durch schwarz gefüllte Rechtecke, die Relationen $ f \in F$ durch gerichtete Kanten zwischen den Kreisen und Rechtecken, wobei die Elemente der Gewichtungsfunktion $W$ an die jeweiligen Kanten gesetzt werden. Die Markierung wird durch schwarze Punkte, die \emph{Marken} genannt werden, in den Plätzen dargestellt. Abbildung~\ref{fig:petrinet} zeigt ein simples Beispiel eines PT-Netzes, sowohl formal beschrieben (Abbildung~\ref{fig:petrinet:formal}) als auch seine grafische Repräsentation (Abbildung~\ref{fig:petrinet:graph}).
\begin{figure}
	\centering
	\begin{subfigure}[b]{.48\textwidth}
		$\begin{aligned}
			N &= (P,T,F,W,m_0)\\
			P &= \{p_1, p_2\}\\
			T &= \{t_1, t_2\}\\
			F &= \{(p_1, t_1), (p_2, t_2), (t_1, p_2)\}\\
			W &= \{((p_1, t_1),1), ((p_2, t_2),2), ((t_1, p_2), 1)\}\\
			m_0 &= \{(p_1, 1), (p_2, 0)\}
		\end{aligned}$
		\subcaption{Formale Definition des PT-Netzes.}\label{fig:petrinet:formal}
	\end{subfigure}
	\hfill
	\begin{subfigure}[b]{.48\textwidth}
		\begin{tikzpicture}
			\node[place, label=$p_1$, tokens=1] (p1) at (0,0) {};
			\node[transV,label=$t_1$, right = of p1] (t1){};
			\node[place, label=$p_2$, right = of t1] (p2) {};
			\node[transV,label=$t_2$, right = of p2] (t2){};

			\draw 
			(p1) edge[post] node[above] {1} (t1)
			(t1) edge[post] node[above] {2} node[below=1cm] {$N$} (p2)
			(p2) edge[post] node[above] {1} (t2);
		\end{tikzpicture}
		\subcaption{Graph-Repräsentation des PT-Netzes.}\label{fig:petrinet:graph}
	\end{subfigure}
	\caption{Beispiel eines PT-Netzes mit zwei Plätzen und zwei Transitionen.}\label{fig:petrinet}
\end{figure}

\subsubsection{Transitionsregel} Um das Verhalten von \glsuserii{Programm} beschreiben zu können, kann die Markierung eines Petri-Netzes anhand der folgenden sogenannten \emph{Transitions-Regel} geändert werden~\cite{Murata1989}.
\begin{enumerate}
	\item Eine Transition $t_i \in T$ heißt \emph{aktiviert}, wenn für alle $(p,t_i) \in F $ gilt: $ W((p,t_i)) \leq M(p)$ wobei $M$ die aktuelle Markierung des Petri-Netzes ist. Es müssen also in allen Plätzen mit eingehenden Kanten genügend Marken bezüglich der Gewichtungsfunktion existieren. Diese Plätze werden auch \emph{Vorbedingungen} für die Transition genannt. Wenn ein Platz genügend Marken für eine Transition enthält, nennt man die Vorbedingungen (bezüglich dieser Transition) \emph{erfüllt}.
	\item Eine aktivierte Transition kann, muss aber nicht \emph{feuern}.
	\item Feuert eine aktivierte Transition $t_i$, geht eine Markierung $M$ in eine Markierung $M'$ über. Dies erfolgt nach den folgenden Regeln.
	\begin{enumerate}
		\item $P$ wird in die disjunkten Mengen $P_\emptyset, P_\rightarrow, P_\leftarrow, P_\leftrightarrow$ unterteilt, wobei 
		\item $P_\emptyset = \{p \in P | (p,t_i) \notin F \land (t_i,p) \notin F\}$ die unbeteiligten Plätze sind,
		\item $P_\rightarrow = \{p \in P | (p,t_i) \in F \land (t_i,p) \notin F\}$ die Plätze mit eingehenden Kanten zur Transition,
		\item $P_\leftarrow = \{p \in P | (p,t_i) \notin F \land (t_i,p) \in F\}$ die Plätze mit ausgehenden Kanten aus der Transition und 
		\item $P_\leftrightarrow = \{p \in P | (p,t_i) \in F \land (t_i,p) \in F\}$ die Plätze mit eingehenden und ausgehenden Kanten.
		\item Dann gilt $
			M'(p) = \left\{ 
				\begin{aligned}
					& M(p) && \; , \; p \in P_\emptyset\\
					& M(p)-W((p,t_i)) && \; , \; p \in P_\rightarrow\\
					&M(p)+W((t_i,p)) && \; , \; p \in P_\leftarrow\\
					& M(p)-W((p,t_i))+W((t_i,p)) && \; , \; p \in P_\leftrightarrow
				\end{aligned}
				\right\}
		$
	\end{enumerate}
\end{enumerate}
Betrachtet man Abbildung~\ref{fig:petrinet} erneut, lässt sich erkennen, dass Transition $t_1$ aktiviert ist und Transition $t_2$ nicht, da sich in $p_2$ keine Marke befindet.

\subsubsection{Erreichbarkeitsgraph}
Mittels der Transitionsregel lassen sich in einem Petri-Netz $N$ ausgehend von der Anfangsmarkierung $m_0$ gegebenenfalls weitere Markierungen erzeugen. Durch die wiederholte Anwendung der Transitionsregel auf alle daraus entstehenden Markierungen lässt sich ein Graph erstellen, der alle von der Anfangsmarkierung aus erreichbaren Markierungen erhält. Dieser wird der \emph{Erreichbarkeitsgraph} $\E{N}$ des Petri-Netzes genannt.
\begin{figure}
	\begin{tikzpicture}[node distance=5mm,auto,state/.append style={rounded rectangle}]
		\node[state, initial] (q1) {$\{(p_1,1),(p_2,0)\}$};
		\node[state] (q2) [right=of q1] {$\{(p_1,0),(p_2,2)\}$};
		\node[state] (q3) [right=of q2] {$\{(p_1,0),(p_2,1)\}$};
		\node[state] (q4) [right=of q3] {$\{(p_1,0),(p_2,0)\}$};
		\path[->] 
		(q1) edge node {$t_1$} (q2)
		(q2) edge node {$t_2$} node[below=6mm] {Graph mit vollständigen Knotenbezeichnungen}(q3)
		(q3) edge node {$t_2$} (q4);

		\node[state,initial] (q5) [below=1.2cm of q1]{$(1,0)$};
		\node[state] (q6) [below=1.2cm of q2] {$(0,2)$};
		\node[state] (q7) [below=1.2cm of q3] {$(0,1)$};
		\node[state] (q8) [below=1.2cm of q4] {$(0,0)$};
		\path[->] 
		(q5) edge node {$t_1$} (q6)
		(q6) edge node {$t_2$} node[below=6mm] {Graph mit gekürzten Knotenbezeichnungen}(q7)
		(q7) edge node {$t_2$} (q8);
		\end{tikzpicture}
		\caption[Erreichbarkeitsgraph des Beispiel PT-Netzes.]{Erreichbarkeitsgraph des Petri-Netzes aus Abbildung~\ref{fig:petrinet}. Der Graph ist zweimal dargestellt. Oben enthalten die Knoten des Graphen die vollständige Auszeichnung der Markierung, unten wird die Kurzschreibweise für die Knotenbezeichnungen genutzt, in der die Position den Platz kodiert. Der \enquote{start} Pfeil kennzeichnet die anfängliche Markierung.}\label{fig:reachability}
\end{figure}
In Abbildung~\ref{fig:reachability} wird der Erreichbarkeitsgraph des in Abbildung~\ref{fig:petrinet} dargestellten Petri-Netzes gezeigt. Die Knoten des Graphen sind die Markierungen des Petri-Netzes, die durch die Transitionsregel erreicht werden können. Da die Plätze des Petri-Netzes nummeriert sind, kann die Markierung verkürzt geschrieben werden, indem der Index des Tupels der Knotenbezeichnung den Platz beschreibt und die Zahl die Anzahl der Marken des Platzes kennzeichnet. Der obere Bereich der Abbildung stellt den Graph mit den vollständigen Markierungsbezeichnungen dar. Darunter wird die verkürzte Schreibweise genutzt.

Ist die Anzahl der Marken pro Platz in jeder Markierung des Erreichbarkeitsgraphen maximal $1$, kann die Schreibweise weiter verkürzt werden, indem nur die Plätze, genauer gesagt die Indizes der Plätze, die eine Marke enthalten, genannt werden. Abbildung~\ref{fig:1-bpetrinet} zeigt eine Abwandlung des Petri-Netzes aus Abbildung~\ref{fig:petrinet}, das die beschriebene Eigenschaft erfüllt, sowie den zugehörigen Erreichbarkeitsgraphen in verkürzter Schreibweise.
\begin{figure}
	\centering
	\begin{subfigure}[b]{.48\textwidth}
		\centering
		\begin{tikzpicture}
			\node[place, label=$p_1$, tokens=1] (p1) at (0,0) {};
			\node[transV,label=$t_1$, right = of p1] (t1){};
			\node[place, label=$p_2$, right = of t1] (p2) {};
			\node[transV,label=$t_2$, right = of p2] (t2){};

			\draw 
			(p1) edge[post] node[above] {1} (t1)
			(t1) edge[post] node[above] {1} (p2)
			(p2) edge[post] node[above] {1} (t2);
		\end{tikzpicture}
		\subcaption{Abwandlung des Petri-Netzes aus Abbildung~\ref{fig:petrinet}. Die Gewichtungsfunktion ist für jede Relation maximal 1.}\label{fig:1-bpetrinet:graph}
	\end{subfigure}
	\hfill
	\begin{subfigure}[b]{.48\textwidth}
		\centering
		\begin{tikzpicture}[node distance=5mm,auto,state/.append style={rounded rectangle}]
			\node[state, initial] (q1) {$1$};
			\node[state] (q2) [right=of q1] {$2$};
			\node[state] (q3) [right=of q2] {};
			\path[->] 
			(q1) edge node {$t_1$} (q2)
			(q2) edge node {$t_2$} (q3);
		\end{tikzpicture}
		\subcaption{Erreichbarkeitsgraph, bei dem die Zahl(en) in den Knoten die  Indizes der Plätze beschreiben, die eine Markierung besitzen.}\label{fig:1-bpetrinet:reachability}
	\end{subfigure}
\caption{Petri-Netz und zugehöriger Erreichbarkeitsgraph, in dem die Anzahl der Markierungen pro Platz stets maximal 1 ist. }\label{fig:1-bpetrinet}
\end{figure} 
\subsubsection{Weitere Definitionen}
Um die Arbeit mit Petri-Netzen zu vereinfachen, werden nun noch einige weitere Definitionen eingeführt. Es sei ein Petri-Netz $N = (P,T,F,W,m_0)$ gegeben.

\begin{enumerate}
	\item Die Menge der Knoten $v$ mit Kanten zu einem Knoten $k \in P\cup T$ heißt \emph{Vorbereich} des Knotens $k$ und ist definiert als $^\circ k = \{v | (v,k) \in F\}$.
	\item Die Menge der Knoten $n$ mit Kanten von einem Knoten $k \in P\cup T$ heißt \emph{Nachbereich} des Knotens $k$ und ist definiert als $k^\circ  = \{n | (k,n) \in F\}$.
	\item Die Menge der Knoten einer Markierung $m$ mit mindesten $n$ Marken $\{x | m(x)\geq n\}$ wird $M_{\geq n}$ genannt.
	\item Das Maximum einer Markierung $m$, $\deg(m) \coloneqq \max \{m(p)|p\in P\}$, wird ihr \emph{Grad} genannt.
	\item Ein Petri-Netz $N$ heißt \emph{$n$-beschränkt}, wenn $n$ das Maximum der Grade der Markierungen des Erreichbarkeitsgraphen $(V,K)=\E{N}$ ist, also $n = \max\{\deg(m)| m\in V \}$.
	\item Gilt für eine Kante $((a, b),w) \in W$, dass $w = 1$, so kann im dazugehörigen Graphen die Beschriftung entfallen. Kanten ohne Beschriftung haben also ein implizites Gewicht von $1$.
\end{enumerate}



\subsubsection{Erweiterte Petri-Netze}
Um das Konzept von Variablenzugriffen einfacher zu modellieren, führen \textcite{Goel1990} ein erweitertes Petri-Netz-Modell ein. Dabei wird das 5-Tupel des Petri-Netzes um die folgenden Komponenten erweitert:
\begin{enumerate}
	\item eine Menge von Variablen $V$,
	\item eine Funktion $L: T \mapsto \Pot(V)$, die \emph{lesenden Zugriff} auf die Variablen modelliert, und
	\item eine Funktion $S: T \mapsto \Pot(V)$, die \emph{schreibenden Zugriff} auf die Variablen modelliert.
\end{enumerate}
$\Pot(V)$ ist dabei die Potenzmenge von $V$, also die Menge aller Teilmengen von $V$.
Ein Beispiel für ein erweitertes Petri-Netz, das ansonsten identisch zu dem Petri-Netz in Abbildung~\ref{fig:petrinet} ist, ist in Abbildung~\ref{fig:augpetrinet} gegeben. Die formale Definition in Abbildung~\ref{fig:augpetrinet:formal} zeigt die neuen Mengen $V$, $L$ und $S$. Der Graph in Abbildung~\ref{fig:augpetrinet:graph} zeigt die Lese- und Schreibmengen unter Transition $t_1$. Die Lesemenge wird durch ein $L$ gekennzeichnet, die Schreibmenge durch ein $S$. 
\begin{figure}
\centering
	\begin{subfigure}[b]{.49\textwidth}
		$\begin{aligned}
			N &= (P,T,F,W,m_0, V, L, S)\\
			P &= \{p_1, p_2\}\\
			T &= \{t_1, t_2\}\\
			F &= \{(p_1, t_1), (p_2, t_2), (t_1, p_2)\}\\
			W &= \{((p_1, t_1),1), ((p_2, t_2),2), ((t_1, p_2), 1)\}\\
			m_0 &= \{(p_1, 1), (p_2, 0)\}\\
			V &= \{a,b\}\\
			L &= \{(t_1,\{a,b\}), (t_2,\emptyset)\}\\
			S &= \{(t_1,\{a\}), (t_2,\emptyset)\}
		\end{aligned}$
		\caption{Formale Definition eines erweiterten Petri-Netzes.}\label{fig:augpetrinet:formal}
	\end{subfigure}
	\hfill
	\begin{subfigure}[b]{.49\textwidth}
		\begin{tikzpicture}
			\node[place, label=$p_1$, tokens=1] (p1) at (0,0) {};
			\node[transV,label=$t_1$, right = of p1, label=below:{\LSset{a,b}{a}}] (t1){};
			\node[place, label=$p_2$, right = of t1] (p2) {};
			\node[transV,label=$t_2$, right = of p2] (t2){};

			\draw 
			(p1) edge[post] node[above] {1} (t1)
			(t1) edge[post] node[above] {2} node[below=2cm] {$N$} (p2)
			(p2) edge[post] node[above] {1} (t2);
		\end{tikzpicture}
		\caption{Graph-Repräsentation des erweiterten Petri-Netzes.}\label{fig:augpetrinet:graph}
	\end{subfigure}
	\caption[Beispiel eines erweiterten Petri-Netzes.]{Beispiel eines erweiterten Petri-Netzes. In (a) ist die formale Definition gegeben, in (b) der dazugehörige Graph. In Platz $p_2$ wird auf die Variablen $a$ und $b$ lesend und auf Variable $a$ schreibend zugegriffen.}\label{fig:augpetrinet}
\end{figure}
\section{Grundlagen zu Threading}


\subsection{Definition von Thread}
Um definieren zu können, was ein Thread ist, müssen zuerst einige weitere Grundbegriffe eingeführt werden. 

Als \emph{Programm} wird in dieser Arbeit die konkrete Niederschrift eines Algorithmus bezeichnet. Die Elemente, aus denen das Programm besteht, werden als \emph{Anweisungen} bezeichnet. Bei der Ausführung eines Programms werden Einzelschritte durchlaufen, diese bezeichnet man als \emph{Aktivitäten}. Eine Sequenz von Aktivitäten, die ein isoliertes Problem abarbeitet, wird \emph{Prozess} genannt. Um diese Definition des Prozessbegriffs von späteren Definitionen abzugrenzen wird er im Folgenden \emph{Rechenprozess} genannt. Die Ausführungseinheit, auf der die Schritte eines Rechenprozesses durchgeführt werden, wird \emph{Prozessor} genannt.\cite{Herrtwich1989}

Wenn ein Programm auf einem Betriebssystem ausgeführt wird, erzeugt das Betriebssystem einen isolierten Adressraum, in dem das Programm ausgeführt wird. Ein Rechenprozess, der auf diese Weise ausgeführt wird, wird in dieser Arbeit als \emph{(System-)Prozess} bezeichnet. \textcite[Kapitel~2]{Tanenbaum2016} liefern eine gute Übersicht über Systemprozesse, dort als Prozesse bezeichnet. Systemprozesse können selbst weitere Systemprozesse starten, wenn das Betriebssystem dies zulässt. Diese Systemprozesse besitzen dann ihren eigenen Adressraum. Die Anzahl der Systemprozesse, die auf einem Rechner laufen, ist meist höher als die Anzahl der Prozessoren des Rechners. Somit können nicht alle Prozesse zur gleichen Zeit laufen. Damit dennoch alle Prozesse voranschreiten können, wechseln moderne Betriebssysteme die Systemprozesse, die auf den Prozessoren des Rechners ausgeführt werden, in schneller Folge. Dabei muss das Betriebssystem einige zu den Systemprozessen gehörende Daten, den \emph{Prozesskontrollblock}, tauschen, sodass der jeweils gerade ausführende Systemprozess seinen eigenen Prozesskontrollblock zur Verfügung hat. Dieser Tausch wird \emph{Kontextwechsel} genannt.

Das Speichern und Laden der Prozesskontrollblöcken im Rahmen der Kontextwechsel von Systemprozessen benötigt eine nicht zu vernachlässigende Menge an Zeit. Ein solcher Zeitverbrauch wird auch als \emph{Overhead} bezeichnet. Um den Overhead von Kontextwechseln zu vermindern, bieten moderne Betriebssysteme Systemprozessen die Möglichkeit Rechenprozesse zu erzeugen, die denselben Adressraum wie der erzeugende Systemprozess besitzen. Ein auf diese Art erzeugter Rechenprozess heißt \emph{Thread}. Da die Menge der thread-eigenen Daten, des \emph{Threadkontrollblocks}, deutlich geringer ist als die des Prozesskontrollblocks, erzeugt der Kontextwechsel zwischen Threads einen geringeren Overhead. Zudem können Threads aufgrund des gemeinsamen Adressraums einfacher auf geteilte Ressourcen zugreifen. Das vereinfacht die Kooperation zwischen diesen Rechenprozessen.\cite[Kapitel~2.2]{Tanenbaum2016}

\paragraph{Anweisungen und Petri-Netze}
Um formal Eigenschaften eines Programms zu beschreiben, ist es möglich dieses mittels eines Petri-Netzes (automatisiert) zu modellieren. Dazu müssen Petri-Netz-Konstrukte genutzt werden, die die Anweisungen des Programms abbilden können. Für die Analyse nebenläufiger Programme sind nur sogenannte \emph{Rendevous}-Anweisungen sowie Kontrollstrukturen relevant~\cite{Goel1990}. Für diese lassen sich Unter-Petri-Netze definieren, die deren Verhalten modelieren. Da die Modellierung von Verzweigungen durch Unter-Petri-Netze in \cite*[Abbildung 3.1]{Goel1990} semantisch nicht korrekt ist, wird sie in Abbildung~\ref{fig:supnetifelse} korrekt dargestellt. 
\begin{figure}
	\centering
	\begin{tikzpicture}[node distance=2cm,on grid, auto]
		\node[place, label=$p_1$] (p1) {};
		\node[transV, label=\texttt{if}, above right = of p1] (if) {};
		\node[transV, label=\texttt{else},below right = of p1] (else) {};
		\node[right = of if] (dots1){$\dots$};
		\node[right = of else] (dots2){$\dots$};
		\node[place, label=$p_2$, right = of dots1] (p2) {};
		\node[place, label=$p_3$, right = of dots2] (p3) {};
		\node[transV, label=$t_1$, right = of p2] (t1) {};
		\node[transV, label=$t_2$, right = of p3] (t2) {};
		\node[place, label=$p_4$, below right = of t1] (p4) {};
		\node[transV, label=\texttt{end if}, right = of p4] (endif) {};
	
	
	
		\draw 
		(p1) edge[post] (if)
		(p1) edge[post] (else)
		(if) edge[post] (dots1)
		(else) edge[post] (dots2)
		(dots1) edge[post] (p2) 
		(dots2) edge[post] (p3) 
		(p2) edge[post] (t1)
		(p3) edge[post] (t2)
		(t1) edge[post] (p4)
		(t2) edge[post] (p4)
		(p4) edge[post] (endif)
		;
	\end{tikzpicture}
	\caption{}\label{fig:supnetifelse}
\end{figure}

Die Plätze $p_1,p_2,p_3,p_4$ und die Transitionen $t_1,t_2$ sind Hilfselemente die die semantische Korrektheit der Transitionen \code{if}, \code{else} und \code{end if} sicherstellen. Eine Markierung kann exlusiv nur von \code{if} oder von \code{else} zum feuern verwendet werden. Durch die Transitionen $t_1$ und $t_2$ entsteht eine Markierung in $p_5$ wodurch \code{end if} feuern kann egal ob anfangs \code{if} oder \code{else} gefeuert hat. $p_2$ und $p_3$ existierten, damit das Petri-Netz ein bipartiter Graph bleibt, sich also immer Plätze und Transitionen \enquote{abwechseln}. Die Modellierung eines \code{switch}-Statements erfolgt analog, indem die Anzahl der Pfade zwischen $p_1$ und $p_4$ angepasst wird.

\subsection{Nebenläufigkeit}\label{sec:nebenl}
Im vorherigen Abschnitt \TODO{hier ist Vergangenheit}{wurde beschrieben}, dass verschiedene Rechenprozesse unabhängig von einander \enquote{gleichzeitig} ablaufen können. Ein Programm wird häufig als \emph{Sequenz} von Anweisungen verstanden. In der Definition von Programm in dieser Arbeit wird bewusst auf das Wort Sequenz verzichtet, denn bei vielen Programmen kann es bei bestimmten Anweisungen irrelevant sein, in welcher Reihenfolge oder ob sie sogar gleichzeitig ausgeführt werden. Man betrachte das folgende Programm in Listing~\ref{lst:squareSeqEx}, das die Quadratsumme zweier Ganzzahlen (im Folgenden auch Quadratsumme genannt) ausgibt: 
\begin{lstlisting}[caption={Beispiel eines Programms das die Summe von Quadraten zweier Ganzzahlen berechnet. Die Berechnung der Quadratzahlen wird nacheinander in einer fest definierten Sequenz durchgeführt.}, label={lst:squareSeqEx}]
printSummedSquare(int a, int b){
  x <- a*a
  y <- b*b
  print(x+y)
}
\end{lstlisting}
Es ist leicht zu erkennen, dass es vollkommen egal ist, ob zuerst \code{x} oder \code{y} ausgerechnet wird oder die Berechnungen simultan stattfinden. Einzig der Ausgabebefehl in Zeile 4 muss ausgeführt werden, nachdem \code{x} und \code{y} berechnet wurden. Zuzulassen, dass die Ausführreihenfolge in dem Programm nicht definiert ist, bringt das Design des Programs näher an die Problembeschreibung \enquote{gib die Quadratsumme zweier Ganzzahlen aus} und ermöglicht auch eine potenziell schnellere Berechnung, da die Quadrate gleichzeitig berechnet werden können.

Paare oder Gruppen von Anweisungen, die (wie Zeile 3 und Zeile 3 in Listing~\ref{lst:squareConcEx}) gleichzeitig oder in beliebiger Reihenfolge ausgeführt werden können, heißen \emph{nebenläufig}. Eine Anweisung kann beispielsweise durch einen Compiler in mehrere Anweisungen geteilt werden oder schon aus mehreren Anweisungen bestehen (man denke zum Beispiel an Funktionsaufrufe). Solche Anweisungen werden \gls{na} genannt. Eine Menge von nebenläufigen Anweisungen, von denen mindestens eine \gls{na} ist, kann \emph{verzahnt} ausgeführt werden. Eine Ausführung von Anweisungen ist verzahnt, wenn zwischen der Ausführung der Teile einer \glsuseri{na} Anweisung andere Anweisungen ausgeführt werden. Abbildung \ref{fig:concAnweisungen} zeigt zur Veranschaulichung die verschiedenen Möglichkeiten, wie zwei nebenläufige \glspl{na} Anweisungen im Zeitverlauf ausgeführt werden können, sowohl auf einem Prozessor als auch auf zwei Prozessoren. 
\begin{figure}[hbt]
\newlength\aOne
\newlength\aTwo
\newlength\aThree
\newlength\bOne
\newlength\bTwo
\pgfmathsetlength{\aOne}{4cm}
\pgfmathsetlength{\aTwo}{2cm}
\pgfmathsetlength{\aThree}{3cm}
\pgfmathsetlength{\bOne}{2.5cm}
\pgfmathsetlength{\bTwo}{3.7cm}
\begin{subfigure}{\textwidth}
\begin{tikzpicture}

	\draw[->] (0,0) -- (15,0) node[right] {Zeit};
	
	\node[draw,fill=red, anchor=south west, minimum width = \aOne] at (0, .5) (A1) {A Teil1};
	\node[draw,fill=red, anchor=west, minimum width = \aTwo] at (A1.east) (A2) {A Teil2};
	\node[draw,fill=red, anchor=west, minimum width = \aThree] at (A2.east) (A3) {A Teil3};
	
	\node[draw,fill=cyan, anchor=south west, minimum width = \bOne] at (0, 1.5) (B1) {B Teil1};
	\node[draw,fill=cyan, anchor=west, minimum width = \bTwo] at (B1.east) (B2) {B Teil2};

\end{tikzpicture}
\subcaption{Gleichzeitige Ausführung der Anweisungen A und B auf zwei Prozessoren}
\end{subfigure}
\\[1.5em]
\begin{subfigure}{\textwidth}
\begin{tikzpicture}

	\draw[->] (0,0) -- (15,0) node[right] {Zeit};
	
	\node[draw,fill=cyan, anchor=south west, minimum width = \bOne] at (0, .5) (B1) {B Teil1};
	\node[draw,fill=cyan, anchor=west, minimum width = \bTwo] at (B1.east) (B2) {B Teil2};
	\node[draw,fill=red, anchor=west, minimum width = \aOne] at (B2.east) (A1) {A Teil1};
	\node[draw,fill=red, anchor=west, minimum width = \aTwo] at (A1.east) (A2) {A Teil2};
	\node[draw,fill=red, anchor=west, minimum width = \aThree] at (A2.east) (A3) {A Teil3};
\end{tikzpicture}
\subcaption{Sequenzielle Ausführung der Anweisungen A und B auf einem Prozessor}
\end{subfigure}
\\[1.5em]
\begin{subfigure}{\textwidth}
\begin{tikzpicture}

	\draw[->] (0,0) -- (15,0) node[right] {Zeit};
	
	\node[draw,fill=cyan, anchor=south west, minimum width = \bOne] at (0, .5) (B1) {B Teil1};
	\node[draw,fill=red, anchor=west, minimum width = \aOne] at (B1.east) (A1) {A Teil1};
	\node[draw,fill=red, anchor=west, minimum width = \aTwo] at (A1.east) (A2) {A Teil2};
	\node[draw,fill=cyan, anchor=west, minimum width = \bTwo] at (A2.east) (B2) {B Teil2};
	\node[draw,fill=red, anchor=west, minimum width = \aThree] at (B2.east) (A3) {A Teil3};

\end{tikzpicture}
\subcaption{Verzahnte Ausführung der Anweisungen A und B auf einem Prozessor}
\end{subfigure}

\caption{Mögliche Ausführungen nebenläufiger Anweisungen}\label{fig:concAnweisungen}
\end{figure}

Eine der obigen Problemstellung nähere Implementierung der Quadratsumme könnte unter Nutzung von Nebenläufigkeit wie Listing~\ref{lst:squareConcEx} aussehen, dabei wird auf die Schreibweise von \textcite{Herrtwich1989} zurückgegriffen:
\begin{lstlisting}[caption={Beispiel eines Programms mit nebenläufigem Code in einem \code{conc}-Block. Das Programm gibt die Summe von zwei Quadratzahlen aus, wobei die Berechnung der Quadratzahlen nebenläufig stattfindet.}, label={lst:squareConcEx}]
printSummedSquare(int a, int b){
  conc 
    x <- a*a ||
    y <- b*b
  conc end
  print(x+y)
}
\end{lstlisting}
In einem \code{conc}-Block, definiert durch \code{conc} und \code{conc end}, sind alle Anweisungen die durch \code{||} getrennt sind als nebenläufig zu verstehen. Wie man an obigem Beispiel erkennen kann, beschäftigt Nebenläufigkeit sich besonders mit der Struktur und der Möglichkeit der Zusammensetzung voneinander unabhängiger Anweisungen. Somit ist es Aufgabe der nebenläufigen Programmierung, Probleme oder Aufgaben in unabhängige Teile zu zerlegen und zu strukturieren~\cite{Pike2012,Hettel2016}. Blöcke von Anweisungen, die (bezüglich der Aufgabe) nebenläufig sein könnten, aber als Anweisungssequenz definiert sind (siehe Listing~\ref{lst:squareSeqEx} Zeile 2 und 3), werden in dieser Arbeit als \emph{pseudo-sequenzialisiert} bezeichnet.

\paragraph{Nebenläufigkeit und Petri-Netze}
Um ein eindeutiges Verständnis des hier genutzten Begriffs der Nebenläufigkeit zu geben, wird dieser auch im Bezug auf Petri-Netze definiert. Wichtig anzumerken ist, dass der hier beschriebene Begriff vom üblichen Verständnis der Nebenläufigkeit in Petrinetzen abweicht. 

Normalerweise werden Petri-Netze  als inherent nebenläufig verstanden, da durch die Transitionsregel zu jeder Zeit jede belibige Transition feuern kann, deren Vorbedinungen erfüllt sind. Der hier verwendete Begriff bezieht sich allerdings darauf, dass Anweisungen unabhäning voneinander, ohne sich gegenseitig zu beeinflussen, ausführbar sind. Das Feuern einer Transition kann aber den Vorbereich einer anderen Transition verändern, wodurch sie beeinflusst wird.

In dieser Arbeit heißen zwei Transitionen $t_1, t_2$ eines Petrinetzes $N$ genau dann nebenläufig, wenn ${}^\circ t_1 \cap {}^\circ t_2 = \emptyset$ gilt und eine Markierung $m_\text{conc}$ im Erreichbarkeitsgraphen $\E{N}$ existiert, in der sowohl $t_1$ als auch $t_2$ aktiviert sind.

\subsection{Unterschied zwischen Nebenläufigkeit und Parallelität}

\subsection{Folgen von Nebenläufigkeit}
Wie in Abschnitt \ref{sec:nebenl} beschrieben, muss ein Programm keine Sequenz von Anweisungen sein sondern kann auch nebenläufige Anweisungen enthalten. Diese Möglichkeit hat eine Reihe von Folgen, die es bei der nebneläufigen Programmierung zu beachten gilt.
\paragraph{Nichtdeterminismus}
Enthält ein Programm nebenläufige Anweisungen, ist die Reihenfolge der daraus resultierenden Aktivitäten nicht definiert und kann sich bei jedem Programmdurchlauf ändern. Die Ausführung ist also \emph{nichtdeterministisch}, als direkte Folge der Nebenläufigkeit~\cite{Herrtwich1989}. Erwartet man von einem Programm \emph{Determiniertheit}\footnote{Es kann durchaus sein, dass Determiniertheit in einem Programm nicht gewünscht ist. Man betrachte beispielsweise ein Programm, das einen ech-ten Zufallsgenerator beschreibt. Hier wäre Determiniertheit ein direkter Widerspruch zur Aufgabe des Programms.}, also die Eigenschaft, dass gleiche Eingaben immer zu den gleichen Ausgaben führen, ist Nichtdeterminismus in der Regel zu vermeiden~\cite{Herrtwich1989}. Liefert ein Programm bei jeder beliebigen Ausführreihenfolge das selbe Ergebnis, ist es trotz Nichtdeterminismus weiterhin determiniert, weil die Ausführreihenfolge für das Ergebnis keine Rolle spielt. 
\paragraph{Nichtreproduzierbarkeit}
Da das Wissen und die Kontrolle über die Ausführreihenfolge abgegeben wird, wird die Nachvollziebarkeit des Programmablaufs erschwert~\cite{Herrtwich1989}. Tritt beispielsweise ein Fehler auf, kann im Nachhinein nicht ermittelt werden was die Ausführreihenfolge war, die zu dem Fehler geführt hat. Dasselbe Problem ergibt sich ebenfalls, wenn ein nichtdeterminiertes Programm eine Lösung ausgibt und nachvollzogen werden soll, welche Ausführreihenfolge zu diesem Ergebnis geführt hat. Insbesondere das Testen von Software wird dadurch erschwert, da es unmöglich ist bei jedem Test die selben Bedinungen herzustellen, sodass beispielsweise Tests Fehler nicht verlässlich aufzeigen, weil diese nur bei bestimmten Ausführreihenfolgen auftreten~\cite{Herrtwich1989}. Somit muss ein Test entweder alle möglichen Ausführreihenfolgen simulieren oder es muss anderweitig sichergestellt werden, dass die Ausführreihenfolge der nebenläufigen Anweisungen keine Rolle spielt.
\paragraph{Wettkampfbedingungen}
\TODO{Reccource/Datum definieren*}{Daten} spielen in Programmen eine zentrale Rolle. Greift eine Gruppe von Anweisungen auf die selben (geteilten) Daten zu, ist die Reihenfolge des Zugriffs, wie alle anderen Aktivitäten, beliebig. Wenn mindestens eine der Anweisungen schreibend auf die geteilten Daten zugreift (diese also verändert), hängt der Zustand der Ausgabe des Programms im Allgemeinen von der Reihenfolge der Ausführung ab. Diese Situation wird \emph{Wettkampfbedinung} (engl. Race Condition) genannt~\cite{Hettel2016}. 

Erwähnenswert ist hier, dass Wettkampfbedingungen nur auftreten können, wenn die geteilten Daten nebenläufig \emph{geändert} werden. Ausschließlich lesende Zugriffe sind unkritisch, da die Daten unabhängig von der Ausführreihenfolge immer identisch sind. Wenn Daten von lesenden Anweisungen zu jederzeit in einem Zustand gefunden werden, der korrekt ist, werden sie als \emph{threadsicher} bezeichnet.

Formal können Wettkampfbedinungen auch mittels erweiterten Petri-Netzen definiert werden. Dazu betrachtet man, ob Variablen des erweiterten Petri-Netzes sich in den Lese- und Schreibmengen von nebenläufigen Transitionen überschneiden. Diese Variablen werden \emph{kritische} Variablen genannt. Die Menge der kritischen Variablen, die zur Existenz von Wettkampfbedingungen fürhen, ist gegeben durch
\begin{align*}
	V_\text{krit} = \;\bigcup_{\mathclap{\substack{s,t\, \in T\\s, t \text{ nebenläufig}}}} \;\left( \strut{S(s) \cap (S(t) \cup L(t))}\right).
\end{align*}

Ein Paar von Transitionen $(t_1, t_2)$ eines erweiterten Petri-Netzes \emph{erzeugt} eine Wettkampfbedingung, wenn die Transitionen dazu führen, dass eine Variable in die Menge der kritischen Variablen aufgenommen wird. Eine Markierung $m$ des Erreichbarkeitsgraphen $\E{N}$ eines Petri-Netzes $N$ enthält eine Wettkampfbedingung, wenn ein Paar von Transitionen $(t_1, t_2)$ existiert, das eine Wettkampfbedingung erzeugt, und $t_1$ und $t_2$ in $m$ aktiviert sind.

\begin{figure}
	\centering
	\begin{tikzpicture}[node distance=2cm,on grid, auto]
		\node[place, label=$p_1$] (p1) {};
		\node[transV, label=$t_1$, right = of p1] (t1) {};
		\node[place, label=$p_2$, tokens=1, above right = of t1] (p2) {};
		\node[place, label=$p_3$, tokens=1, below right = of t1] (p3) {};
		\node[transV,label=$t_2$, right = of p2,label=below:{\LSset{a,{\color{red}b}}{a}} ] (t2){};
		\node[transV,label=$t_3$, right = of p3,label=below:{\LSset{b}{{\color{red}b}}}] (t3){};
		\node[place, label=$p_4$, right = of t2] (p4) {};
		\node[place, label=$p_5$, right = of t3] (p5) {};
		\node[transV, label=$t_4$, below right = of p4] (t4) {};
		\node[place, label=$p_6$, right = of t4] (p6) {};
	
	
	
		\draw 
		(p1) edge[post] (t1)
		(t1) edge[post] (p2)
		(t1) edge[post] (p3)
		(p2) edge[post] (t2)
		(p3) edge[post] (t3)
		(t2) edge[post] (p4)
		(t3) edge[post] (p5)
		(p4) edge[post] (t4)
		(p5) edge[post] (t4)
		(t4) edge[post] (p6)
		;
	\end{tikzpicture}
	\caption{Ein erweitertes Petri-Netz, dessen Markierung eine Wettkampfbedingung enthält. Die Schreibmenge von $t_3$ enthält die Variable $b$, diese ist allerdings auch in der Lesemenge von $t_2$ enthalten. Die problematische Variable {\color{red}$b$} ist rot markiert. Das Auftauchen von $b$ in der Lesemenge von $t_3$ stellt kein Problem dar.}\label{fig:wettkampfpetri}
\end{figure}

Um das Konzept zu veranschaulichen, ist in Abbildung~\ref{fig:wettkampfpetri} ein erweitertes Petri-Netz gezeigt, dessen Markierung eine Wettkampfbedingung enthält. Die Plätze $p_2$ und $p_3$ enthalten je eine Markierung. Daher müssen nun die Transitionen betrachtet werden, die aktiviert sind. Diese sind $\{t_2,t_3\}$. Wie in der Abbildung rot markiert, ist ersichtlich, dass sich die Lese- und Schreibmenge der Transitionen überschneiden, da $V_\text{krit} = \{b\} \neq \emptyset$ ist. Somit enthält die gezeigte Markierung eine Wettkampfbedingung.

Ein erweitertes Petri-Netz enthält eine Wettkampfbedingung, wenn mindestens eine Markierung seines Erreichbarkeitsgraphen eine Wettkampfbedingung enthält.

Aufgabe der nebenläufigen Programmierung ist es also, Anweisungen so zu strukturieren, dass alle von nebenläufigen Anweisungen genutzten Daten threadsicher sind oder Daten, die nicht threadsicher sind, explizit synchronisiert werden. Über die Modellierung von Petri-Netzen ausgedrückt, gilt es also die Struktur eines Programms so zu definieren, dass in dem Petri-Netz, welches das Program modelliert, keine Wettkampfbedingungen existieren.

\subsection{Synchronisierung}
Um das Auftreten von Wettkampfbedingungen beim schreibenden Zugriff von nebenläufigen Anweisungen auf geteilte Ressourcen zu vermeiden, muss sichergestellt werden, dass der Endzustand nach der Ausführung der Anweisungen unabhängig von der Ausführreihenfolge der daraus resultierenden Aktivitäten ist. Dieser Vorgang wird Synchronisierung genannt.

\paragraph{Snychronisierungsanforderungen} Nach \textcite{Herrtwich1989} gibt es allgemein zwei Arten von Anforderungen an Synchronisierungmechanismen, je nachdem welche Art von Zugriff auf geteilte Ressourcen stattfindet. Sie unterscheiden dabei zwischen \emph{kausal abhäningen} Relationen und \emph{kausal unabhängigen} Relationen zwischen nebenläufigen Anweisungen. Ist Anweisung $B$ kausal abhängig von Anweisung $A$, so dürfen die Aktivitäten von $B$ erst ausgeführt werden, nachdem die Aktivitäten von $A$ abgeschlossen sind. Man schreibt $ A \to B$\todo{Überlegen wie das mit Nebenläufigkeit zusammenhängt. Insbesondere: Unterschied logische Abhängigkeit/strukturelle Abhängigkeit}. 

\paragraph{Synchronisierungsmethoden} Es gibt mehrere Möglichkeiten, die Synchronisierung zwischen nebenläufigen Anweisungen zu erreichen. \textcite{Michael1996} unterteilen Algorithmen diesbezüglich in zwei Kategorien: \emph{blockende} und \emph{nicht-blockende}. Blockende Synchronisierungsalgorithmen führen laut ihnen möglicherweise dazu, dass Rechenprozesse durch andere Rechenprozesse beliebig lang beim Zugriff auf geteilte Daten aufhalten können. Nicht-blockende Algorithmen dagegen garantieren, dass mindestens einer der beteiligten Rechenprozesse mit einer endlichen Anzahl an Aktivitäten endet.


\chapter{Analyse der blocklib}

\section{Game Loop}
Ein nahezu universelles Designelement von Spielen ist die sogenannte \emph{Game Loop}~\cite[S.~161~ff.]{Nystrom2015}. Sie ist für die Aktualisierung des Spiels zuständig und umfasst typischerweise die Aufgaben 
\begin{itemize}
  \item Berechnung des neuen Spielzustands
  \item (grafische) Ausgabe
  \item Verarbeitung der Eingaben
\end{itemize}
Die Blocklib besitzt ebenfalls eine Game Loop und führt die beschriebenen Aufgaben in der obigen Reihenfolge sequenziell aus. Eigentlich gibt es sogar zwei Game Loops, eine für die Ausführung als Server (ohne graphische Ausgabe) und eine für einen Client oder Einzelspieler. Die Game Loop der Einzelspieler-Blocklib sieht vereinfacht wie folgt aus:
\begin{lstlisting}
while(!shutdown){
	//...
	update(delta); // Berechnung des neuen Spielzustands
	//...
	render(); // grafische Ausgabe
	//...
	window.update(); // Verarbeitung der Eingaben
}
\end{lstlisting}

Die Methode \code{update(delta)} ruft ihrerseits die \code{update(delta)} Methoden der einzelnen Simulationssysteme auf.
\begin{lstlisting}
private void update(float delta) {
	//...
	Context con = Context.getInstance();
	//...
	con.getChunkManager().update(delta);
	con.getEffectManager().update(delta);
	//...
	con.getAudioManager().update();
	con.getMainScheduler().update(delta);
	con.getEntityManager().update(delta);
	con.getFluidManager().update(delta);
	//...
}
\end{lstlisting}
Bei der Berechnung des neuen Spielzustands gibt es in der Blocklib kein zentrales System, das den Zustand des letzten Loopdurchlaufs vorhält. Somit ist es möglich, dass das Verhalten der Blocklib von der Reihenfolge der \code{update(...)} abhängt. Das lässt sich einfach mit einem Beispiel veranschaulichen. 

Man nehme an, dass der \class{EntityManager} prüft, welche Chunks geladen sind, um zu ermitteln, wo ein neuer Gegner erschaffen werden soll. Der \class{ChunkManager} ist dafür zuständig, abhängig von der aktuellen Kameraposition zu bestimmen, welche Chunks geladen werden. Werden nun ein Chunks entfernt, auf dem der EntityManager einen Gegner erschaffen hat, könnte das zu unvorhergesehenem Verhalten führen, da der \class{EntityManager} erwartet, dass der erschaffene Gegner auf einem existenten Chunk erstellt wurde.

Die (mögliche) Abhängigkeit des Verhaltens von der Reihenfolge der updates, zeigt, dass sich diese Berechnungen nicht einfach parallelisieren lassen, da man den gesamten Simulationscode auf mögliche Race Conditions prüfen müsste.


\chapter{Entwicklung einer Threading API}


\subsection{Threading in Java}

\section{Anforderungen der blocklib an eine Threading API}
\subsection{Kontrolle über Threads an einer Stelle}
\input{chapters/Anforderungen/Kontrolle.tex}
\subsection{Verteilung von Berechnungen an Worker Threads}
\subsection{Verhinderung von Livenessproblemen durch Archtektur}
\subsection{Einfaches starten von Hintergrundtasks und einmaligen Aufgaben}

\section{Design der Threading API}
Bei der Nutzung von Multithreading in Spielen gibt es zwei prominente Ansätze, zum einen die Nutzung eines separaten Threads, der das Rendern übernimmt, und zum anderen den Einsatz einer Jobarchitektur.

Die Idee der Nutzung eines Renderthreads ergibt sich daraus, dass Simulation des Spiels und Anzeige zwei unabhängige Bereiche sind, die also nebenläufig ausgeführt werden können. Da Rendering sehr rechenintensiv ist, ist es also sinnvoll diese Aktivitäten in einen eigenen Thread auszulagern, damit sie den gesamten Frame für die Ausführung nutzen können.

Der Einsatz eines separaten Renderthreads ist dabei auch in der Jobarchitektur vorgesehen.

Da es sich bei der Blocklib, mit knapp $60000$ Zeilen Code und über $900$ Java Dateien, um ein großes bereits bestehendes System handelt, ist es unrealistisch, in kurzer Zeit das gesamte Programm so umzustrukturieren, dass es vollständig eine Jobarchitektur nutzt. Diese Umstrukturierung erfordert tiefes Verständnis für jedes zu ändernde System. Zudem müssen viele Bereiche grundlegend geändert werden, um die Abhängigkeiten der Systeme zu verringern beziehungsweise auszuschließen.

Die Blocklib enthält beispielsweise eine Klasse \class{Context}, die als eine Art \gls{singleton}-Implementierung eines \glspl{servicelocator}~\cite[S.~301~ff.]{Nystrom2015} dient. Da \glspl{singleton} ähnlich wie globale Variablen von überall aufgerufen werden können, wird der Code dadurch potenziell schwieriger nachvollziehbar~\cite[S.~108]{Nystrom2015}. Dadurch und durch die ebenfalls resultierende Kopplung unterschiedlicher Komponenten würden bei einer naiven Umsetzung der Jobarchitektur unvorhersehbar Wettkampfbedingungen auftreten.
Man erinnere sich, dass Wettkampfbedingungen auftreten, wenn nebenläufige Aktivitäten auf dieselben Ressourcen zugreifen. Durch den \class{Context} ist es nun schwierig, einen Überblick zu haben von wo aus auf welche Ressourcen zugegriffen wird. Die Wahrscheinlichkeit, dass also Zugriffe existieren, die zu Wettkampfbedingungen führen, ist also sehr hoch, solange dabei nicht sorgsam vorgegangen wird.

\subsection{Renderthread und Job-API}
Da es sich bei der Blocklib, mit knapp $60000$ Zeilen Code und über $900$ Java Dateien, um ein großes bereits bestehendes System handelt, ist es unrealistisch, in kurzer Zeit das gesamte Programm so umzustrukturieren, dass es vollständig eine Jobarchitektur nutzt. Diese Umstrukturierung erfordert tiefes Verständnis für jedes zu ändernde System. Zudem müssen viele Bereiche grundlegend geändert werden, um die Abhängigkeiten der Systeme zu verringern beziehungsweise auszuschließen.

Die Blocklib enthält beispielsweise eine Klasse \class{Context}, die als eine Art Singletonimplementierung eines Service Locators~\cite[S.~301~ff.]{Nystrom2015} dient. Da Singletons ähnlich wie globale Variablen von überall aufgerufen werden können, wird der Code dadurch potenziell schwieriger nachvollziehbar~\cite[S.~108]{Nystrom2015}. Dadurch und durch die ebenfalls resultierende Kopplung unterschiedlicher Komponenten würden bei einer naiven Umsetzung der Jobarchitektur unvorhersehbar Wettkampfbedingungen auftreten.
Man erinnere sich, dass Wettkampfbedingungen auftreten wenn nebenläufige Aktivitäten auf die selben Ressourcen zugreifen. Durch den \class{Context} ist es nun schwierig, einen Überblick zu haben von wo aus auf welche Ressourcen zugegriffen wird. Die Wahrscheinlichkeit, dass also Zugriffe existieren, die zu Wettkampfbedingungen führen, ist also sehr hoch, solange dabei nicht sorgsam vorgegangen wird.

\subsection{Design des Renderthreads}\label{sec:desgignRenderthread}
\begin{figure}
	\centering
	\begin{tikzpicture}[scale=1.1]
		\fill[lightgray]  (0,0) rectangle (11,1);
		\fill[lightgray] (0,-1.5) rectangle (11,-0.5);
		\fill[lightgray]  (0,1.5) rectangle (11,2.5);
		
		\node[anchor=east] at (0,2) {Thread 1};
		\node[anchor=east] at (0,0.5) {Thread 2};
		\node[anchor=east] at (0,-1) {Renderthread};
		
		
		\fill [orange,draw=lightgray] (0.5,1.5) rectangle node[black,font=\footnotesize] {Sim 1} (1.5,2.5);
		\fill [orange,draw=lightgray] (0.5,0) rectangle node[black,font=\footnotesize] {Sim 2} (1.5,1);
		\fill [orange,draw=lightgray] (1.5,1.5) rectangle node[black,font=\footnotesize] {Sim 3} (2.5,2.5);
		\fill [orange,draw=lightgray] (2.5,1.5) rectangle node[black,font=\footnotesize] {Sim 4} (3.5,2.5);
		\fill [orange,draw=lightgray] (1.5,0) rectangle node[black,font=\footnotesize] {Sim 5} (2.5,1);
		\fill [orange,draw=lightgray] (3.5,1.5) rectangle node[black,font=\footnotesize] {Sim 6} (4.5,2.5);
		\fill [orange,draw=lightgray] (2.5,0) rectangle node[black,font=\footnotesize] {Sim 7} (3.5,1);
		\fill [orange,draw=lightgray] (3.5,0) rectangle node[black,font=\footnotesize] {Sim 8} (4.5,1);
		
		\fill [orange,draw=lightgray] ($(4.5,0)+(0.5,1.5)$) rectangle node[black,font=\footnotesize] {Sim 1} ($(4.5,0)+(1.5,2.5)$);
		\fill [orange,draw=lightgray] ($(4.5,0)+(0.5,0)$) rectangle node[black,font=\footnotesize] {Sim 2} ($(4.5,0)+(1.5,1)$);
		\fill [orange,draw=lightgray] ($(4.5,0)+(1.5,1.5)$) rectangle node[black,font=\footnotesize] {Sim 3} ($(4.5,0)+(2.5,2.5)$);
		\fill [orange,draw=lightgray] ($(4.5,0)+(2.5,1.5)$) rectangle node[black,font=\footnotesize] {Sim 5} ($(4.5,0)+(3.5,2.5)$);
		\fill [orange,draw=lightgray] ($(4.5,0)+(1.5,0)$) rectangle node[black,font=\footnotesize] {Sim 4} ($(4.5,0)+(2.5,1)$);
		\fill [orange,draw=lightgray] ($(4.5,0)+(3.5,1.5)$) rectangle node[black,font=\footnotesize] {Sim 6} ($(4.5,0)+(4.5,2.5)$);
		\fill [orange,draw=lightgray] ($(4.5,0)+(2.5,0)$) rectangle node[black,font=\footnotesize] {Sim 7} ($(4.5,0)+(3.5,1)$);
		\fill [orange,draw=lightgray] ($(4.5,0)+(3.5,0)$) rectangle node[black,font=\footnotesize] {Sim 8} ($(4.5,0)+(4.5,1)$);
	
		\fill [magenta] (0.5,-0.5) rectangle node[black]{Render $n-1$} (4.2,-1.5);
		\fill [magenta] (5,-0.5) rectangle node[black]{Render $n$} (8.7,-1.5);
		
		\node at (2.5,3.5) {Frame $n$};
		\node at (7,3.5) {Frame $n+1$};
		
		\draw  (0.5,3) rectangle (4.5,-2);
		\draw  (5,3) rectangle (9,-2);
	\end{tikzpicture}
	\caption{Darstellung des Designs der Multithreading Architektur der Blocklib. Es existiert ein gesonderter Renderthread, der einen großen Teil der Rechenzeit während eines Frames nutzt.}\label{fig:optimalArchitecture}
	\todo{Caption}
\end{figure}

Da der Performancegewinn, in Form von \ac{fps}, bei der Nutzung eines Renderthreads als hoch zu erwarten ist, soll dieser in die Blocklib integriert werden. Um die Anforderungen von Kapitel~\ref{sec:anforderungen} zu Erfüllen, wird ein Jobsystem implementiert. Da die Blocklib OpenGL als Grafik Schnittstelle nutzt und OpenGL, wie in Abschnitt~\ref{sec:gamesJobsystem} beschrieben, Multithreading nicht unterstützt, kann das Rendering selbst nicht nebenläufig durchgeführt werden.

Der Renderthread kann einerseits als Teil des Jobsystems designt werden, andererseits gibt es die Möglichkeit den Renderthread von diesem zu trennen. Ist der Renderthread Teil des Jobsystems, kann dieser voll zur Bearbeitung von Jobs mitgenutzt werden. Da üblicherweise die Anzahl der Threads der Anzahl der Prozessorkerne entspricht, kann man so automatisch immer die Leistung aller Prozessorkerne nutzen. Trennt man den Thread dagegen ab, entsteht die Problematik zu entscheiden, wann welche Anzahl von Threads genutzt wird, um möglichst alle Kerne zu nutzen, aber gleichzeitig zu verhindern, dass sich die Threads in der Ausführung gegenseitig behindern. Andererseits gestaltet sich die Implementierung eines getrennten Renderthreads als deutlich einfacher und intuitiver~\cite{Tatarchuk2014}, im Gegensatz zur Jobsystemintegration.

Nach Messungen ergibt sich, dass das Rendering in etwa \SI{50}{\percent}\todo{messen} der Zeit eines Frames benötigt. Daher bietet es sich an, den Renderthread zu separieren. Der prinzipiell mögliche Performancegewinn durch die Integration in das Jobsystem ist ausgeschlossen, da das Rendering selbst sie gesamte Rechenleistung des Kerns jeden Frame beansprucht. Da das Rendering im sequentialisierten Fall circa \SI{50}{\percent} der Zeit benötigt, entspricht das annähernd \SI{100}{\percent}, sobald Simulation und Rendering in zwei Threads nebenläufig ausgeführt werden. Wird die Simulation selbst nebenläufig durchgeführt verstärkt sich dieser Effekt, wodurch die Framezeit, durch das Rendering bestimmt wird. Damit lässt sich auch die Anzahl der Threads des Jobsystems bestimmen, indem die Anzahl der Jobthreads um eins verringert wird ohne dadurch Prozessorkerne ohne Arbeit zu verursachen.

Da der Renderthread auf Daten des Spiels zugreift, um die sichtbaren Elemente zu zeichnen, muss sichergestellt werden, dass diese Daten keinen Wettkampfbedingungen unterliegen. In der Spieleentwicklung ist es üblich einen \emph{Spielzustand} (engl. Game State) zu definieren, der während der Simulation in jedem Frame angepasst wird. Der Teil des Zustandes auf den der Renderthread zugreift muss also konstant sein. Während der Simulation wird dieser allerdigs verändert. Eine Möglichkeit, dieses Problem zu beheben, besteht darin einen \emph{Double Buffer}~\cite[S.~143]{Nystrom2015} zu nutzen, um den gesamten Spielzustand zwischenzuspeichern~\cite{Tatarchuk2014}. Die Blocklib ist nicht mit diesem Hintergrund entwickelt worden, weswegen es keine einfache Möglichkeit gibt, den gesamten Spielzustand auf diese Weise zwischenzuspeichern. Die Objekte, die den Spielzustand darstellen, sind über die Blocklib hinweg verteilt. Um dennoch einen Renderthread nutzen zu können, müssen die Objekte identifiziert werden, die für das Rendering benötigt werden. Für diese Objekte muss dann ein geeigneter Double Buffer erzeugt werden, sodass die Daten aus Renderthread-Sicht konstant sind.

Somit ist eine nebenläufige Architektur, wie sie in Abbildung~\ref{fig:optimalArchitecture} dargestellt wird, erstrebenswert. Es gibt einen Renderthread, der die von der Simulation im vorherigen Frame berechneten Objekte zeichnet. Alle anderen verfügbaren Hardwarethreads können in dem Jobsystem für die Simulation genutzt werden. 

\subsection{Design des Jobsystems}\label{sec:desgignJobsystem}
\begin{figure}
	\centering
	\begin{tikzpicture}[scale=1.1]
		\fill[lightgray]  (0,0) rectangle (11,1);
		\fill[lightgray] (0,-1.5) rectangle (11,-0.5);
		\fill[lightgray]  (0,1.5) rectangle (11,2.5);
		
		\node[anchor=east] at (0,2) {Thread 1};
		\node[anchor=east] at (0,0.5) {Thread 2};
		\node[anchor=east] at (0,-1) {Renderthread};
		
		
		\fill [orange,draw=lightgray] (0.5,1.5) rectangle node[black] {Simulation $n$} (4.5,2.5);
		\fill [orange,draw=lightgray] (0.5,0) rectangle node[black,font=\footnotesize] {Job 1} (1.5,1);
		\fill [orange,draw=lightgray] (1.5,0) rectangle node[black,font=\footnotesize] {Job 2} (2.5,1);
		\fill [orange,draw=lightgray] (2.5,0) rectangle node[black,font=\footnotesize] {Job 3} (3.5,1);
	
		\fill [orange,draw=lightgray] (5,1.5) rectangle node[black] {Simulation $n+1$} (9,2.5);
		\fill [orange,draw=lightgray] (5,0) rectangle node[black,font=\footnotesize] {Job 1} (6,1);
		\fill [orange,draw=lightgray] (6,0) rectangle node[black,font=\footnotesize] {Job 2} (7,1);
		\fill [orange,draw=lightgray] (7,0) rectangle node[black,font=\footnotesize] {Job 3} (8,1);
	
		\fill [magenta] (0.5,-0.5) rectangle node[black]{Render $n-1$} (4.2,-1.5);
		\fill [magenta] (5,-0.5) rectangle node[black]{Render $n$} (8.7,-1.5);
		
		\node at (2.5,3.5) {Frame $n$};
		\node at (7,3.5) {Frame $n+1$};
		
		\draw  (0.5,3) rectangle (4.5,-2);
		\draw  (5,3) rectangle (9,-2);
	\end{tikzpicture}
	\caption{Design der Threadingarchitektur der Blocklib}\label{fig:plannedArchitecture}
	\todo{Caption}
\end{figure}

Aufgrund des Umfangs der Blocklib und der Unübersichtlichkeit an einigen Stellen kann das Jobsystem nicht, wie in Abbildung~\ref{fig:optimalArchitecture} gezeigt, umfassend integriert werden. Um die Anforderungen von Kapitel~\ref{sec:anforderungen} zu erfüllen, wird daher ein Jobsystem implementiert, das nur an ausgewählten Stellen mit der Blocklib konsolidiert wird und ansonsten für die zukünftige Nutzung bereitsteht.

Da das Jobsystem keine vollständige Integration in die Blocklib erfährt, verändert sich die Architektur konzeptuell leicht. Diese Änderung ist in Abbildung~\ref{fig:plannedArchitecture} zu sehen. Anstatt die gesamte Simulation in viele kleine Jobs zu zerlegen, bleibt eine sequentialisierte Simulation bestehen, die dann aber die Möglichkeit besitzt, weitere nebenläufige Jobs zu starten. Mit dieser Architektur ist es möglich, die Blocklib inkrementell zu der in Abbildung~\ref{fig:optimalArchitecture} gezeigten Architektur umzuwandeln, indem in zukünftigen Arbeiten immer mehr pseudosequentialisierte Anweisungen der Simulation als nebenläufige Jobs definiert werden, bis sie vollständig aus Jobs besteht.

Java bietet mit dem Interface \class{ExecutorService} bereits eine gute Schnittstellendefinition für ein Jobsystem. Daher baut das Design des Jobsystems der Blocklib auf diesem Interface auf. 

\begin{figure}
	\includesvg[width=\textwidth]{GrobesDesign.svg}
	\caption{Struktur }\label{fig:GrobesDesign}
	\todo{Caption}
\end{figure}

In Abbildung~\ref{fig:GrobesDesign} ist die Struktur des Designs für das Jobsystem der Blocklib dargestellt. Es wird ein Interface \class{BlocklibExecutorService} definiert, das von dem Interface \class{ScheduledExecutorService} der Java Bibliothek abgeleitet ist. Das Interface wird so erweitert, dass die verschiedenen \code{submit(...)} Methoden jeweils Objekte vom Typ \class{CompletableFuture} zurückgeben. Die \code{schedule(...)} Methoden geben jeweils ein \class{ScheduledCompletableFuture} Objekt zurück, das später noch näher erläutert wird.

Das Interface \class{BlocklibExecutorService} wird durch die Klasse \class{BlocklibExecutor} implementiert. Sie nutzt zur Durchführung von Jobs den von der Java Bibliothek definierten \class{ScheduledThreadPoolExecutor}. Um die für die Rückgabewerte nötigen \class{CompletableFuture} Objekte zu erzeugen, wird eine Klasse \class{CompletableFutueWrapper} erstellt. Eine vollständige Auflistung der von den drei Interfaces bereitgestellten Methoden ist in Anhang~\ref{appendix:BlocklibExecutorService} zu finden. 

Die API des Jobsystems bietet verschiedene Methoden für zum Starten von nebenläufigen Anweisungen. Mittels der \code{submit(...)} Methoden können Anweisungen definiert werden, die sobald wie möglich ausgeführt werden. Da diese Methoden ein \class{CompletableFuture} Objekt zurückgeben, lassen sich über die dort definierten Methoden einfach nachgelagerten nebenläufige Anweisungen definieren, die abhängig von der Vollendung der ursprünglichen Anweisung ausgeführt werden. Mit den \code{schedule(...)} Methoden wird analog dazu eine Möglichkeit geboten Anweisungen zu definieren, die nach Ablauf eines bestimmten Zeitintervalls nebenläufig ausgeführt werden. Mittels \code{scheduleAtFixedRate(...)} und \code{scheduleWithFixedDelay(...)} können periodisch durchzuführende Anweisungen zur Ausführung gebracht werden.
\subsection{Etablierung einer Kompositionroot}

\section{Implementierung der Threading API}
Wie das Design, ist auch die Implementierung der nebenläufigen Architektur prinzipiell in zwei Bereiche unterteilt, die Implementierung des Renderthreads und die Implementierung der Job API.

Um einen Renderthread in die Architektur der Blocklib zu integrieren, muss besonders darauf geachtet werden, dass Wettkampfbedingungen vermieden werden. Wie in Abschnitt~\ref{sec:desgignRenderthread} kann dies unter anderem durch die Nutzung von Double Buffern gelingen. Folgend wird die Implementierung des Renderthreads beschrieben. Im darauffolgenden Abschnitt wird die Implementierung des Jobsystems beschrieben.
\paragraph{Schattenflackern}
Bei der parallelen Ausführung des Renderthreads ist bei der Darstellung der Schatten ein Flackern zu erkennen. Die Tatsache, dass das Flackern in irregulären Intervallen auftritt, lässt auf das Vorhandensein einer Race Condition schließen. Diese Vermutung kann dadurch bestätigt werden, dass das Flackern nichtmehr auftritt, sobald \code{update()} und \code{render()} sequentialisiert werden.

Da Variablen zeitgleich von verschiedenen Threads bearbeitet werden, ist die Suche nach dem Ursprung des Flackerns schwierig. Da der Fehler nur bei nebenläufiger Ausführung auftritt, müssen die problematischen Stellen auf \code{update()} und \code{render()} verteilt sein. Da der Fehler visuell sichtbar ist, lässt sich das Problem über die Anweisungen in \code{render()} nachverfolgen. 

Eine naive Herangehensweise mittels Debugging ist nahezu unmöglich, da das Flackern zufällig auftritt. Zuerst muss also eine Möglichkeit gefunden werden, um das Auftreten des Flackerns programmatisch zu erkennen.

In der \class{Configuration}-Klasse gibt es die Variable \const{SHOW_SHADOW_MAP_FRAME}. Ist diese auf \code{true} gesetzt, wird in der linken oberen Ecke des Fensters ein Feld angezeigt, das die sogenannte \class{ShadowMap} zeigt. Abbildung~\ref{fig:ShadowMap} zeigt einen Screenshot der Blocklib mit \code{SHOW_SHADOW_MAP_FRAME = true}. Die \class{ShadowMap} wird genutzt, um zu berechnen, an welchen Stellen Schatten gerendert werden.
\begin{figure}
	\caption{Screenshot}\label{fig:ShadowMap}
\end{figure}
\textcite{Ebbinger2018} beschreibt, wie genau das Rendering von Schatten in der Blocklib umgesetzt ist. Wie der Name der Klasse \class{ShadowMap} vermuten lässt, wird für die Berechnung der Schatten das sogenannte \emph{Shadow Mapping} verwendet. Dazu wird eine zweite Kamera implementiert, die die Spielwelt aus der Sicht einer Lichtquelle, im Fall der Blocklib die Sonne, betrachtet. Die so gerenderte Sicht wird dann genutzt, um zu berechnen ob ein Fragment das erste Hindernis aus Sicht der Lichtquelle ist. Ist dies der Fall, so wird das Fragment beleuchtet, ansonsten ist etwas anderes vor dem Fragment und es befindet sich im Schatten.

Wird die \class{ShadowMap} mitangezeigt, so kann man erkennen, dass diese sich zeitgleich mit dem Flackern verschiebt. Die Position der \class{ShadowMap} wird über die Klasse \class{ShadowBounds} bestimmt. Die Position muss in zwei Fällen geändert werden, zum einen, wenn die Sonne sich bewegt, also, wenn der Tag-Nacht-Zyklus aktiv ist, zum anderen, wenn sich die Spielerkamera bewegt, da die Darstellung von Schatten immer mit dem Sichtfeld des Spielers übereinstimmen muss.

Die Ausgabe der Position der \class{ShadowBounds} während des Spiels bei deaktiviertem Tag-Nacht-Zyklus und stillstehender Spielerkamera bestätigt, dass diese Position sich tatsächlich zeitgleich mit dem Auftreten des Flackerns verändert. Das Auftreten der Wettkampfbedingung lässt sich also erkennen, indem eine Änderung der Position der \class{ShadowBounds} in aufeinanderfolgenden Frames erkannt wird.

Nun gilt es die Ursache der Positionsänderung ausfindig zu machen. Die Position der \class{ShadowBounds} hängt selbst von mehreren Variablen ab. Verfolgt man den Verlauf der Änderungen, finden sich die folgenden Abhängigkeiten:

\begin{tabular}{ll}
	\class{ShadowMap} &$\to$ \code{ShadowBounds.update}\\
	& $\to$ \class{LightViewMatrix}\\
	& $\to$ \code{DayNightLighting.getSunUp()} \\
	& $\to$ \code{DayNightLighting.position}\\
	& $\to$ \code{DayNightLightig.updateLightPosition(float, boolean)}
\end{tabular}

\code{DayNightLightig.updateLightPosition(float, boolean)} wird nicht im Renderthread ausgeführt, sondern während \code{update()} in einem anderen Thread. Der relevante Abschnitt der Methode ist 
\begin{lstlisting}[]
private void updateLightPosition(float progress, boolean day) {
	// ...
	position = new Vector3f(direction.x, direction.y, direction.z);
	position.scale(SUN_HEIGHT); (*\label{lst:updateLightPosition:scale}*)
	// let sun stay relative to the player
	position = Vector3f.add(position, Context.getInstance().getCamera().getPosition(), new Vector3f());
}
\end{lstlisting}
Wird also der Wert von \code{DayNightLighting.position} im Renderthread ausgelesen, während die Methode in Ausführung ist, beispielsweise während Zeile~\ref{lst:updateLightPosition:scale}, so enthält \code{DayNightLighting.position} einen vollkommen falschen Wert. Hier existiert also die gesuchte Wettkampfbedingung.
\subsection{Umstellung auf zustandsloses Rendering}
In der Blocklib implementieren alle zu zeichnenden Elemente das Interface \class{Renderable}. Dieses definiert die Funktionalität, die zum Zeichnen eines Elements notwendig ist, sowie die Methoden \code{show()} und \code{hide()}. Das Rendersystem ist so aufgebaut, dass ein Element nach einem Aufruf von \code{show()} solange gezeichnet wird, bis \code{hide()} aufgerufen wird. Dies wird erreicht, indem eine Datenstruktur im \class{MasterRenderer} alle Renderables speichert. Diese zustandsbehaftete Zeichenmethode birgt Vor- und Nachteile.

\todo{itemize mit + und - ?}
\begin{itemize}
	\item[$+$] Da \code{show()} nur einmal aufgerufen werden muss, können auch Systeme ohne Updatemethode das Zeichnen von Elementen veranlassen.
	\item[$+$] Da die Elemente gespeichert sind, müssen sie nicht jedes Mal neu hinzugefügt werden. Das verringert den Rechenbedarf.
	\item[$-$] Werden Renderables häufig ausgetauscht, müssen die alten Elemente jedes Mal entfernt werden.
	\item[$-$] Da die Einführung eines parallelen Renderthreads ansonsten zu Race Conditions führen würde, muss diese Datenstruktur aus Sicht des Renderthreads während des Zeichnens unverändert sein. Wie beschrieben, wird dazu ein Double Buffer eingesetzt. Bei einem zustandsbehafteten Double Buffer müssen beim Swap die Elemente des einen Buffers tatsächlich in den anderen Buffer kopiert werden. Das erfordert Zeit, die nicht parallelisiert werden kann, da das Wechseln des Buffers synchronisiert sein muss.
	\item[$-$] Die Existenz von paarweise aufzurufenden Funktionen birgt die Gefahr, dass der zweite Aufruf vergessen wird. Wie bei \code{malloc()} und \code{free()} in C entsteht durch einen fehlenden Aufruf von \code{hide()} ein Speicherleck. Zudem wird dann die Anzahl der zu zeichnenden Elemente immer größer und der Rendervorgang wird verlangsamt. Des weiteren werden möglicherweise Renderables gezeichnet, die nicht gezeichnet werden sollen.
\end{itemize}

Um den Zeichenaufwand zu veringern, implementiert die Klasse \class{ChunkManager} das sogenannte \emph{Frustum Culling}. Es wird berechnet, welche Chunks sich im Sichtfeld der Kamera befinden. Nur diese Chunks sollen gezeichnet werden. Dazu entfernt der ChunkManager jeden Frame alle Chunks mittels \code{hide()} aus der Datenstruktur der zu zeichnenden Elemente und fügt nur die als sichtbar ermittelten Chunks wieder ein. Die Chunks machen mit etwa 
% 334*2/(334*2+4+73+142)
75 \% bis
% 529*2/(529*2+4+73+142)
82 \% einen Großteil aller Rendereables aus. Es wird also bereits ein Großteil der zu zeichnenden Elemente in jedem Frame neu hinzugefügt. Des weiteren gibt es in der Blocklib bis jetzt keine Klasse, die Renderables zu der Datenstruktur hinzufügt, aber keine Updatemethode besitzt. Somit werden die Gefahr von vergessenen \code{hide()} Aufrufen und der Kopieraufwand des Double Buffers als wichtiger eingeschätzt, als die oben beschriebenen Vorteile.

Die Datenstruktur ist nun wie folgt implementiert. Die Methode \code{hide()} entfällt ersatzlos. Stattdessen wird nach dem Swap des Double Buffers der Renderables, der Buffer für die nächsten zu zeichnenden Elemente geleert, sodass Renderables automatisch nicht mehr gezeichnet werden, wenn sie nicht mehr hinzugefügt werden. Die Methode \code{show()} wird zu \code{draw()} umbenannt, um zu signalisieren, dass es sich um einen einmaligen Vorgang handelt. Alle bisherigen Aufrufe von \code{show()} werden durch \code{draw()} in Updatemethoden ersetzt, sodass das Verhalten zum vorherigen Stand identisch ist.
\section{Integration und Performanceanalyse}

\chapter{Fazit und Ausblick}

\printnoidxglossaries

\printbibliography[title={Literaturverzeichnis},heading=bibintoc,notkeyword=online]

\printbibliography[title={Quellenverzeichnis},heading=bibintoc,keyword=online] 


\end{document}